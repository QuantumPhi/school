\documentclass[a4paper]{article}

\usepackage{amsmath, amssymb}
\usepackage{bm}
\usepackage{booktabs}
\usepackage{dcolumn}
\usepackage{graphicx}
\usepackage{siunitx}
\usepackage{pgfplots}
\usepackage{pgfplotstable}

\def\mean#1{\left< #1 \right>}

\title{Investigating the Fracturing of Raw Spaghetti}
\date{1 September 2015}
\author{Tarik Onalan}

\begin{document}
    \maketitle
    \section{Introduction}
        \subsection{Purpose}
            To understand how the separation of the applied forces on the spaghetti affect
            the length of the secondary fracture.
        \subsection{Hypothesis/Prediction}
            I predict that as the distance between the two forces decreases, the length of
            the broken spaghetti pieces will decrease, as there will be less space for a
            cascading fracture to take place.
        \subsection{Variables}
            \textbf{Independent Variable}
            \begin{itemize}
                \item Width between forces [\si\cm]
            \end{itemize}
            \textbf{Dependent Variables}
            \begin{itemize}
                \item Length of broken spaghetti pieces [\si\cm]
            \end{itemize}
            \textbf{Controlled Variables}
            \begin{itemize}
                \item Length of starting spaghetti [\SI{25.0}{\cm}]
                \item Width of starting spaghetti [\SI{0.2}{\cm}]
                \item Mass of starting spaghetti [\SI{5.4}{\g}]
            \end{itemize}
    \section{Materials}
        \begin{itemize}
            \item Spaghetti $\cdot 20$
            \item Rotatable clamps $\cdot 2$
            \item Ruler $\cdot 1$
            \item Scale $\cdot 1$
        \end{itemize}
    \section{Procedure}
        \begin{enumerate}
            \item (CONTROL) Measure length of spaghetti
            \item (CONTROL) Measure width of spaghetti
            \item (CONTROL) Measure mass of spaghetti
            \item Pick a spaghetti, check it is within margin of error (\SI{0.05}{\cm})
            \item Measure distance between clamps (25, 20, 15, and \SI{10}{\cm})
            \item Clamp spaghetti
            \item Turn clamps at the same speed until spaghetti breaks
            \item Measure length of broken spaghetti pieces, if any (else record \SI{0.0}{\cm})
            \item Repeat steps 4-8 as necessary for data collection
            \item Repeat step 9 with remaining distances
        \end{enumerate}
        \resizebox{\linewidth}{!}{
            \includegraphics{./figure.png}
        }
    \section{Data}
        \resizebox{\linewidth}{!}{
            \pgfplotstabletypeset[
                multicolumn names,
                display columns/0/.style={
                    column name=Separation,
                    column type={S},string type},
                display columns/1/.style={
                    column name=Trial 1,
                    column type={S},string type},
                display columns/2/.style={
                    column name=Trial 2,
                    column type={S},string type},
                display columns/3/.style={
                    column name=Trial 3,
                    column type={S}, string type},
                display columns/4/.style={
                    column name=Trial 4,
                    column type={S}, string type},
                display columns/5/.style={
                    column name=Trial 5,
                    column type={S}, string type},
                display columns/6/.style={
                    column name=Average,
                    column type={S}, string type},
                columns/t1/.append style={
                    postproc cell content/.append style={
                        /pgfplots/table/@cell content/.add={}{$\pm0.05$}
                    }
                },
                columns/t2/.append style={
                    postproc cell content/.append style={
                        /pgfplots/table/@cell content/.add={}{$\pm0.05$}
                    }
                },
                columns/t3/.append style={
                    postproc cell content/.append style={
                        /pgfplots/table/@cell content/.add={}{$\pm0.05$}
                    }
                },
                columns/t4/.append style={
                    postproc cell content/.append style={
                        /pgfplots/table/@cell content/.add={}{$\pm0.05$}
                    }
                },
                columns/t5/.append style={
                    postproc cell content/.append style={
                        /pgfplots/table/@cell content/.add={}{$\pm0.05$}
                    }
                },
                columns/avg/.append style={
                    postproc cell content/.append style={
                        /pgfplots/table/@cell content/.add={}{$\pm0.05$}
                    }
                },
                every head row/.style={
                    before row={\toprule},
                    after row={
                        \si\cm & \si\cm & \si\cm & \si\cm & \si\cm & \si\cm & \si\cm\\
                        \midrule}
                },
                every last row/.style={after row=\bottomrule}
            ]{data.dat}
        }

        \begin{tikzpicture}
            \begin{axis}[
                title={Length of Spaghetti Piece Relative to Breaking Force Separation},
                scale=1.75,
                xlabel={Separation [\si\cm]},
                ylabel={Length [\si\cm]},
                xmin=9.0, xmax=26.0,
                ymin=0.0, ymax=3.0,
                x dir=reverse,
                legend pos=north east,
                ymajorgrids=true,
                grid style=dashed
            ]
                \addplot [
                    color=blue,
                    only marks
                ] plot [
                    error bars/.cd,
                        x dir=both,
                        y dir=both,
                        x explicit,
                        y explicit,
                        x fixed=0.05,
                        y fixed=0.05,
                ] table [
                    x=sep,
                    y=avg,
                ]{data.dat};

                \addplot [thick, red] table [y={create col/linear regression={y=avg}}]{data.dat};

                \draw (axis cs:24.95,2.45) -- (axis cs:10.05,-0.05);
                \draw (axis cs:25.05,2.35) -- (axis cs:9.95,0.05);

                \addlegendentry{Average}
                \addlegendentry{$\pgfmathprintnumber{\pgfplotstableregressiona} \cdot x \pgfmathprintnumber[print sign]{\pgfplotstableregressionb}$}
            \end{axis}
        \end{tikzpicture}
    \section{Explanations}
        \subsection{Uncertainty}
            There are uncertainty bars for both $x$ and $y$ axes, though they are difficult
            to see on the $x$ axis due to scaling issues. Uncertainty was assumed to be
            \SI{0.05}{\cm}, as the ruler used could only measure within the nearest
            \SI{0.1}{\cm}, meaning that the \SI{0.1}{\cm} region that was $50\%$ offset
            from the ruler would be least certain.
        \subsection{Intercepts}
            There is only one intercept on the graph, and that is when, at a separation of
            \SI{10.0}{\cm}, the spaghetti does not fracture into any extra pieces during any
            of the five trials.
        \subsection{Slope}
            The slope is positive ($0.15$), meaning that as the width between the two clamps
            increases, the length of the secondary broken spaghetti piece will also increase.
    \section{Calculations}
        \subsection{Average}
            \begin{equation}
                \mean{x_i^*}
            \end{equation}
            \begin{equation}
                \frac{\displaystyle\sum_i{x_i}}{i}
            \end{equation}
            \begin{equation}
                \frac{x_1+x_2+...+x_{i-1}+x_i}{i}
            \end{equation}
        \subsection{Error and Minimization for Line of Best Fit}
            \centerline{Error was calculated using sum squared error, which is defined as follows:}
            \begin{equation}
                E=\displaystyle\sum_{i=1}^n{(y_i-f(x_i))^2}
            \end{equation}
            \begin{center}
                While a program carried out the minimization of the function, the basic
                premise of minimization is to ``follow the gradient,'' so to speak. The
                program follows the gradient
            \end{center}
            \begin{equation}
                E'=\frac{\partial}{\partial x}(\displaystyle\sum_{i=1}^n{(y_i-f(x_i))^2})
            \end{equation}
            \begin{center}
                similar to Euler's method for solving differential equations. Minimizing
                the function is simply following the ``negative'' slope to the local minima
                of the function.
            \end{center}
    \section{Conclusion}
        My hypothesis, that the length of the secondary break will shorten as the separation of
        the applied forces on the spaghetti was reduced, was supported by the data. When the separation
        was \SI{25.0}{\cm}, the secondary break averaged a length of \SI{2.4}{\cm}. At \SI{20.0}{\cm},
        the average dropped to \SI{0.4}{\cm}. \SI{15.0}{\cm} had an average of \SI{0.2}{\cm}, and,
        the smallest separation, \SI{10.0}{\cm}, produced an average of \SI{0.0}{\cm}, meaning that
        no secondary pieces were produced from the breaking of the spaghetti through the five trials.
        This makes sense, as when the distance between the two clamps was decreased, there was less
        ``spaghetti'' available for a cascading fracture to take place. A cascading fracture, in this
        case, is the secondary fracturing of the spaghetti due to regions (namely the centre region)
        of the spaghetti straightening before other regions (like the bases). In regards to the data,
        however, I believe there is too much random error at play to make an outright statement of the
        validity of my hypothesis. Spaghetti has too much inherent randomness to make it a reliable
        object of measurement. A spaghetti could have microscopic weaknesses that predispose it to
        breaking from certain locations: perhaps another spaghetti was slightly more dense; it would
        have been impossible to tell with my measuring devices. Additionally, as I was the one turning
        the clamps, I was introducing error into the measurements; humans are not precise. However,
        because the data shows such an overwhelming trend in support of my hypothesis, I believe I can
        claim that my hypothesis is valid.

        The design of the investigation was not incredible by any stretch of the mind, but given that
        it was quite simple, the implementation, I believe, was up-to-par with the standards I had in
        mind. That said, I was not impressed by the error accrued throughout the investigation. As I
        said before, the spaghetti was a central source of randomness throughout the investigation.
        A particular with my investigation, the clamps, also introduced randomness, but this was the
        product of a compromise instead of ignorance. As I was investigating the secondary breakage
        of spaghetti (around the centre), I wanted to make sure that I was avoiding contact with the
        centre region of the spaghetti at the time of breakage. As a result, I had to resort to rotating
        clamps as a compromise, as even though I still control their movement, I felt that they would
        provide a more consistent torque than my hands would. The error introduced by both sources, in
        the end, were mostly random; the results could essentially have gone either way.

        One improvement that comes to mind is replacing the spaghetti with thin plastic sticks; the
        material would be more uniform (mass, dimensions, elasticity), and would thus introduce less
        random error into test results. However, this would undoubtedly be more expensive, so whether
        or not it would be feasible is unknown.
\end{document}
