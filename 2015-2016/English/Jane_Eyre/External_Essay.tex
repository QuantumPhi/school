%!TEX program = xelatex

\documentclass[a4paper,12pt]{article}

\usepackage{polyglossia}
\setdefaultlanguage{english}

\usepackage[backend=biber,style=mla]{biblatex}
\usepackage[autostyle]{csquotes}
\usepackage{fancyhdr}
\usepackage{fontspec}
\usepackage[margin=1.0in]{geometry}
\usepackage{lineno}
\usepackage{setspace}

\addbibresource{External_Essay.bib}
\doublespacing
\MakeOuterQuote{"}
\setmainfont{Times New Roman}

\pagestyle{fancy}
\rhead{Onalan \thepage} 
\renewcommand{\headrulewidth}{0pt} 
\renewcommand{\footrulewidth}{0pt} 
\setlength\headsep{0.333in}

\begin{document}
    \begin{flushleft}
        Tarik Onalan\\
        16 October 2015\\
        \begin{center}
            On the conclusion of \textit{Jane Eyre}
        \end{center}

        Brontë's conclusion to Jane Eyre is the rather gray appendage of an otherwise black-and-
        white novel. From the direct, almost jarring tone of the previous chapter---full to the brim
        with brisk exchanges between Jane and Mr. Rochester---Brontë retracts to a simple,
        straightforward statement: "Reader, I married him" \cite[521]{brontec}. Brontë develops Jane's
        character as a woman who strives to be independent: to rely on no one but herself; yet, the
        swansong of Jane's journey is marked by none other than the single most binding contract
        between two individuals: marriage. Moreover, the clash between heat and cold, fire and ice,
        passion and asceticism that remains a recurring theme throughout the novel, is strangely
        absent; Brontë uses absolutely no imagery pertaining to heat or cold in the entirety of the
        final chapter, which reveals Brontë's main point: unity and harmony are more desirable than
        discord and fragmentation. Thus, in an odd, almost twisted way, Brontë's sedated conclusion
        to the gripping novel that is Jane Eyre is the perfect finale, much like the bitter garnish
        on a sweet dish.

        Almost immediately, it becomes clear through Brontë's change in syntax that the conclusion
        is almost set a step apart from the rest of the novel. The previous chapter contains, as
        previously mentioned, a multitude punctuated, pithy dialogues between Jane and Mr.
        Rochester:
        \begin{quote}
            `Ah! Jane. But I want a wife.'\\
            `Do you, sir?'\\
            `Yes: is it news to you?'\\
            `Of course: you said nothing about it before.'\\
            `Is it unwelcome news?'\\
            `That depends on circumstances, sir—on your choice.'\\
            `Which you shall make for me, Jane. I will abide by your decision.'\\

            \cite[516]{brontec}
        \end{quote}
        Brontë’s short, curt syntax both increases the perceived pace of the text and the tension
        present in the tone of the text. In marked contrast, the final chapter has a longer, more
        flowing syntax, with parallelism and extensive use of colons and commas:
        \begin{quote}
            A quiet wedding we had: he and I, the parson and clerk, were alone present. When we got
            back from church, I went into the kitchen of the manor-house, where Mary was cooking the
            dinner, and John cleaning the knives.
            
            \cite[521]{brontec}
        \end{quote}
        Here, Brontë’s syntax has the opposite effect of the quick bursts of dialogue present only
        five pages before: the pace of the text is slowed, and the tone is much calmer and quieter.
        Indeed, Brontë’s near omission of dialogue in the concluding chapter seems to reveal a
        tantalizing insight: there is little to be gained from binary interpretations, both on the
        individual level and the interpersonal level. Instead, there must be compromise: not black
        and white, fire and ice, light and dark, but a simple gray, mundane water, or dull twilight.

        Also of importance is Brontë’s use of contrast—or lack thereof—in the concluding chapter.
        Where in the rest of her novel, there are recurring themes of contrast, as in the contrast
        between Mr. Rochester, with his "black eyes darting sparks" \cite[236]{brontec}, and St. John, with
        his "blue-pictorial looking eyes" \cite[399]{brontec}; in the concluding paragraph, Brontë is much
        more direct, writing through the lens of Jane herself:
        \begin{quote}
            My tale draws to its close: one word respecting my experience of married life, and one
            brief glance at the fortunes of those whose names have most frequently recurred in this
            narrative, and I have done.
            
            \cite[523]{brontec}
        \end{quote}
        This, interestingly, has a simple, but important side effect: Jane’s thoughts seem more
        direct, becoming, in a sense, the centerpiece of the writing. This, as opposed to Jane’s
        dissociative descriptions earlier in the novel, where she merely describes the events
        transpiring around her:
        \begin{quote}
            I had put on some clothes, though horror shook all my limbs; I issued from my apartment.

            \cite[236]{brontec}
        \end{quote}
        As a result of this shift from the passive to the active tense Brontë’s tone becomes warmer
        and more inviting. Taking this into consideration, an intriguing observation can be made of
        Brontë’s text: when the writing is unified and direct, the tone is warmer; when the writing
        is fragmented and passive, the tone is colder. This seems to suggest that
        unification—cooperation between many elements—is better than having many discontinuous
        parts. This, essentially, returns to Brontë’s message of the importance of unification and
        compromise: unity is superior—the only option, even—when compared to discord.

        Considering the previous two points, what is arguably the most contradictory
        characterization in the conclusion becomes easier to understand. Through the course of her
        novel, Brontë characterizes Jane as a woman who cannot bear being controlled, who "[cares]
        for [herself] [;] [t]he more solitary, the more friendless, the more unsustained [she is],
        the more [she respects herself]" \cite[366]{brontec}.   However, in the concluding chapter, Jane
        seems quite the opposite, speaking of marriage and belonging:
        \begin{quote}
            I know what it is to live entirely for and with what I love best on earth. I hold myself
            supremely blest—blest beyond what language can express; because I am my husband’s life
            as fully as he is mine. No woman was ever nearer to her mate than I am: ever more
            absolutely bone of his bone and flesh of his flesh.

            \cite[523]{brontec}
        \end{quote}
        While there is, at first glance, a contrast between Brontë’s previous characterization of
        Jane and her current characterization, closer inspection yields a different conclusion;
        Jane’s current characterization seems to be more an evolution of her previous personality,
        instead of a contrast. Regarding this, it is important to reiterate that Jane does not want
        to be controlled; while a relationship in which both sides are of equal power is acceptable,
        one where "[she] must be [another’s]" \cite[364]{brontec}—where her will is superseded—is not
        acceptable to her. Noting this, Brontë’s characterization of Jane in the final chapter
        begins to seem less contradictory; Jane is married, yes, but she and her husband—Mr.
        Rochester—have a reciprocal relationship, where "[she] is [her] husband’s life as fully as
        he is [hers]" \cite[523]{brontec}. Essentially, Jane’s relationship with Mr. Rochester is one where
        they are unified, which further reinforces Brontë’s point: harsh, hardline separation is
        neither productive, nor preferable in comparison to compromise and unity.

        Ultimately, however, everything must come to a close: loose ends must be tied together,
        ideas solidified and unified, a cohesive argument presented; and that is precisely what
        Brontë does in the conclusion of Jane Eyre. Jane, throughout the novel, is straining against
        limitations imposed on her by herself, by others, and by Victorian society; she tries to
        "run" from limitations by isolating herself, but Brontë, in her conclusion, recognizes the
        fact that regardless of how much an independent woman wants to shed the shackles of her own
        self-doubt, other’s judgment, and society’s expectations, a balance must be struck in order
        to continue as an active participant in the surrounding world.
    \end{flushleft}
\end{document}