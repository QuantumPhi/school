%!TEX program = xelatex

\documentclass[a4paper,12pt]{article}

\usepackage{polyglossia}
\setdefaultlanguage{english}

\usepackage[backend=biber,style=mla,noremoteinfo=false]{biblatex}
\usepackage[autostyle]{csquotes}
\usepackage{fontspec}
\usepackage{hyperref}
\usepackage{setspace}
\usepackage{titlesec}

\renewcommand{\thesection}{\Roman{section}}

\newcommand{\HRule}{\rule{\linewidth}{0.5mm}}
\newcommand{\sectionbreak}{\clearpage}

\doublespacing

\addbibresource{./Single_Party_States.bib}

\begin{document}
    \begin{titlepage}
    \vspace*{\fill}
    \begin{center}
        \HRule \\[0.4cm]
        { \large \bfseries Analyze the methods used by Joseph Stalin to obtain power and to what extent did Stalin follow the aims of his declared ideology? \\[0.4cm] }
        \HRule \\[1.5cm]

        Tarik Onalan
        \\[0.4cm]
        SCHOOL
        \\[0.4cm]
        19 June 2015
        \\[0.4cm]
        Word Count:
    \end{center}
    \vspace*{\fill}
\end{titlepage}
    
    %--------------------------------------------------------------------------------------------------%

    The early $20$th century was a period of upheaval in Russia: the first World War had just ended,
    leaving in its aftermath millions of dead Russians and the Russian national pride scarred in
    defeat; the second industrial revolution left behind a large industrial capacity, but the
    devastating defeat of Russia during World War I had left Russia's manufacturing sector in a
    shambles \cite[39]{danielsr}; the people of Russia, most of whom were peasantry, were expressing
    their disdain of the Romanov dynasty, the family that was at the head of Russia's Tsarist
    government. The peasant class was angry at the government's ignorance and abuse of the
    peasantry. The state practiced progressive terrorization and enserfment of the peasantry,
    ignoring their conditions, with shortages rampant and growing unrest. Vladimir Lenin, the leader
    of the Bolshevik Party, spoke of creating a new socialist state and removing the incompetent
    government: a call for revolution. The October Revolution in November of 1917---also known as
    the Bolshevik Revolution---was the culmination of the frustrations of the people of Russia, and
    thrust Stalin into the political scene as Lenin's protégé. Stalin used his influence as Lenin's
    direct subordinate to slowly build his network of political allies, all the while weeding out
    his enemies, until he was able to consolidate political power. After gaining power, Stalin only
    followed his declared ideology ---Marxism-Leninism-Leninism---to a moderate extent: while he did
    nationalize industry and implement some level of redistribution, he never created a socialist
    state, instead heading a centralized government.

    %--------------------------------------------------------------------------------------------------%

    \section{Origins}

        The origins of Stalin's single-party state stem to the period two decades before the
        Bolshevik revolution. While there is no single universally accepted cause of the Bolshevik
        Revolution, it is generally agreed upon that the Bolshevik Revolution was a product of
        previous trends; it was not a spontaneous event \cite[331]{danielsr}. The rapid
        industrialization of Russia's economy through the late $19$th and into the early $20$th
        century showcased the corresponding stagnancy of the Russian political system. Mistreatment
        of the growing permanent working class---composed primarily of extorted, abused serfs---
        incited more and more strikes and mutinies, paralyzing the government \cite[32]{pitirims}.
        The Russian Empire was composed of institutions that were increasingly obsolete in the new
        century, and was ill-suited to deal with a changing political and economic climate.
        \cite[18]{pitirims}.

        Lenin led the Russian Social-Democratic Labor Party, a Marxist group that was among the many
        entities disillusioned with the current government. The creation of the Bolshevik Party came
        around 1903, when Lenin's party fractured into the Bolsheviks---those who wanted revolution
        immediately---and the Mensheviks---those who wanted to wait longer before the revolution
        \cite[37]{pitirims}. Stalin, attracted by Lenin's call for revolution and vision of a
        Marxist state, joined the Bolsheviks \cite[54]{servicer}. Stalin became known for his crude,
        simple and yet pragmatic approach to politics, recognizing, like Lenin, the importance of
        propaganda and organization, and though he and Lenin did not agree on everything, Lenin
        promoted the still obscure Stalin in the Bolshevik Party, taking him under his patronage and
        promoting his career \cite[77,124]{servicer}.

        After world war I, the resentment harbored by the people began to boil over. Bloody Sunday,
        the humiliation of the Russo-Japanese war, and now, the destruction of World War I all
        contributed to a common feeling that the Romanov government was inadequate
        \cite[18,26,32]{pitirims}. The government's credibility was further compromised with the
        Rasputin scandal, disintegrating the government from the inside, causing the State Duma---
        the legislative assembly of the Russian Empire---to create a provisional government
        controlled by the Bolsheviks' rivals, the Mensheviks \cite[44]{pitirims}. The State Duma was
        already viewed as a travesty of an institution by the people; the provisional government it
        created was not viewed any better, and was seen as an incompetent institution
        (\cites[338]{danielsr}[32]{kuromiyah}). Lenin played upon the mood of the masses to further
        antagonize the people, not afraid of inciting violence in the revolution
        \cite[335]{danielsr}.

        The October Revolution took place in early November. By the end of the revolution, the
        Preparliament and Constituent assemblies were dismantled, and Lenin's Bolshevik party was in
        complete power \cite[46-47]{basilj}. More importantly, however, Stalin was appointed the
        People's Commissar, publicly creating direct association between himself and Lenin, opening
        for Stalin the pathway to consolidating influence and power in the new government
        \cite[124]{servicer}.

    %--------------------------------------------------------------------------------------------------%

    \section{Establishment}

        What Stalin was able to do with his newfound political influence as the People's Commissar
        ---and, more importantly, Lenin's direct subordinate---was build up a following in the
        Bolshevik Party and project policies in the interests of wide circles of the party
        \cite[3]{rigbyt}. Stalin was skilled with political tactics, manipulation, and bargaining,
        which would frequently allow him to get others to do what he wanted them to do
        (\cites[3]{rigbyt}[4]{carre}); Stalin's ``human touch'' in negotiations misled others into
        trusting him, after which he would use them for his own gain \cite[718]{kuromiyah}. As
        Stalin consolidated more political power, he was able to be more vocal about his ideologies;
        he was most serious about his support of Marxism-Leninism \cite[27]{reee}.

        This does not mean he was incapable of change, however. Stalin constantly adjusted to his
        surroundings with regards to his ideology, making sure that while he was maintaining a
        Marxist ideology, he was also remaining flexible in changing times \cite[720]{kuromiyah}.
        Stalin was, for example, an orthodox Marxist with regards to his belief that nothing good
        could come from a bourgeois state \cite[29]{reee}. However, he did inject ``Russian
        tradition'' into his version of Marxism-Leninism, centralizing it slightly to concur with
        the Russian ``cult of personality'' (\cites[23]{reee}[32]{hingleyr}). Even before he was the
        leader of his single-party state, he was building his ``cult of personality'' so as to gain
        influence over the public, as well as the government.

        This maneuvering was for a metaphorical ``charge'' to power. When Lenin suffered a stroke in
        1922, his position of leadership in the Bolshevik Party was effectively terminated. At this
        point, Stalin was a member of the Politburo---a group of six people that carried out many of
        the executive decisions in the government---and, more importantly, in this case, he was the
        General Secretary of the government; with these credentials Stalin now had---for the most
        part---the ability to influence without the support of Lenin, which Lenin recognized as the
        source of Stalin's growing power \cite[160,167]{kortm}. Lenin had originally tapped Trotsky
        as his preferred successor, but Trotsky did not place his supporters in the party
        \cite[167]{kortm}. Stalin used this opportunity to fill the party with his patrons;
        patronage was, after all, common in the Soviet government \cite[4-5]{rigbyt}.

        After filling the Bolshevik Party with his supporters, all that remained for Stalin to do to
        consolidate power was to remove political enemies. Stalin's greatest political enemy at this
        time was---arguably---Leon Trotsky \cite[167]{kortm}. In 1927, Stalin used his supporters in
        the congress to remove Trotsky from the Bolshevik Party and exile him from the country,
        removing his greatest political enemy, and leaving him on the top of the Soviet Union

    %--------------------------------------------------------------------------------------------------%

    \section{Rule}

        Stalin, after gaining power, followed only parts of his ideology. One of the first things
        Stalin on gaining power the Five Year Plan, a plan that would end private ownership of all
        land \cite[51]{basilj}. Having the government take control of all private land is,
        essentially, a Marxist economic decision. Stalin was also actively nationalizing industries,
        again a Marxist economic decision \cite[44]{remingtont}. Stalin's decisions while in power,
        at face value, are concurrent with his Marxist ideology. However, Stalin's decisions were
        not completely ``Marxist'', nor was he setting up a pure Marxist government; on the
        contrary, Stalin's centralized rule was against the basic tenets of Marxism-Leninism
        \cite[49]{remingtont}.

        Events like the Great Terror highlight how Stalin ruled; in response to fear of opposition
        within his party, Stalin launched the Great Terror: a campaign to forcibly remove political
        dissidents by exiling them or killing them \cite[716]{kuromiyah}. What Stalin accomplished
        with the Great Terror was that he increased his political power by removing political
        opponents, and he also increased his public influence. Stalin heavily used terror tactics to
        keep the public---or, more specifically, dissidents---in check; the amount of terror in
        society was an accurate measure of the political and social malleability of the population
        \cite[386]{conquestr}. The reason why Stalin used terror tactics so heavily can be traced
        back to the tsars: most had an explosive temper and doled out harsh punishments to deter
        further deviance \cite[64]{hingleyr}. Stalin admired the heavy handed policies of the tsars,
        and those policies, evidently, made their way into his rule \cite[26]{remingtont}.

        Stalin's relationship with the economy was slightly more standard with respect to his
        Marxist ideology. Stalin's economic policies were essentially centred around the belief that
        state direction of the economy was a necessary social development \cite[50]{remingtont}.
        Stalin nationalized land, nationalized industry, and collectivized agriculture
        \cite[50]{remingtont}. His belief that the economy should eventually be run by the armed
        proletariat, though not fully realized, was a basic statement of Marxist economic policy
        \cite[29]{reee}. Stalin, publicly, at least, focused on the ``distribution'' of the economy
        ---as in, distribution of control---heavily while in power. In the background, however, he
        sometimes made economic decisions that were essentially for progress at all costs, be it
        human or otherwise. Stalin's labor conscription in the name of industrializing the state was
        not an acceptable Marxist policy, yet he implemented it anyway \cite[49]{remingtont}.
        Ultimately, Stalin's economic policy can be described as a mixture of both his ideology and
        his desire for economic progress.

        With regards to the public---or, more accurately, the public's social space--- Stalin's
        actions were not entirely concurrent with his Marxist ideology. Stalin implemented heavy
        censorship laws as single-party leader; all newspapers were taken under control of the
        Bolshevik government, even if they were not opposing the government in any way
        \cite[47-49]{basilj}. Later, even newspapers that were supportive of the government were
        taken under the control of the state, censored or propagandized \cite[49]{basilj}. Through
        censorship and propaganda, Stalin was able to hone his ``cult of personality'', elevating
        his image in the eyes of the public \cite[718]{kuromiyah}. Stalin used his ``cult of
        personality'' to control the public, even though elevating the image of an individual is not
        permissible in Marxism-Leninism-Leninism \cite[718]{kuromiyah}. Essentially, with regards to
        the public, Stalin's rule meant that information was changed or lost for the purpose of
        further centralizing Stalin's state.
        
    %--------------------------------------------------------------------------------------------------%

    \newpage

    After gaining power, Stalin only partially followed the aims of his declared ideology. While he
    did carry out many of the nationalization or collectivization tasks that were expected in his
    ideology, his usage of terror to control the public as well as the extensive promotion of his
    public image go against his declared ideology. This is partially due to the fact that Stalin's
    ideology---and, through that, rule---was very much influenced by the example set by the tsars
    and traditional Russian culture. Concepts like the ``cult of personality'' were traits seen even
    in the first tsars. Policies like the Great Terror were also reminiscent of the punishments
    given out to dissidents by the tsars. As such, the best way to describe Stalin's actual ideology
    would be a kind of ``Russianized'' Marxism-Leninism; it takes the economic policies of Marxism-
    Leninism and melds it with the political and social policies of the tsars. One thing to note,
    however, is that Stalin did not always follow his ideology with respect to the economy, either.
    Conscripted labor is not a desired economic policy in Marxism-Leninism; it is what Stalin
    believed he had to do to industrialize the Soviet Union. Ultimately, Stalin is closest to his
    declared ideology in economic policy; all in all, however, Stalin only followed his declared
    political ideology---Marxism-Leninism---to a moderate extent.

    %--------------------------------------------------------------------------------------------------%

    \printbibliography
\end{document}