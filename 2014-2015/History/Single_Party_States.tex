%!TEX program = xelatex

\documentclass[a4paper,12pt]{article}

\usepackage{polyglossia}
\setdefaultlanguage{english}

\usepackage[backend=biber,style=mla,noremoteinfo=false]{biblatex}
\usepackage[autostyle]{csquotes}
\usepackage{fontspec}
\usepackage{hyperref}
\usepackage{setspace}
\usepackage{titlesec}

\renewcommand{\thesection}{\Roman{section}}

\newcommand{\HRule}{\rule{\linewidth}{0.5mm}}
\newcommand{\sectionbreak}{\clearpage}

\doublespacing

\addbibresource{./Single_Party_States.bib}

\begin{document}
    \begin{titlepage}
    \vspace*{\fill}
    \begin{center}
        \HRule \\[0.4cm]
        { \large \bfseries Analyze the methods used by Joseph Stalin to obtain power and to what extent did Stalin follow the aims of his declared ideology? \\[0.4cm] }
        \HRule \\[1.5cm]

        Tarik Onalan
        \\[0.4cm]
        SCHOOL
        \\[0.4cm]
        19 June 2015
        \\[0.4cm]
        Word Count:
    \end{center}
    \vspace*{\fill}
\end{titlepage}
    
    %--------------------------------------------------------------------------------------------------%

    The early $20$th century was a period of upheaval in Russia: the first World War had just ended,
    leaving in its aftermath millions of dead Russians and the Russian national pride scarred in
    defeat; the second industrial revolution left behind a large industrial capacity, but the
    devastating defeat of Russia during world war 1 had left Russia's manufacturing sector in a
    shambles \cite[39]{danielsr}; the people of Russia, most of whom were peasantry, were expressing
    their disdain of the Romanov dynasty, the family that was at the head of Russia's Tsarist
    government. The peasant class was angry at the government's ignorance and abuse of the
    peasantry. The state practiced progressive terrorization and enserfment of the peasantry,
    ignoring their conditions, with shortages rampant and growing unrest. Vladimir Lenin, the leader
    of the Bolshevik Party, spoke of creating a new socialist state and removing the incompetent
    government: a call for revolution. The October Revolution in November of 1917---also known as
    the Bolshevik Revolution---was the culmination of the frustrations of the people of Russia, and
    thrust Stalin into the political scene as Lenin's protégé. Stalin used his influence as Lenin's
    direct subordinate to slowly build his network of political allies, all the while weeding out
    his enemies, until he was able to consolidate political power. After gaining power, Stalin only
    followed his declared ideology ---Marxism-Leninism---to a moderate extent: while he did
    nationalize industry and implement some level of redistribution, he never created a socialist
    state, instead heading a centralized government.

    %--------------------------------------------------------------------------------------------------%

    \section{Origins}

        The origins of Stalin's single-party state stem to the period two decades before the
        Bolshevik revolution. While there is no single universally accepted cause of the Bolshevik
        Revolution, it is generally agreed upon that the Bolshevik Revolution was a product of
        previous trends; it was not a spontaneous event \cite[331]{danielsr}. The rapid
        industrialization of Russia's economy through the late $19$th and into the early $20$th
        century showcased the corresponding stagnancy of the Russian political system. Mistreatment
        of the growing permanent working class---composed primarily of extorted, abused serfs---
        incited more and more strikes and mutinies, paralyzing the government \cite[32]{``The
        Bolshevik Revolution''}. The Russian Empire was composed of institutions that were
        increasingly obsolete in the new century, and was ill-suited to deal with a changing
        political and economic climate. \cite[18]{pitirims}.

        Lenin led the Russian Social-Democratic Labor Party, a Marxist group that was among the many
        entities disillusioned with the current government. The creation of the Bolshevik Party came
        around 1903, when Lenin's party fractured into the Bolsheviks---those who wanted revolution
        immediately---and the Mensheviks---those who wanted to wait longer before the revolution
        \cite[37]{pitirims}. Stalin, attracted by Lenin's call for revolution and vision of a
        Marxist state, joined the Bolsheviks \cite[54]{servicer}. Stalin became known for his crude,
        simple and yet pragmatic approach to politics, recognizing, like Lenin, the importance of
        propaganda and organization, and though he and Lenin did not agree on everything, Lenin
        promoted the still obscure Stalin in the Bolshevik Party, taking him under his patronage and
        promoting his career \cite[77,124]{servicer}.

        After world war I, the resentment harbored by the people began to boil over. Bloody Sunday,
        the humiliation of the Russo-Japanese war, and now, the destruction of World War I all
        contributed to a common feeling that the Romanov government was inadequate
        \cite[18,26,32]{pitirims}. The government's credibility was further compromised with the
        Rasputin scandal, disintegrating the government from the inside, causing the State Duma---
        the legislative assembly of the Russian Empire---to create a provisional government
        controlled by the Bolsheviks' rivals, the Mensheviks \cite[44]{pitirims}. The State Duma was
        already viewed as a travesty of an institution by the people; the provisional government it
        created was not viewed any better, and was seen as an incompetent institution
        (\cites[338]{danielsr}[32]{kuromiyah}). Lenin played upon the mood of the masses to further
        antagonize the people, not afraid of inciting violence in the revolution
        \cite[335]{danielsr}.

        The October Revolution took place in early November. By the end of the revolution, the
        Preparliament and Constituent assemblies were dismantled, and Lenin's Bolshevik party was in
        complete power \cite[46-47]{basilj}. More importantly, however, Stalin was appointed the
        People's Commissar, publicly creating direct association between himself and Lenin, opening
        for Stalin the pathway to consolidating influence and power in the new government
        \cite[124]{servicer}.

    %--------------------------------------------------------------------------------------------------%

    \section{Establishment}

        Stalin's position as the People's Commissar 

    %--------------------------------------------------------------------------------------------------%

    \section{Rule}

        Hello, world!
        
    %--------------------------------------------------------------------------------------------------%

    \printbibliography
\end{document}