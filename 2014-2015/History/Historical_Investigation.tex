\documentclass[a4paper]{article}

\newcommand{\HRule}{\rule{\linewidth}{0.5mm}}

\begin{document}
    \begin{titlepage}
    \begin{center}
        \HRule \\[0.4cm]
        { \large \bfseries To what extent was the creation of the Manhattan Project a product of Germany's desire to expand between 1932 and 1942? \\[0.4cm] }
        \HRule \\[1.5cm]

        Tarik Onalan
        \\[0.4cm]
        \#\#\#\#\#\#-\#\#\#\#
        \\[0.4cm]
        SCHOOL
        \\[0.4cm]
        Word Count: 1706
        \vfill

        {\large June 2015}
    \end{center}
\end{titlepage}
    
    \section{Plan of Investigation}
        The task of this investigation is to analyze to what extent the creation of the Manhattan project
        was a product of Germany's desire to expand between 1932 and 1942.

        The primary method of this investigation is to analyze accounts of the objectives of the Manhattan
        Project in the years before its inception to understand possible reasons behind its founding. This
        investigation will also analyze accounts of the interaction of the German and American propaganda
        machines to understand the influence Germany had on the global populace. Additionally, accounts of
        German expansion by Americans will be analyzed to understand to what extent the German expansion
        was viewed as a direct threat to America.

        This investigation will focus on two organizations: the Nazis of Germany, and the office of Franklin
        Delano Roosevelt. The scope of this investigation will be limited to sources pertaining to these
        organizations.

    % Conversely, although Roosevelt
    % did respond to Szilard's 1939 petition by appointing a commission
    % to explore the exigencies of building an atomic weapon, U.S.
    % policymakers did not take energetic action to conceptualize, fund,
    % organize, and administer the enterprise until 1940-41, when attention
    % was much captivated by the war in the Pacific.
    %
    % Norris 12

    \section{Summary of Evidence}
        \begin{itemize}
            \item Fear of Germans and the Manhattan Project:
            \begin{itemize}
                \item American leadership perceived Germany as the only enemy with the capacity to take advantage of information taken from the Americans (Groves 141)
                \item Security from Germany was one of the founding objectives of the Manhattan Project (Groves 140)
                \item Leo Szilard---a physicist and refugee---brings information about German atomic weapons development, which led President Roosevelt to begin  atomic weapons research (Norris 12)
                \item The US began atomic weapons development out of fear that the Germans would produce a similar weapon (Steinberg 88)
            \end{itemize}
            \item Competition to produce weapons:
            \begin{itemize}
                \item German scientists went to Germany to contribute to atomic development against America (Powers 6)
                \item America feared that Germany would produce an atomic weapon (Stoff 136)
                \item Rockets could replace and outclass long-range guns, and constituted a loophole in the Versailles treaty (Neufeld 2)
                \item Rockets could be used to attack America, thus making them appealing to Germans and fear-inducing to Americans (Neufeld 157)
            \end{itemize}
            \item Propaganda:
            \begin{itemize}
                \item American:
                \begin{itemize}
                    \item The American government controlled media outlets to spread propaganda against Nazi Germany (Laurie 2)
                \end{itemize}
                \item Nazi:
                \begin{itemize}
                    \item Nazi Germany had ``a distorted vision of national grandeur'' (Kallis 37)
                    \item Expansion into Europe made Americans fearful because of the image portrayed by the ``distorted vision'' (Kallis 37)
                    \item Increasing German global influence (e.g. Argentina, Austria) (Pyenson 2)
                    \item Increasing German focus on cultural and territorial spread pre-WWII (Pyenson 17)
                    \item The Germans were looking to create a Nordic Europe (Guettel 192)
                    \item The Germans, among other regions, had extensive propaganda distribution in Latin America, which it planned to use as leverage against the United States (Kris 59)
                    \item The German government employed self-victimization to get popular support for expansion into areas populated by Germans (Bergen 1)
                \end{itemize}
            \end{itemize}
        \end{itemize}
    \section{Evaluation of Sources} % O + P + 2V + 2L
        The author of the article \textit{Dividing the Indivisible: The Fissured Story of the Manhattan Project},
        Margot Norris, is a professor of English and Comparative Literature at the University of California, Irvine.
        Norris' purpose in writing this article is to separate the facts of the Manhattan Project from the myths,
        her thesis being that the actual ambitions of the Manhattan Project were hidden from the public as political
        policy, with stated ambitions feeding misinformation to the public as part of propaganda. 
\end{document}