%!TEX program = xelatex

\documentclass[a4paper,12pt]{article}

\usepackage{polyglossia}
\setdefaultlanguage{english}

\usepackage[backend=biber,style=mla,noremoteinfo=false]{biblatex}
\usepackage[autostyle]{csquotes}
\usepackage{hyperref}
\usepackage{indentfirst}
\usepackage{setspace}

\renewcommand{\thesection}{\Roman{section}}

\newcommand{\HRule}{\rule{\linewidth}{0.5mm}}

\doublespacing

\addbibresource{./Historical_Investigation.bib}

\begin{document}
    \begin{titlepage}
    \begin{center}
        \HRule \\[0.4cm]
        { \large \bfseries To what extent was the creation of the Manhattan Project a product of Germany's desire to expand between 1932 and 1942? \\[0.4cm] }
        \HRule \\[1.5cm]

        Tarik Onalan
        \\[0.4cm]
        \#\#\#\#\#\#-\#\#\#\#
        \\[0.4cm]
        SCHOOL
        \\[0.4cm]
        Word Count: 1706
        \vfill

        {\large June 2015}
    \end{center}
\end{titlepage}
    
    %--------------------------------------------------------------------------------------------------%

    \section{Plan of Investigation}

        The task of this investigation is to analyze to what extent the creation of the Manhattan
        project was a product of Germany's desire to expand between 1932 and 1942.

        The primary method of this investigation is to analyze accounts of the objectives of the
        Manhattan Project in the years before its inception to understand possible reasons behind
        its founding. This investigation will also analyze accounts of the interaction of the German
        and American propaganda machines to understand the influence Germany had on the global
        populace. Additionally, accounts of German expansion by Americans will be analyzed to
        understand to what extent the German expansion was viewed as a direct threat to America.

        This investigation will focus on three points: fear of Germans by the Americans, competition to
        produce atomic weapons, and propaganda campaigns. The scope of this investigation will be
        limited to sources pertaining to these points.

    %--------------------------------------------------------------------------------------------------%

    % Conversely, although Roosevelt
    % did respond to Szilard's 1939 petition by appointing a commission
    % to explore the exigencies of building an atomic weapon, U.S.
    % policymakers did not take energetic action to conceptualize, fund,
    % organize, and administer the enterprise until 1940-41, when attention
    % was much captivated by the war in the Pacific.
    %
    % Norris 12

    \section{Summary of Evidence}
        \begin{itemize}
            \item Fear of Germans and the Manhattan Project:
            \begin{itemize}
                \item American leadership perceived Germany as the only enemy with the capacity to take advantage of information taken from the Americans \cite[141]{grovesl}.
                \item Security from Germany was one of the founding objectives of the Manhattan Project \cite[140]{grovesl}.
                \item Leo Szilard---a physicist and refugee---brings information about German atomic weapons development and asks Roosevelt to pursue development, which led President Roosevelt to begin  atomic weapons research \cite[12]{norrism}.
                \item America feared that Germany would produce an atomic weapon \cite[136]{stoffm}.
            \end{itemize}
            \item Competition to produce weapons:
            \begin{itemize}
                \item German scientists went to Germany to contribute to atomic development against America \cite[6]{powerst}.
                \item The US began atomic weapons development out of fear that the Germans would produce a similar weapon \cite[88]{steinbergg}.
                \item The US began atomic weapons development to deter Germany's nuclear first-strike capability \cite[89]{steinbergg}.
                \item Rockets could replace and outclass long-range guns, and constituted a loophole in the Versailles treaty \cite[2]{neufeldm}.
                \item Rockets could be used to attack America with various payloads, thus making them appealing to Germans and fear-inducing to Americans \cite[157]{neufeldm}.
            \end{itemize}
            \item Propaganda:
            \begin{itemize}
                \item American:
                \begin{itemize}
                    \item The American government controlled media outlets to spread propaganda against Nazi Germany \cite[2]{lauriec}.
                    \item The Manhattan Project was indirectly used as deterrent against German expansion \cite[140]{grovesl}.
                \end{itemize}
                \item Nazi:
                \begin{itemize}
                    \item Nazi Germany had ``a distorted vision of national grandeur'' \cite[37]{kallisa}.
                    \item Expansion into Europe made Americans fearful because of the image portrayed by the ``distorted vision'' \cite[37]{kallisa}.
                    \item Increasing German global influence (e.g. Argentina, Austria) \cite[2]{pyensonl}.
                    \item Increasing German focus on cultural and territorial spread pre-WWII \cite[17]{pyensonl}.
                    \item The Germans were looking to create a Nordic Europe \cite[192]{guettelj}.
                    \item The Germans, among other regions, had extensive propaganda distribution in Latin America, which it planned to use as leverage against the United States \cite[59]{krise}.
                    \item The German government employed self-victimization to get popular support for expansion into areas populated by Germans \cite[1]{bergend}.
                \end{itemize}
            \end{itemize}
        \end{itemize}

    %--------------------------------------------------------------------------------------------------%

    \section{Evaluation of Sources} % O + P + 2V + 2L
    
        The author of the article \textit{Dividing the Indivisible: The Fissured Story of the
        Manhattan Project}, Margot Norris, is a professor of English and Comparative Literature at
        the University of California, Irvine. The purpose of this article is to separate the facts
        of the Manhattan Project from the myths, with Norris' thesis being that the actual ambitions
        of the Manhattan Project were hidden from the public as political policy, with stated
        ambitions feeding misinformation to the public as part of propaganda. A value of the origin
        is that as a professor of English, the author brings a new perspective to the topic that is
        based on wide access to literary works instead of political documents. A limitation of the
        origin is that the author's field is not focused on history, causing the author to have
        limited resources for research, which may not be comprehensive. The value of the purpose is
        that it incorporates the political perception of the Manhattan Project, which is highly
        relevant in understanding the role of fear of German expansion. The limitation of the
        purpose is that it is more focused on the role of the US in the Manhattan Project, with a
        lack of sources from foreign countries, and thus a lack of a foreign perspective.

        The author of the book \textit{Cultural imperialism and exact sciences: German expansion
        overseas, 1900-1930}, Lewis Pyenson, is a professor of the History of Science with a Ph.D.
        from Johns Hopkins University. He is the author of many books about mathematics and physics
        in Germany. The purpose of this book is to evaluate the increasing influence of Germany
        towards the beginning of World War 2, with the thesis being that Germany's scientific
        prowess led to its increasing influence around the world. The value of the origin is that as
        a professor of history, Pyenson has access to large amounts of resources concerning the
        topics he has written about in his book. The limitation of the origin is that the author may
        not consider political and social issues in his analysis, as Pyenson is a professor of the
        history of science, meaning that the politics behind an event are not as central to his
        research as the science behind an event. The value of the purpose is that it provides a
        direct insight into the relation between German expansion and scientific development. The
        limitation of the purpose is that it does not focus on the involvement of politics to the
        extent that it does the sciences.

    \section{Analysis}

        German expansion's correlation on the Manhattan Project can be described through three
        different categories: fear of Germans by the Americans, competition to produce weapons, and
        propaganda. Fear of Germans represents the American perception of the Germans, weapons
        development competition represents the American reaction to German expansion, and propaganda
        on the part of the Germans is direct evidence of their desire to expand; American counter-
        propaganda, on the other hand, falls between the categories of perception and reaction. The
        creation Manhattan Project is analogous to the sum of these individual parts: perception
        caused a reaction, which, in turn, caused the creation of the Manhattan Project.

        The Manhattan Project was founded in some part out of fear of the Germans. The American
        government feared the production capabilities of Germany, perceiving them as the only enemy
        with the capacity to take advantage of information taken from the Americans (Groves 6). This
        belief alone provides insight on the perspective Americans had of Germans: the German state
        as a technologically advanced country capable of scientific and industrial development on a
        scale comparable to the United States. The big fear was that this development capacity would
        go towards producing an atomic weapon that could be used against the United States
        \cite[136]{stoffm}. Leo Szilard---a physicist and refugee from Germany---brought information
        on German atomic weapons development to Roosevelt, asking him to consider development of
        atomic weapons technologies \cite[12]{norrism}. Roosevelt, soon after that, created the
        Manhattan Project \cite[12]{norrism}. The original objective of the Manhattan Project was,
        in fact, to have security from Germany \cite[140]{grovesl}. With this information in mind,
        it becomes easier to see that fear of Germans was partially a motivating factor for founding
        the Manhattan Project.

        The competition between the Americans and the Germans to produce weapons was also a
        motivating factor for creating the Manhattan Project. When German scientists went to Germany
        to contribute to atomic weapons development, the United States felt pressured to produce a
        competing atomic weapons program to prevent Germany from acquiring an atomic weapon first
        (Powers 6; Steinberg 88). Given that one of the founding objectives of the Manhattan Project
        was security from the Germans, this fact becomes more pronounced, as the pressure to produce
        an atomic weapon is amplified both by the fact that the German atomic weapons program was
        rather advanced and the fact that Germany was prepared to use an atomic weapon when it had
        produced one. This pressure was increased after extensive German rocket development. Rockets
        constituted a loophole in the Versailles treaty, allowing Germany to potentially build up an
        arsenal of weaponry that could operate at ranges far enough to attack America (Neufeld 2;
        Neufeld 157). Rockets could be used to deliver nuclear warhead payloads, which made them
        particularly appealing to the Germans \cite[157]{neufeldm}. The United States, in trying to
        deter a nuclear first-strike by Germany, created the Manhattan Project as a competing
        nuclear weapons program to Germany's nuclear weapons program \cite[89]{steinbergg}. What the
        rocket technology accomplished was suddenly make America a reachable target with any kind of
        payload; America's response to this development was to make the Manhattan Project to outpace
        the German atomic weapons program so a German nuclear payload would never reach America.    

        The aggressiveness of the German propaganda campaign helps expose other motivations for
        creating the Manhattan Project. Propaganda campaigns on both sides, American and German,
        shed some light into both the fear and weapon-competition issues; the Germans were, with
        their expansion, looking to create a Nordic Europe, using their control of the press to gain
        popular support (\cites[142]{guettelj}[1]{bergend}). What made this desire to expand
        disturbing to the Americans was the perception that the Germans had a distorted national
        vision \cite[37]{kallisa}. The Germans already had influence--- through extensive propaganda
        ---in the Latin American region, planning to use their cultural and political influence in
        the region as leverage against the Americans \cite[2,17]{pyensonl}. The fact that Germany
        had a foothold so close to the American homeland brings the issue into the realm of fear;
        given that the Germans were so aggressive with their expansion, the fact that they were ``on
        the doorstep'', so to speak, of the Americans was disturbing, causing no end of counter-
        propaganda by the American government \cite[2]{lauriec}. Ultimately, the propaganda reveals
        the simple fact that a good portion of the Manhattan Project was simply for deterrent
        purposes against Germany, allowing for increased safety from German aggression
        \cite[140]{grovesl}.

    %--------------------------------------------------------------------------------------------------%

    \section{Conclusion}

        Fear of Germans, competition to produce weapons, and propaganda against Germans all tie back
        to the fact that the Germans were more aggressive in their expansion in the decade before
        the creation of the Manhattan Project. While the it is believed that the Pacific War was a
        primary cause for the creation of the Manhattan Project, it is clear that Germany's desire
        to expand also contributed greatly to the founding of the Manhattan Project. Ultimately, the
        Manhattan Project's creation was influenced by Germany's desire to expand by a great extent
        between the years of 1932 and 1942.

    %--------------------------------------------------------------------------------------------------%

    \printbibliography
\end{document} 