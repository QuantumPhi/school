\documentclass[12pt]{article}

\usepackage[english]{babel}
\usepackage[autostyle]{csquotes}
\usepackage{ifpdf}
\usepackage{mla}
\usepackage{setspace}

\doublespacing
\MakeOuterQuote{"}

\title{Deterioration, Dissonance, and Disillusionment}
\date{12 December 2014}
\author{Tarik Onalan}

\begin{document}
    \maketitle
    \centerline{Word Count: 1399}
    Through Meursault's social alienation and the dissonance between his
    physical self and his emotions, Camus illustrates how in order to have a true
    understanding of life and cope with life's irrationality, one must separate
    themselves from society. When Meursault is thrown
    into prison, Camus uses parallelism and a detached tone to demonstrate how isolation
    affects Meursault's mental state. Through the motif of heat, Camus symbolizes
    the dissonance between Meursault's physical and emotional selves, thus illustrating
    the effects of isolating and suppressing one's emotions. Meursault's emotional
    and social isolation provides him with an objective perspective of reality, which
    strongly contrasts to the attempts of minor characters' at creating meaning.
    In conclusion, Camus portrays how in a world of irrationality, the only
    way to cope and understand life is to separate yourself, either
    emotionally or physically, from society in order to have a clear and
    unmanipulated perception of reality. Thus, separation provides both a true understanding
    of life, and freedom to live one's life as they choose.

    \newpage

    In The Stranger, written by Albert Camus and
    translated by Matthew Ward, Camus explores the motifs of social and physical
    alienation, and through that, explores how isolation impacts one's perspective
    of life. Meursault's experience in the prison towards the end of the novel is
    one demonstration of Camus' portrayal of alienation in the novel. Interestingly,
    due to his alienation, Meursault more easily copes with the pointlessness and
    the randomness of the world around him, unlike the other minor characters, such
    as the magistrate, who try to perceive a false reality. Thus, through
    Meursault's social alienation and the dissonance between his physical self and
    his emotions, Camus illustrates how in order to have a true understanding of
    life and cope with life's irrationality, one must separate themselves from
    society. \\

    When Meursault is thrown into prison,
    Camus uses parallelism and a detached tone to demonstrate how isolation
    affects Meursault's mental state. At first, "[Meursault] sort of [waits]
    for something to happen" (Camus 72). After killing the Arab, he receives
    little punishment, and almost gets treated kindly, as Meursault gets the
    impression of being "one of the family" (Camus 71). Meursault is waiting
    for a physical acknowledgement of the consequences of his actions.
    Meursault's need for physical acknowledgement is
    demonstrated when he goes on a liaison with Marie, and experiences
    physical verification from "Marie's heart beating softly" under his
    head (Camus 20). The lack of physical verification is emphasized by how
    Camus has Meursault paraphrase his own words instead of directly
    narrating what he said. Instead of "I killed an Arab," Meursault states
    "I said I'd killed an Arab", making it seem like he is
    separated from reality (Camus 72). This is demonstrated by Meursault's isolation
    from society; not only does Meursault not belong in any cultural groups,
    as "most of [the prisoners are] Arabs," he is physically isolated and
    "put in a cell by [himself]" (Camus 72). Meursault's isolation from any
    physical contact and disconnect from his own physicality draws him down
    into unconsciousness, as only after "many long" days in the prison does
    he become aware that "[the voice] that had been ringing in [his ears]"
    was his own (Camus 81). Thus, Camus illustrates how some level of human
    interaction is necessary for maintaining an attachment to reality, as
    demonstrated by Meursault's increasing detachment as he spends more time
    alone in the prison. Furthermore, Meursault begins to settle into the
    routine of the prison. Camus describes this when Meursault is going to
    see Marie, that Meursault "[goes] down a long corridor, down some stairs
    and, finally, another corridor" (Camus 73). Camus' use of
    parallelism--"down" and "corridor"--emphasizes the repetitiveness of
    prison. Meursault simply sees "the main problem [of prison as] killing
    time" , and that prison "[takes] away freedom" (Camus 78). When
    Meursault finds the story about the Czechoslovakian, he remarks that
    "[he] must have read [the] story a thousand times" (Camus 80). At this
    point, Camus is not emphasizing the fact that he read the story "a
    thousand times," but that Meursault seems to be reading out of routine:
    the story just becomes another way for him to pass the time. Through
    Meursault's attachment to routine, Camus emphasizes how in order to cope
    with the lack of emotional connections, Meursault relies on physical
    routine as a distraction. \\

    Through the motif of heat, Camus symbolizes the dissonance between
    Meursault's physical and emotional selves, thus illustrating the effects
    of isolating and suppressing one's emotions. In prison, Meursault notes
    that his summers blended together, "[a]nd [knows] that as soon as the
    weather [turns] hot that something new [is] in store for him" (Camus
    82). Heat is a pervasive symbol in the novel, and, in this instance, it
    functions as a symbol for Meursault's lack of emotional understanding.
    Camus also uses this symbol leading up to the scene when Meursault
    shoots the Arab and the "heat [is] pressing down on [him]" (Camus 57).
    Camus uses heat to symbolize the disconnect between Meursault's
    emotional and physical being. Meursault seems cold and objective even as
    "[he fires] four more times at the motionless body" of the Arab, which
    stands in stark contrast to the heat that he experiences (Camus 59).
    Thus, Camus illustrates the dissonance between Meursault's emotional and
    physical self-understanding. Although he is aware of himself physically,
    Meursault seems to be emotionally absent from the scene. For example,
    when describing the moment he shoots the Arab, Meursault states, "The
    trigger gave; I felt the smooth underside of the butt" (Camus 59).
    Again, Camus emphasizes Meursault's focus on his physical environment
    rather than his motives or emotions, as he strangely focuses on the
    "smooth underside" of the gun he's holding. This becomes especially
    apparent during the scene where Meursault shoots the Arab, as Meursault
    "[strains] every nerve in order to overcome the sun" (Camus 57). The
    heat from the sun represents how Meursault focuses on the physical
    being, rather than the emotional. With this, Camus foreshadows the
    emotional disconnect of Meursault's trial. Throughout the trial,
    Meursault notices "how hot…it [is]" in the courtroom, all while the
    prosecutor exclaims that Meursault "[does not] have a soul… nothing
    human" (Camus 101). Again, Meursault chooses to focus on the physical
    being instead of his emotions. Thus, Camus emphasizes that because
    Meursault does not want to face his emotions, he resorts to focusing on
    physical feelings. The sun and the heat that appear throughout the novel
    are symbolic of the emotion that Meursault is suppressing and hiding,
    even from himself. \\

    Meursault's emotional and social isolation provides him with an
    objective perspective of reality, which strongly contrasts to the
    attempts of minor characters' at creating meaning. Thus, Camus
    demonstrates how although Meursault's emotional and social isolation are
    a detriment to his mental state, they interestingly provide Meursault
    with a clearer understanding of the futility of creating meaning out of
    life. As Meursault nears his execution, he comes to the realization that
    "there [is] no way [to escape death]" (Camus 81). Camus contrasts
    Meursault's realization with the perspectives of minor characters to
    emphasize the clarity of Meursault's perspective of reality. For
    example, the chaplain states, "sometimes we think we're sure in when in
    fact we're not" (Camus 116). Camus illustrates Meursault's certainty
    through the irony of the chaplain's own convictions. The irony of the
    chaplain's statement stems from the fact that he relies on religion to
    create meaning in things he does not understand. The chaplain remarks
    that "[e]very man [he] has known in [Meursault's] position has turned to
    Him," which makes Meursault's rejection of religion confusing (Camus 116). Camus
    emphasizes Meursault's clarity of perspective through Meursault's
    statement: "what did… God or the lives people choose or the fate they
    think they elect matter… when we're all elected by the same fate?"
    (Camus 121). Separated from society, Meursault is able to recognize the
    illusion of choice of fate. Meursault understands that the only fate for
    humans is death, that "[he] had lived [his] life one way and [he] could
    just as well have lived it another… [but] nothing [matters]" (Camus
    121). Meursault's isolation from society is illustrated by the fact that
    "at one time, way back, [he] had searched for a face in [the walls]...
    [but] [n]ow it was all over" (Camus 119). Thus, Camus emphasizes the
    fact that while Meursault's isolation from society and objectiveness
    make him blind to his emotions, it gives him a clearer perspective. \\

    In conclusion, Camus portrays how in a world of irrationality, the only
    way to cope and understand life is to separate yourself, either
    emotionally or physically, from society in order to have a clear and
    unmanipulated perception of reality. In the last few pages of the novel,
    Meursault mentions his mother, and how she "played at beginning again"
    (Camus 122). Instead of fighting against her mortality and trying to
    make sense of life, Meursault's mother decides to live her life as she
    pleases, without trying to create a false reality for herself. In
    essence, she separates herself from society by refusing to live her last
    days on anyone else's terms. Thus, not only does Camus illustrate that
    alienation and separation gives people a clearer view of life and
    reality, he illustrates that it can also create happiness in one's life,
    as they realize that the only person who has any control over their own
    life is themselves. Thus, separation provides both a true understanding
    of life, and freedom to live one's life as they choose.

    \begin{workscited}
        \bibent
        Camus, Albert. \textit{The Stranger}. Trans. Matthew Ward. New York:
            Vintage International, 1989. Print.
    \end{workscited}
\end{document}
