\documentclass[12pt,a4paper]{article}

\usepackage[english]{babel}
\usepackage[autostyle]{csquotes}
\usepackage{ifpdf}
\usepackage{mla}
\usepackage{setspace}

\doublespacing
\MakeOuterQuote{"}

\begin{document}
    \begin{mla}{Tarik}{Onalan}{Calvert}{English}{20 November 2014}{Discussion Reflection}
        \centerline{\textit{The Stranger} by Albert Camus, translated by Matthew Ward}
        \noindent{}Word Count: 355 words \\
        Albert Camus, in \textit{The Stranger}, maintains a cultural vaguity that
        caused me to fill in what I did not understand--culturally, that is--with
        what I thought Camus was portraying: for example, Meursault. I initially
        thought Meursault was a Frenchman living in Algeria, enjoying the same rights
        as a Frenchman in France. In the discussion, however, it was brought up
        that Meursault was most likely a 'pied noir', a Frenchman born in Algeria,
        instead of a French-born Frenchman. 'Pieds noirs' were as alienated from
        mainland France as they were from Algeria, making up an in-between. \\

        I also learned about French--Algerian relations of the time period (1940s).
        Frenchmen (or people born of French ancestry), were under legal protection when
        concerning actions towards Arabs. It was much easier for an Arab to be executed
        for the murder of a Frenchman than a Frenchman to be executed for the murder of an
        Arab; in the event that an Arab murdered a Frenchman, hundreds of Arabs were
        murdered along with the original perpetrator of the crime. Noting that, the
        confrontation between Meursault and the Arab takes on a new light. I had initially
        thought that the Arab was the more powerful, frightening figure. After learning
        about Meursault's possible legal immunity, however, my perspective changed:
        Meursault was the more powerful figure. Not only did he have political immunity,
        anything the Arab did to Meursault would be met by a response several magnitudes
        more extreme. \\

        What I found most surprising, however, was that I had, for the most part,
        filled in character ethnicities based on what I thought they would be, without
        supporting my reasoning. The prison in which Meursault was thrown was one such
        case. I had assumed that his fellow inmates were Arabs. In the discussion,
        I learned that prisons were generally separated into French and Arab sections;
        Meursault, being a 'pied noir', was most likely put into the French section.
        This dramatically changed my perspective on the prison, and, more broadly, on
        character ethnicities: I went from blindly assigning what I \textbf{thought}
        was the right ethnicity to what would be the \textbf{most likely} ethnicity,
        which were not always the same.
    \end{mla}
\end{document}
