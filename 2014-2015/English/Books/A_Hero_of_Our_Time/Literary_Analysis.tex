\documentclass[12pt,a4paper]{article}

\usepackage[english]{babel}
\usepackage[autostyle]{csquotes}
\usepackage{ifpdf}
\usepackage{mla}
\usepackage{setspace}

\MakeOuterQuote{"}

\doublespacing

\begin{document}
    \begin{mla}{Tarik}{Onalan}{Calvert}{English 1}{13 October 2014}{A Literary Analysis of Lermontov's \textit{A Hero of Our Time}}
        \textit{A Hero of Our Time} is a Russian novel written by Mikhail Lermontov
        and translated by Vladimir and Dmitri Nabokov. The main character of the
        novel is Pechorin, a situational hero with a superfluous existence. Lermontov,
        by using foil characters to portray Pechorin from different perspectives,
        alters how Pechorin is viewed as a character in each of his interactions
        with the foil characters. Thus, Lermonotov expresses how perceptions do not
        reflect the truth because they are manipulated by our personal beliefs and
        experiences.\\

        Maksim Maksimych's ignorance of Pechorin's shortcomings is a product of Maksim's
        own shortcomings. Maksim would rather see Pechorin in his own distorted way
        than see him in reality.

        Pechorin "[is]...a charming fellow" (Lermontov 23) according to Maksim.
        Maksim only focuses on what he perceives as fitting of his ideal persona.
        Saying that Pechorin is "charming" is a representation of this ideal
        persona. Maksim is not the most charming character; he is rather coarse in
        both culture and speech. Maksim, to compensate for his own shortcomings,
        projects his ideal persona onto Pechorin, and refuses to believe that Pechorin
        is not a hero, and may well be a monster. Lermontov emphasizes how the Pechorin
        Maksim sees and the Pechorin in reality are two different people. Maksim's projection
        only covers up the imperfect Pechorin that he does not want to see. Even
        though "[Pechorin] cause[s] [Maksim] no end of trouble, [that]...is not what
        Maksim [chooses] to remember him by" (Lermontov 23). Lermontov explores how
        people, regardless of circumstance, only choose to percieve what they want to
        perceive, and how Maksim, ignoring all of Pechorin's mistakes, maintains that
        Pechorin is a "Hero" figure; he believes that it was "assigned, at [Pechorin's]
        birth, to have all sorts of extraordinary things happen to [him]" (Lermontov 23).
        Lermontov exposes a key fact with Maksim: Maksim is not upholding Pechorin as a
        hero. Instead, Maksim is uplifting his ideal hero, cast onto Pechorin, as a hero.
        Lermontov extends this to perceptions, that they are not the product of logic
        and reason, but more the product of emotion: love, hatred, envy, vanity. Maksim
        "is nonplussed" (Lermontov 34) after hearing Pechorin explain his reasoning for
        kidnapping Bela. Lermontov notes that what people cannot understand, they ignore,
        as it unbalances their overall perception of the world around them. Maksim
        cannot understand why Pechorin chose to kidnap Bela, so he passes over the
        reasons behind Pechorin's actions in leiu of his preferred reality.\\

        Grushnitsky's inability to see beyond his own shortcomings makes him unable
        to see that Pechorin is also a less than ideal character.

        Grushnitsky's envy for Pechorin is resultant from his pride; while Grushnitsky
        "has the reputation of an exceptionally brave man" (Lermontov 85), he cannot
        become the hero of a novel as Pechorin supposedly is. Grushnitsky
        "rushes [into battle] with closed eyes" (Lermontov 85) which, according to Pechorin
        "is not Russian courage" (Lermontov 85). Grushnitsky, almost in retaliation, creates
        an alternate image for himself, trying to imitate what seems to come to Pechorin so
        naturally. Lermontov shows how Grushnitsky, in his efforts to protect his pride,
        produces a false expectation of himself by comparing himself to someone he is not.
        When Pechorin expresses his indifference at Grushnitsky's relationship with
        Princess Mary, Grushnitsky, his pride hurt, says "\enquote{so much the worse
        for you}" (Lermontov 105). Essentially, Lermontov emphasizes Grushnitsky's lack
        of self-composure in the face of indifference. Lermontov seems to extend this
        to how Grushnitsky cannot see beyond his own faults into Pechorin's own. When
        Grushnitsky finally realizes that Pechorin is as imperfect as him, he "despise[s]
        [himself] and hate[s] [Pechorin]" (Lermontov 155) for trying to emulate the monster
        that had, moments before, promised that he would kill Grushnitsky without hesitation.
        Lermontov seems to explain, through Grushnitsky's final realization, that
        regardless of power and influence, perceptions will always be limited by what
        people want, not by what they have.\\

        Princess Mary's innocence in regards to social interaction makes it easy for
        Pechorin to manipulate her perception of him.

        After Pechorin ignores Princess Mary, she "accuses herself of having treated
        [Pechorin] coldly" (Lermontov 126), while it is actually Pechorin manipulating
        her. Princess Mary cannot see past the fact that while the men around her are
        constantly chasing her, Pechorin is indifferent to her existence. Lermontov
        seems to suggest that people try to adhere to a specific perspective with specific
        expectations. When Pechorin is indifferent to Princess Mary, Princess Mary
        cannot make sense of why Pechorin is indifferent. Her perception shifts away
        from \textit{who} Pechorin is to \textit{why} Pechorin is not paying attention
        to her. Pechorin effectively skips his introduction to Princess Mary, and immediately
        gets to occupying her thoughts, "[bringing] [her] to a point where [she can]
        convince [herself]" (Lermontov 131) to love Pechorin. Princess Mary does not
        see or look for Pechorin's true character. Much like Maksim Maksimych, she
        ignores Pechorin's idiosyncrasies for what matters to her, passion.\\

        After realizing the truth about Pechorin's character, Maksim Maksimych, Grushnitsky,
        and Princess Mary all experience disillusionment with Pechorin. After realizing
        that Pechorin is not the loyal, charming, witty person that he idealized, Maksim
        Maksimych is immediately put off by Pechorin, as if "know[ing] all along that
        [Pechorin] was a volatile fellow on whom one could not rely" (Lermontov 64).
        Grushnitsky, after realizing that Pechorin is more of the monster than the
        hero of a novel, says that "[he] despise[s] [himself]" (Lermontov 155) for trying to
        be like a monster. Princess Mary, after understanding that Pechorin will not reciprocate
        on her feelings, says that "[she] hate[s] [Pechorin]" (Lermontov 162). Every character,
        after realizing the truth about Pechorin, cannot accept his reality, and reject
        him for it. Lermontov exposes the simple fact that people would rather accept
        a lie than a truth if the truth were contrary to their perceptions.
    \end{mla}
\end{document}
