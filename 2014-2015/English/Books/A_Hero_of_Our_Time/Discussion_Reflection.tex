\documentclass{article}

\usepackage{setspace}

\doublespacing

\begin{document}
    \noindent{}Onalan, Tarik \\
    Interlake High School \\
    22 September 2014 \\

    \noindent\textit{A Hero of Our Time} by Mikhail Lermontov, translated by Vladimir and Dmitri Nabokov \\

    \noindent"How was your understanding of the cultural and contextual considerations of
    the work developed through the interactive oral?" \\

    \noindent{}Word count: 334\\

    Before participating in the discussion, I thought that the cultural references
    made in the novel were for the purpose of setting. Instead, the cultural references
    highlight limits in perception. Lermontov layers the setting under multiple
    perspectives to achieve this affect. Each perspective imposes a new limitation
    on what would be the original setting, so any judgments made would be limited
    as well. This is the fundamental limitation that comes up when trying to analyze
    Lermontov's development of Pechorin, because the imposed cultural perspective
    filters details that may be important for understanding why Lermontov makes
    Pechorin do what he does.

    Lermontov starts off the novel with cultural layering: \textit{Bela} is a story
    about Pechorin and Maxim Maximych, told by Maxim Maximych, written down by an
    unnamed narrator. Each character introduces another culture, another layer
    that limits the view of the setting. Before the discussion, I would have thought
    that Lermontov only uses the layered narratives as foils for Pechorin. While
    this is partly true, it ignores the fact that each character Lermontov creates
    ignores certain events that go on around them, or at least do not perceive them
    as being as significant as they actually are. Maxim can conspiracize about how
    Ossetians are trying to get a tip out of him, but he will never know exactly
    why or how they go around extracting the tip, because the linguistic gap between
    Maxim and the Ossetians prevents him from understanding what they are saying in
    their native tongue.

    Each limitation stacks on top of another, until someone reads the novel. Again,
    cultural perspective comes into play. Someone from America in 2014 may not
    know what was significant in Russia in 1840, leading to the loss of some information.
    Perhaps information is ignored because it goes against their cultural values.
    This was Lermontov's greatest difficulty; he can control the perspectives of
    his characters, but he cannot control the perspective of the reader, which
    essentially leaves his work at the mercy of the perception of the reader.
\end{document}
