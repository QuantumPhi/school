\documentclass[12pt,a4paper]{article}

\usepackage[english]{babel}
\usepackage[autostyle]{csquotes}
\usepackage{ifpdf}
\usepackage{mla}
\usepackage{tikz}

\usetikzlibrary{plotmarks}
\MakeOuterQuote{"}

\title{Betwixt the Sky and Earth}
\date{30 March 2015}
\author{Tarik Onalan}

\begin{document}
    \maketitle
    \noindent
    \begin{tikzpicture}
        \draw (0,0)--(\linewidth,0);
        \node[mark size=7.5pt,color=red] at (\linewidth/2,0) {\pgfuseplotmark{square*}};
    \end{tikzpicture}
    \\
    In this section, Hawthorne references the color red primarily through two aisles:
    flame and a scarlet letter. Note that it is "a" scarlet letter and not "the" scarlet
    letter, as in this section, Dimmesdale's letter is the primary focus. Towards the
    end of the excerpt, Hawthorne describes the appearance of "an immense letter...burning
    duskily through a veil of cloud". This combination of the letter and fire is reminiscent
    of Hester's first experiences with the scarlet letter, where it is described as being
    "deeply branded" and an agony to endure.
    \\
    \begin{tikzpicture}
        \draw (0,0)--(\linewidth,0);
        \node[mark size=7.5pt,color=green] at (\linewidth/2,0) {\pgfuseplotmark{square*}};
    \end{tikzpicture}
    \\
    Hawthorne describes the all-illuminating nature of the meteor, its "light [gleaming]
    far and wide over all the muffled sky". Muffled references the clouds that are present,
    and the fact that the light is able to penetrate through the "veil" of clouds suggests
    that it is unshrouding whatever the clouds are hiding---in this case, the earth.
    It is important to note that the meteor is out of the ordinary; instead of the
    sun or the moon, a foreign celestial object shedding light, almost as if another
    person were to provide another perspective on a situation.
    \\
    \begin{tikzpicture}
        \draw (0,0)--(\linewidth,0);
        \node[mark size=7.5pt,color=blue] at (\linewidth/2,0) {\pgfuseplotmark{square*}};
    \end{tikzpicture}
    \\
    Hawthorne frequently references celestial objects in this section, be it meteors
    or the sky. Celestial objects are put far off from earth, the most striking image
    of that being Dimmesdale, Pearl, and Hester holding hands and looking at the
    meteor lighting up the sky. This image specifically speaks to the contrast between
    the chaotic life on earth and the long, graceful arcs of celestial objects. This
    is why the image of the meteor illuminating the sky is so striking, because the
    "act" of illumination seems like a judgement of sorts.
    \\
    \begin{tikzpicture}
        \draw (0,0)--(\linewidth,0);
        \node[mark size=7.5pt,color=purple] at (\linewidth/2,0) {\pgfuseplotmark{square*}};
    \end{tikzpicture}
    \\
    Hawthorne, in this section, suggests that biased interpretation is present everywhere.
    The meteor casts a light with "a singularity of aspect"; people have "coloring,
    magnifying, and distorting" imaginations. However, that interpretation is also
    what provides illumination, because it is others' light that, in the end, defines
    what is seen. Just as the earth cannot become illuminated without the sun, people
    cannot be defined without other people illuminating their "letters".
    \newpage
    \textbf{Thesis:} In the scene with the meteor, Hawthorne suggests that reality
        is a product of others' perspective through light and dark imagery and symbol
        of the meteor.

    \textbf{TS1:} Early in the section, Hawthorne, through references to celestial objects
        and almost exclusively light imagery, emphasizes that it takes a separate perspective
        to illuminate shrouded secrets.
    \begin{itemize}
        \item "a light gleamed far and wide over all the muffled sky"
        \item "The great vault brightened, like the dome of an immense lamp"
    \end{itemize}
    Hawthorne immediately begins describing the meteor with "a light [that] gleamed
    far and wide over all the muffled sky". Two things stand out in this quote. The
    fact that the light gleamed "far and wide", as if it was all-encompassing---all-seeing,
    as it were---and the fact that the sky is "muffled". The muffle references the clouds,
    a veil over the earth, and the light from the meteor is penetrating the veil, illuminating
    all that is on the surface. It would be impossible to light up the entire town, no
    less the entire side of the planet, from on earth; it takes a separate perspective,
    the perspective of the meteor, or the moon, or the sun, to illuminate the earth.
    Hawthorne describes the "great vault [brightening], like the dome of an immense
    lamp". Here, the "great vault" is a reference to the sky, and is not important
    if mentioned alone. However, the fact that the image of the sky lit by the meteor is
    paralleled with the dome of a lamp is interesting, because a lamp is not lit from
    the outside; a lamp is lit from the inside. However, there is no difference. It takes
    the bulb of a lamp to reveal the patterns on a lamp dome, just as it takes the light
    of the meteor to reveal objects shrouded in the darkness.
    \\
    \textbf{TS2:} Towards the middle of the section, Hawthorne's descriptions of how
        people would try to prophesize based on supernatural events suggest that people
        only see what they want or expect to see.
    \begin{itemize}
        \item "the colored, magnifying, and distorting medium of his imagination"
        \item "the firmament itself should apear no more than a fitting page for his soul's history and fate"
    \end{itemize}
    \textbf{TS3:} In the last paragraph of the section, Hawthorne's combination of
        light and dark imagery, along with descriptions of the meteor, emphasize how
        for most people, reality is whatever society makes it.
    \begin{itemize}
        \item "no such shape as his guilty imagination gave it"
        \item "another's guilt might have seen another symbol in it"
    \end{itemize}

    \textbf{Conclusion:}
    Hawthorne's light and dark imagery, when combined with the symbol of the meteor,
    serve to reiterate the fact that what is "seen" is a reflection of what is shined
    upon something, and the perspective of the shined light can always alter what is
    perceived. All that prevents this lies betwixt the sky and the earth.
\end{document}
