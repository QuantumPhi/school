\documentclass[12pt]{article}

\usepackage[english]{babel}
\usepackage[autostyle]{csquotes}
\usepackage{ifpdf}
\usepackage{mla}
\usepackage{setspace}

\MakeOuterQuote{"}

\doublespacing

\begin{document}
    \begin{mla}{Tarik}{Onalan}{Calvert}{English}{13 October 2014}{A Literary Analysis of Lermontov's \textit{A Hero of Our Time}}
        \textit{A Hero of Our Time} is a Russian novel written by Mikhail Lermontov
        and translated by Vladimir and Dmitri Nabokov. The main character of the
        novel is Pechorin, a situational hero with a superfluous existence. Lermontov
        does not portray Pechorin as he himself sees Pechorin; instead, Lermontov
        hides Pechorin behind layers of perception, characters that each see him
        in a different light. Lermontov plays with the image of Pechorin as a child
        would play with a doll, twisting, stretching it, as if he had little regard
        for the consequences. Lermontov, by twisting perspective, is able to control
        readers' perceptions of Pechorin as a puppetmaster can control a puppet.
    \end{mla}
\end{document}
