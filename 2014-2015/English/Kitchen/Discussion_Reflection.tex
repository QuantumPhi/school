\documentclass[12pt,a4paper]{article}

\usepackage[english]{babel}
\usepackage[autostyle]{csquotes}
\usepackage{ifpdf}
\usepackage{mla}
\usepackage{setspace}

\doublespacing
\MakeOuterQuote{"}

\begin{document}
    \begin{mla}{Tarik}{Onalan}{Calvert}{English}{19 January 2015}{Discussion Reflection}
        \centerline{\textit{Kitchen} by Banana Yoshimoto, translated by Megan Backus}
        \noindent{}Word Count: 397 \\
        \textit{Kitchen} was a novel that I, at first read, could not understand. I did
        not understand what Yoshimoto meant to represent with her out-of-the-ordinary
        characters. Growing up in an age where identity--specifically sexual--is viewed
        as fluid , I did not understand how the existence of a transvestite, Eriko,
        was supposed to be "shocking", or otherwise odd. Not knowing the idea of a traditional
        family in Japan, I did not see how Mikage and Yuichi living together as friends
        was "weird". I did not see how a murder was supposed to be a surprising matter,
        growing up in a country where gun homicide rates are 20 times higher than any
        other developed country. \\
        In the class discussion, I learned how the mere existence of a transvestite in
        Japan was surprising. The first gender reassignment surgery in Japan took place
        in 1998; Yoshimoto writes of a man who undergoes gender reassignment surgery
        in \textbf{1988}. However, the big surprise came from learning
        about the traditional Japanese family unit: a dominant father, a mother, the
        children. Yuichi at first lives with his transvestite "mother", Eriko, and a
        girl "friend"--Mikage; after Eriko's death, Yuichi lives with Mikage, still in
        a non-romantic fashion. Yoshimoto challenges almost every element of the Japanese
        family tradition, and, through the discussion, I understood that this is Yoshimoto's
        purpose: shock-and-awe mixed with a reality check, that families--companionship--can
        exist in many forms. \\
        After gaining a basic understanding of Yoshimoto's characters, I was able to decipher
        the symbols Yoshimoto chooses to use in \textit{Kitchen}. Yoshimoto's recurring
        references to light and dark at first made me picture good and evil, though
        it did not feel appropriate to the novel. After learning that Yoshimoto was
        emphasizing the importance and universality of companionship, I revised my thoughts;
        light and dark were focused not on good and evil, but companionship and loneliness,
        respectively. Similarly, Yoshimoto's references to the kitchen were lost on me.
        I did not understand why, for example, Yoshimoto would have Mikage sleep in the kitchen.
        However, I had only the vision of the common American kitchen, not the small, much more
        intimate Japanese kitchen. After learning about the centrality of the Japanese
        kitchen in the home, I was able to understand why Mikage had such an intense attraction
        to kitchens; they served as a starting point for her emotional reconstruction while
        dealing with loss, a place to experience all of her emotions.
    \end{mla}
\end{document}
