\documentclass[12pt,a4paper]{article}

\usepackage[english]{babel}
\usepackage[autostyle]{csquotes}
\usepackage{ifpdf}
\usepackage{mla}

\MakeOuterQuote{"}

\title{Ananas}
\date{9 February 2015}
\author{Tarik Onalan}

\begin{document}
    \maketitle
    \centerline{Word Count: }
    In \textit{Kitchen}, written by Banana Yoshimoto and translated by Megan Backus,
    Yoshimoto highlights the fact that people find comfort in chaos through contrasting
    imagery. \textit{Kitchen} is, in in of itself, a novel of contrasts. Yoshimoto
    repeatedly contrasts images of light and dark, warm and cold, and noise and silence.
    The dark, cold, and silence are all controlled systems, but are, as a result, devoid
    of interaction. In contrast, the light, warmth, and noise are messy and imperfect.
    However, Yoshimoto, through Mikage's attraction to light, warmth, and noise suggests
    that that is the crux of life, and is something to be sought and embraced.
    Through contrasting imagery, Yoshimoto illustrates how chaos is inherent---and
    required---in life. \\

    Yoshimoto's contrast of light and dark calls back to a time when humans were
    primitive, and all life revolved around the fire. Mikage, in the much more modern
    time period of the 1980s, finds her own fire, revolving around "[t]ruly great people
    [that] emit a light that warms the hearts of those around them" (Yoshimoto 55). Yoshimoto
    suggests that, indeed, truly great people are like small fires, beacons of light in the
    darkness. However, the nature of the fire is still chaotic, as a fire can just as easily
    burn down a mountainside as it can thaw frozen hearts. Mikage, after her grandmother's
    death, begins to believe that life is a "truly dark and solitary path" (Yoshimoto 21),
    which is a direct and symbolic reference to her loneliness. Yoshimoto emphasizes
    the fact that darkness is lifeless through the words "solitary path." There is no
    guidance and no interaction in loneliness. Furthermore, Mikage notes that "when [a] light
    is put out, a heavy shadow of despair descends" (Yoshimoto 53). Yoshimoto reaffirms
    the fact that companionship is like a fire, giving off light, and that the shadows---the
    lifeless---take over in the darkness. After Eriko's death, Mikage's "mind was blank; in
    [her] eyes, everything was dark" (Yoshimoto 49). In Turkish, when one says their eyes
    are dark, it means they have lost consciousness. In Japanese, this would translate
    to "go to the end." Looking at the above quote from that perspective suggests that
    Yoshimoto is emphasizing the finality of darkness. There is no interaction at "the end."
    There is no life at "the end." On the other hand, Mikage states that the moon "[that]
    was almost full shed an incredible brightness" (Yoshimoto 61). Yoshimoto suggests
    that even when someone is shaken, the light of companionship is still strong. The fire
    still burns bright, and interactions can still occur. \\

    The warmth of the fire was just as important as the light it cast into the night.
    Mikage, while visiting her old apartment, notes that it is "cold and dark, [with] not
    a sigh to be heard" (Yoshimoto 22). The fact that Mikage notes the coldness of
    the apartment emphasizes its lifelessness. Just as a corpse gets cold over time,
    the apartment gets cold when there is no life, and, as such, feels dead to Mikage.
    Mikage even notes that "time died" in the apartment (Yoshimoto 22). Yoshimoto
    emphasizes the stagnant nature of the apartment; the dynamic chaos of life has
    left it. In contrast, when a spring storm strikes, Mikage notes how
    a "warm wind came roaring up" (Yoshimoto 28). The storm symbolizes the chaos of
    life, warm and swirling, just like the warm winds. Yoshimoto emphasizes how even
    though the storm can be destructive, it is simply a part of nature, just as chaos
    is a part of life. In that regard, Mikage notes that "it's cold [when she's] all alone,
    [and when] somebody's there... it's warm" (Yoshimoto 4). The fire of companionship
    will warm those around it.
\end{document}
