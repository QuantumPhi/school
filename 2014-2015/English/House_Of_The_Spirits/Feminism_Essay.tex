\documentclass[12pt,a4paper]{article}

\usepackage[english]{babel}
\usepackage[autostyle]{csquotes}
\usepackage{ifpdf}
\usepackage{indentfirst}
\usepackage{mla}

\MakeOuterQuote{"}

\begin{document}
    \begin{mla}{Tarik}{Onalan}{Calvert}{English}{4 May 2015}{I am (almost) a staunch (quasi-)"feminist" [sic]}
        \textit{House of the Spirits}, written by Isabel Allende and translated by Magda Bogin, is a
        "feminist" novel. The feminism that \textit{House of the Spirits} seems to align with most is
        the traditional liberal feminism (Fudge 1), that women should enjoy the same privileges as men.
        The most recognizable evidence for this is the emphasis on the issues of social equality and
        agency; Allende's portrayal of these issues in \textit{House of the Spirits} seem to stem from
        the distinct spheres of the worldly and the supernatural. "Worldly events", in this case, implies
        the rapes that women endured, and represents the issue of social equality; the supernatural,
        on the other hand, tends to represent the issue of agency.

        Equality is central to the liberal feminist ideal, and the rapes that occur in \textit{House
        of the Spirits} are an indirect reference to said equality. The first rape that Allende describes
        is that of Pancha Garcia by Esteban; however, Allende reveals that "before [Pancha], her mother---
        and before her, her grandmother---had suffered the same animal fate" (Allende 57). Initially,
        this seems like a counterexample to social equality. However, that is a simplistic analysis:
        Allende's emphasis of the repeated rapes through the generations highlights the existence of
        an accepted social inequality with regards to women. In her essay, \textit{Shunned}, Meredith
        Hall concurs, noting how her friend---a male---was able to fight back against injustice "because
        he felt powerful", but she was not, "[her] messy failure...feasted upon" (Hall 2). Later in the
        novel, Alba is raped by Esteban Garcia---interestingly, the grandchild of Pancha and Esteban---
        "another link in the chain of events that had to complete itself" (Allende 431). Allende's
        description of this cycle again suggests the social inequality women endure. Similarly, women
        of the New Left movement of the 1970s suggest that "heterosexuality is a compulsory institution
        designed to perpetuate the social power of men" (Krolokke 10). A rape is, in essence, the "powerful"
        abusing---attacking---the powerless, a statement of inequality. Ana's rape by the guards "in
        the presence of her lover" is yet another example of such inequality (Allende 412. However,
        while Ana is at the receiving end of institutionalized social inequality, she remains "an
        indomitable woman" (Allende 412). Allende's emphasis of Ana's push against inequality suggests
        that remaining strong against social inequality is key for any improvement. Interestingly, where
        this strength should come from is a contested topic: Erin Morales-Williams, in her essay, notes
        that "writing for [herself] has always brought relief", while Alice Walker states that only
        "collectively [women] can effect change" (Morales-Williams 1; Schnall 4). On one hand, there
        is strength from within, and on the other, there is strength in numbers. In this case, Allende's
        portrayal of Ana Diaz seems to agree more with Morales-Williams' approach to gaining social
        equality.

        Another key issue for liberal feminists is the need for agency; if a woman cannot control her
        own actions, how is there to be equality? Allende seems to use the supernatural in \textit{House
        of the Spirits} to symbolize agency. Given that Clara is the focal point of the novel with
        respect to spirits, most of the supernatural concerns Clara, but the Mora sisters and Rosa,
        among others, also interact with the supernatural. Esteban, in his marriage to Clara, desires
        to control her, but realizes that "if she continued living in a world of apparitions...[he]
        probably never [could]" (Allende 96). Allende, through the supernatural imagery, emphasizes
        how Clara is able to remain independent of Esteban. The "world of apparitions" is separate of
        the physical world, just how Clara is separate of Esteban. This is similar to the Suffragists'
        goal, to break out of the "cult of domesticity", and that women should "wield only indirect
        influence" (Krolokke 5). The concept of self-determination is further emphasized when Clara
        and Esteban are discussing the names of their yet unborn twins: "[Clara's] decision [is]
        inflexible" with regards to their names, even in the face of Esteban's attempts to frighten
        her to reconsider (Allende 114). Before delving into the analysis, one thing must be elaborated:
        this scene does not seem like a "supernatural" one. However, it is important to note that Clara,
        while pregnant, is very much detached from the world, and seldom comes down from her "Brahmanic
        refuge" (Allende 114). As such, this scene holds some supernatural significance. Allende's
        emphasis of Clara's passive resistance, so to speak, is yet another statement of agency.
        While Esteban is venting his rage, trying to make Clara do his bidding, Clara is doing as
        she pleases. Another example is Clara's increased separation from Esteban in the period
        after the earthquake, where she claims "to have lost her natural inclination for the flesh",
        and, as a result, decides to sleep in a room separate from Esteban (Allende 179). Again,
        even though Esteban tries to make her do his bidding, be it in a sexual nature or just getting
        her to be with him, Clara decides to do what she wants, going to her own room. It is important
        to note that again, the supernatural, while not explicitly described in this scene, is
        suggested by Clara's loss of her "inclination for the flesh", which implies an increased
        connection with the spiritual.

        Feminism in \textit{House of the Spirits} is a complex topic. There are many different types
        of feminism to choose from, and, from there, many issues to focus on. The issue of social
        equality is less a one-to-one symbol to the rapes than an inverse representation, a negative
		image of sorts. On the other hand, the issue of agency is, in essence, described through the
		independence of the spiritual realm from the physical realm, most heavily symbolized by the
		independence of Clara from Esteban, Clara with her three-legged table, and Esteban with his
		unending physical advances, be it with blows from his cane or rapes. But perhaps this is not
		the case. Surely, someone, somewhere could have a completely different opinion. However, one
		statement is irrefutable: \textit{House of the Spirits} is a feminist novel$^{[citation~needed]}$.
    \end{mla}

    \newpage
    \noindent
    \centerline{Works Cited}
	Allende, Isabel. \textit{The House of the Spirits}. Trans. Magda Bogin. New York. Print.
	\\
	Fudge, Rachel. ``Everything You Always Wanted to Know About Feminism But Were Afraid to
		\\\indent{Ask.'' \textit{Bitch}. 2005.}
	\\
	Hall, Meredith. "Shunned." 2003.
	\\
	Krolokke, Charlotte. \textit{Gender Communication Theories and Analyses}. 2005. Print.
	\\
	Morales-Williams, Erin. "Occupying Myself." \textit{the feminist wire}. 2012. Web.
		\\\indent{$<$http://www.thefeministwire.com$>$}
	\\
	Schnall, Marianne. "Conversation with Alice Walker." \textit{feminist.com}. 2006.
		Web. \\\indent{$<$http://www.feminist.com$>$}

	\newpage
	\noindent
	\centerline{Bibliography}
	Allende, Isabel. \textit{The House of the Spirits}. Trans. Magda Bogin. New York. Print.
	\\
	Ferguson, Ann and Hennessy, Rosemary. "Feminist Perspectives on Class and Work". The
		\\\indent{Stanford Encyclopedia of Philosophy, 2010. $<$http://plato.stanford.edu$>$.}
	\\
	Fudge, Rachel. ``Everything You Always Wanted to Know About Feminism But Were Afraid to
		\\\indent{Ask.'' \textit{Bitch}. 2005.}
	\\
	Gillibrand, Kirsten. "Ending Oppression and Empowering Women \& Girls Around The World."
		\\\indent{\textit{Huffington Post}. 2012. Web. $<$http://www.huffingtonpost.com$>$.}
	\\
	Hall, Meredith. "Shunned." 2003.
	\\ 
	Krolokke, Charlotte. \textit{Gender Communication Theories and Analyses}. 2005.
		Print.
	\\
	Morales-Williams, Erin. "Occupying Myself." \textit{the feminist wire}. 2012. Web.
		\\\indent{$<$http://www.thefeministwire.com$>$}
	\\
	Rich, Adrienne. "Split at the Root: An Essay on Jewish Identity."
	\\
	Schnall, Marianne. "Conversation with Alice Walker." \textit{feminist.com}. 2006.
		Web. \\\indent{$<$http://www.feminist.com$>$}
	\\
	Shayne, Julie. "feminist activism in Latin America."
\end{document}
