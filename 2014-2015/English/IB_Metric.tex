\documentclass[a4paper,12pt]{article}

\usepackage[english]{babel}
\usepackage[backend=biber,style=mla]{biblatex}
\usepackage[autostyle]{csquotes}

\newcommand{\mcite}{\mancite\cite}

\addbibresource{./IB_Assignment.bib}

\title{Written Assignment Reflection}
\date{12 June 2015}
\author{Tarik Onalan}

\begin{document}
    \maketitle

    \section{Issues}
        \begin{enumerate}
            \item Awkward expression, syntax, and inclusion of text:\\
            I read my essay out loud to make sure that there were no glaring syntax errors.
            \item A critical lack of (useful) text:\\
            I made sure that the text I used in my essay was not just plot, and was relevant,
            analyzable material.
            \item Topic sentences tend towards plot:\\
            This is a product of the following point. However, I still made sure that my topic
            sentences were more than just statements of plot by using the same test I used for
            a thesis. (``If I had a small essay, would this provide direction for my argument?'')
            \item Thesis did not provide clear direction to essay and was awkwardly expressed:\\
            Before writing my essay, I made sure that at a fundamental level, my thesis made sense.
            I did this by diagramming the logic behind the claim, and making sure that it was
            coherent. To make sure that it provided clear direction for my essay, I paid attention
            to my introduction, making sure that I defined the setting/topic that I would be working
            in.
            \item The essay did not communicate the thesis:\\
            This was a multi-part problem: lack of direction (listed), lack of text (listed),
            awkward syntax (not listed), flawed ideas (partially listed), etc. To fix this, I
            tried to make sure that after a person who had not read the book read my essay, they
            would understand what my argument was, and how I proved it.
        \end{enumerate}

    \section{Comparison}
        \begin{tabular}{|p{0.5\textwidth}|p{0.5\textwidth}|}
            \hline
            \bfseries Original Thesis & \bfseries Revised Thesis
            \\\hline
            Lermonotov expresses how perceptions do not reflect the truth because they are
            manipulated by our personal beliefs and experiences.
            &
            Lermontov, through the flawed judgment of Pechorin by foil characters, emphasizes
            that everybody perceives differently.
            \\\hline
        \end{tabular}

    \section{Selection}
        Simply put, Pechorin does not always do the most laudable things, like when he kidnapped Bela.
        Maksimych, instead of focusing on the fact that Pechorin kidnapped a young girl from her village,
        focuses on how ``Pechorin would make [Bela] some present every day'' \mcite[35]{book}. In the face
        of an offense like a kidnapping, a ``reconciliatory present'' hardly seems adequate: especially so
        after it becomes apparent that Pechorin merely seduced her to ease his boredom, hoping ``[t]he
        love of a wild girl [was better] than that of a lady of rank'' \mcite[48]{book}. Here, Lermontov
        illustrates  the gap between Maksimych's perception of Pechorin and reality; where Maksimych
        sees Pechorin giving presents, Pechorin thinks of using Bela to ease his boredom.
\end{document}