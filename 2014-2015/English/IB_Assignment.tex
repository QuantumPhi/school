\documentclass[a4paper,12pt]{article}

\usepackage[english]{babel}
\usepackage[backend=biber,style=mla]{biblatex}
\usepackage[autostyle]{csquotes}
\usepackage{setspace}

\newcommand{\mcite}{\mancite\cite}

\addbibresource{./IB_Assignment.bib}
\doublespacing

\title{A Villain of Their Time}
\date{12 June 2015}
\author{Tarik Onalan}

\begin{document}
    \begin{titlepage}
    \vspace*{2.0cm}
    \begin{center}
        Tarik Onalan
        \\[0.4cm]
        IB Candidate Number: \#\#\#\#\#\#-\#\#\#\#
        \\[1.2cm]
        IB Literature HL
        \\\vspace*{\fill}
        An analysis of foil characters and how they illustrate the fact that everybody
        perceives differently in Mikhail Lermontov's \textit{A Hero of Our Time},
        translated by Vladimir and Dmitri Nabokov.
        \\\vspace*{\fill}
        Word Count: 1213
        \\[0.4cm]
        12 June 2015
    \end{center}
\end{titlepage}


    %====================================================================================================%

    \maketitle

    %====================================================================================================%

    \textit{A Hero of Our Time} is a novel by Mikhail Lermontov and translated by Vladimir and
    Dmitri Nabokov. In his novel, Lermontov shows Pechorin, the main character, through the eyes
    of foil characters like Maksim Maksimych, Werner, and Princess Mary. Each character develops
    a different image of Pechorin; however, with every character, regardless of gender or age,
    Lermontov illustrates the existence of a judgment flaw: a gap between what they think are
    Pechorin's motives and his actual motives. Thus,
    %----------------------------------------------------------------------------------------------------%
    Lermontov, through the flawed judgment of Pechorin by foil characters, emphasizes that
    everybody perceives differently.

    %====================================================================================================%

    The judgment flaw of Lermontov's first foil character, Maksim Maksimych, is that he ignores
    Pechorin's faults. 
    %----------------------------------------------------------------------------------------------------%
    Simply put, Pechorin does not always do the most laudable things, like when he kidnapped Bela.
    Maksimych, instead of focusing on the fact that Pechorin kidnapped a young girl from her village,
    focuses on how ``Pechorin would make [Bela] some present every day'' \mcite[35]{book}. In the face
    of an offense like a kidnapping, a ``reconciliatory present'' hardly seems adequate: especially so
    after it becomes apparent that Pechorin merely seduced her to ease his boredom, hoping ``[t]he
    love of a wild girl [was better] than that of a lady of rank'' \mcite[48]{book}. Here, Lermontov
    illustrates  the gap between Maksimych's perception of Pechorin and reality; where Maksimych
    sees Pechorin giving presents, Pechorin thinks of using Bela to ease his boredom. Later, around
    the time when Pechorin and Maksimych reunite, Lermontov again emphasizes Maksimych's flawed
    perception of Pechorin. Maksimych, after hearing that Pechorin was in town, believes that
    ``[Pechorin would] come right away at a run'' \mcite[59]{book}, but is instead met with ``a
    friendly smile [and a] stretched out [hand]'' \mcite[62]{book}. Lermontov, in this scene, uses a
    different method: first, he introduces Maksimych's false image, then counters it with the
    ``real'' Pechorin. Even so, Lermentov still effectively demonstrates that Maksimych has a false
    perception of Pechorin. Maksimych still believes, even after learning that Pechorin gets bored
    with most \textit{everything}---even a woman's love---that Pechorin will come to him as an ``old
    pal''. Lermontov further emphasizes this when Maksimych says to Pechorin, ``[t]his is not the
    way I thought we would meet again'' \mcite[63]{book}, referring to Pechorin's ``cold'' treatment
    of Maksimych upon their reunion. Simply the fact that Maksimych had a different expectation for
    his reunion with Pechorin reinforces Lermontov's point:
    %----------------------------------------------------------------------------------------------------%
    Maksimych's flawed judgment is illustrated through Lermontov's comparison of Maksimych's relatively
    ``flawless'' image of Pechorin to the \textit{real} Pechorin.

    %====================================================================================================%

    Werner's judgment flaw, on the other hand, is that he assumes Pechorin's motives are mostly
    intellectual in nature, while in reality, Pechorin is much more affected by his emotions.
    %----------------------------------------------------------------------------------------------------%
    Later in \textit{Princess Mary}, when Werner finishes talking to Pechorin about being careful
    of marriage, he assumes that Pechorin would heed his advice, ``convinced that [he] had put
    [Pechorin] on his guard'' \mcite[129]{book}. However, Pechorin only focuses on the fact that
    rumors have been spread about him by Grushnitsky, and that ``[he would] have to pay''
    \mcite[129]{book}. Here, Lermontov unearths an obvious contrast between Werner's image of
    Pechorin and Pechorin himself. On one hand, a forced marriage could end Pechorin's current
    lifestyle, on the other, a personal affront of sorts. Where Werner believes Pechorin will focus
    on the important information: that Pechorin should beware of being obliged to marry, Pechorin
    focuses on the fact that Grushnitsky spread a rumor about him. Lermontov, through this contrast,
    emphasizes Werner's false image of Pechorin. When Werner is first introduced, Lermontov outlines
    Werner's false image: as he is having a philosophical discussion with Pechorin, Pechorin remarks
    that ``between [them] there can be no exchange of feelings and thoughts'' \mcite[92]{book}.
    While this is Pechorin's own statement, it can speak for Werner's perspective of Pechorin;
    Werner's image of Pechorin is governed by what he is ``shown'', per se. If Pechorin chooses not
    to exchange feelings or thoughts, then all Werner will see in Pechorin is the intellectual.
    Lermontov, however, shatters this image during Pechorin's duel with Grushnitsky. Before the duel,
    Werner asks Pechorin to immediately expose the trickery of Grushnitsky; Pechorin simply responds
    that he wants ``[Grushnitsky to] hugger-mugger a little'' \mcite[150]{book}. There is no
    difference in the end result---a bullet in Pechorin's weapon---whether he does it earlier or
    later; the only difference is that Pechorin gets to see Grushnitsky struggle with his own
    reservations about the duel. Essentially, Pechorin does it as a form of entertainment for
    himself, not for efficiency or strategy. Lermontov's contrast of Pechorin's ``entertainment''
    and Werner's intellectualized image of Pechorin reinforces the fact that Werner's image is
    false, and,
    %----------------------------------------------------------------------------------------------------%
    ultimately, Lermontov's contrast of Werner's image of Pechorin and Pechorin's emotions
    uncovers the fact that Werner's judgment of Pechorin is flawed.

    %====================================================================================================%

    Finally, Princess Mary: her judgment flaw is that she assumes that Pechorin is a grand,
    upper-class figure, while Pechorin is actually vain, and acts on his own desires.
    %----------------------------------------------------------------------------------------------------%
    Princess Mary's first introduction to Pechorin does not come courtesy of himeself; it comes
    instead through her mother. After Princess Ligovsky describes Pechorin's ``exploits [to Princess
    Mary]...in [her] imagination, [Pechorin] became the hero of a novel'' \mcite[93]{book}. This,
    especially after Werner mentions that Princess Ligovsky must have met Pechorin ``in Petersburg
    [] at some fashionable reception'' \mcite[93]{book}, makes Pechorin seem like the classic
    upper-class figure. Lermontov continues to construct this image during the ball when Pechorin
    protects her from a drunken man, and ``[is] rewarded by a deep, wonderful glance'' \mcite[109]{book}.
    Here, Lermontov strengthen's Mary's image of Pechorin as being a grand figure using the stereotypical
    ``male hero saves maiden'' image. However, Lermontov quickly disproves Mary's image when she
    openly declares her love for him. Pechorin, in response, says curtly: ``What for?'' \mcite[133]{book}.
    This transaction is interesting because it occurs immediately after Princess Mary considers the
    possibility that Pechorin could just be toying with her ``to trouble [her] soul, and then leave
    [her]'' \mcite[133]{book}, but, in the end, displaying ``tender trust'' \mcite[133]{book}. Suddenly,
    the phrase ``What for?'' has a different meaning: Pechorin does not care. Lermontov, through
    this, counters Mary's grand image of Pechorin with the reality: Pechorin is more interested in
    himself than the people around him. Lermontov, at the end of \textit{Princess Mary}, makes a
    final blow to Mary's image of Pechorin, when Pechorin questions Princess Mary with: ``Am I not
    right that even if you loved me, from that moment on you despise me?'' \mcite[162]{book}. After
    all, Pechorin could not marry her; he was just using her to prevent himself from ``[feeling]
    bored and oppressed'' \mcite[162]{book}. Lermontov emphasizes Pechorin's vanity, that his need
    for stimulation takes priority; and, through this, exposes the flaw in Mary's judgment.
    %----------------------------------------------------------------------------------------------------%
    Princess Mary saw Pechorin as he was initially presented to her, and Lermontov illustrates Princess
    Mary's judgment flaw through her relationship with Pechorin, and how Pechorin treats her through
    the course of their relationship.

    %====================================================================================================%

    To recapitulate: Maksim Maksimych sees Pechorin as relatively ``flawless'' when compared to
    ``real'' Pechorin; Werner sees Pechorin as almost exclusively intellectual, while ``real''
    Pechorin is not always in control of his emotions; and Princess Mary sees Pechorin as a
    grand upper-classman, while he is, in fact, vain and acts in response to his desires.
    Every one of these judgments is flawed, and every one of them is different. Therein lies
    Lermontov's argument: everybody perceives differently. Lermontov's use of disordered
    chronology, his use of layered narrative, the unnamed narrators redactions of Pechorin's
    journal: all were simply examples of Lermontov's argument. Everybody perceives differently
    because nobody has all of the information available to them: not even the reader.

    %====================================================================================================%

    \newpage
    \printbibliography
\end{document}