%!TEX program = xelatex

\documentclass[a4paper]{article}

\usepackage{polyglossia}
\setdefaultlanguage{english}

\usepackage{amsmath}
\usepackage[backend=biber]{biblatex}
\usepackage[autostyle]{csquotes}
\usepackage{fontenc}
\usepackage{graphicx}
\usepackage{hyperref}
\usepackage{parskip}

\newcommand{\HRule}{\rule{\linewidth}{0.5mm}}

\addbibresource{./Internal_Assessment.bib}
\everymath{\displaystyle}

\title{Investigating Fourier's Solution to the Heat Equation\\$\frac{\partial{u}}{\partial{t}}-\alpha\nabla^{2}u=0$}
\date{June 2015}
\author{Tarik Onalan}

\begin{document}
    \begin{titlepage}
    \begin{center}
        \HRule \\[0.4cm]
        { \large \bfseries Investigating Fourier's Solution to the Heat Equation \\[0.4cm] }
        \HRule \\[1.5cm]

        Tarik Onalan
        \\[0.4cm]
        \#\#\#\#\#\#-\#\#\#\#
        \\[0.4cm]
        SCHOOL
        \\[0.4cm]
        Word Count:
        \vfill

        {\large June 2015}
    \end{center}
\end{titlepage}
    
    %--------------------------------------------------------------------------------------------------%

    \maketitle

    %--------------------------------------------------------------------------------------------------%

    \tableofcontents

    %--------------------------------------------------------------------------------------------------%

    \section{Introduction}

        The heat equation is used to describe the distribution of heat in an object over time. While
        not introduced by Fourier, it is a method for solving the equation proposed by Fourier in
        his \textit{Théorie analytique de la chaleur}---translated as ``an analytical theory of
        heat''---that we will be investigating. I chose to investigate Fourier's solution to the heat
        equation because it is through his solution that the Fourier Transform, a popular tool in signal
        processing, was defined. I frequently use the Fourier Transform while doing data analysis, and I
        wanted to learn how and why it was created.

        \begin{center}
            \includegraphics{./heat_eq.png}
        \end{center}

    %--------------------------------------------------------------------------------------------------%

    \section{Solution}
        \subsection{Defining the heat equation}
            The heat equation (as defined in a rod) is as follows:
            \begin{equation}
                u_{t} = \alpha u_{xx}
            \end{equation}
            where $u = u\left(x,t\right)$ is a function of two variables $x$ and $t$.
            \begin{itemize}
                \item $x$: position, where $x \in [0,L]$ and $L$ is the length of the rod
                \item $t$: time, where $t\geq0$
            \end{itemize}
            assuming the initial condition
            \begin{equation}
                u\left(x,0\right) = f\left(x\right) \quad \forall x \in [0,L]
            \end{equation}
            and the following boundary conditions
            \begin{equation}
                u\left(0,t\right) = 0 = u\left(L,t\right) \quad \forall t > 0
            \end{equation}

            \begin{center}
                \includegraphics{./temp_dist.png}
            \end{center}

        \subsection{Our objective}
            We are looking for nontrivial solutions to to $\left(1\right)$ that satisfy the conditions defined by
            equations $\left(2\right)$ and $\left(3\right)$. We still need more information, however. Let us look at equation
            $\left(1\right)$ again:
            \[
                u_{t} = \alpha u_{xx}
            \]
            We can rearrange this equation so that it appears as follows:
            \[
                u_{t} - \alpha u_{xx} = 0
            \]
            What does this mean? The equation is a linear homogeneous partial differential equation,
            meaning that $u$ can be written as the product
            \begin{equation}
                u\left(x,t\right)=X\left(x\right)T\left(t\right)
            \end{equation}
            through separation of variables.

        \subsection{Solving the constituent equations $T\left(t\right)$ and $X\left(x\right)$}
            \[
                \begin{cases}
                    u_{t} = X\left(x\right)T'\left(t\right) \\
                    u_{xx} = X''\left(x\right)T\left(t\right) \\
                \end{cases}
            \]
            \\
            \[
                X\left(x\right)T'\left(t\right) = \alpha X''\left(x\right)T\left(t\right) \\
            \]
            \[
                \frac{T'\left(t\right)}{\alpha T\left(t\right)} = \frac{X''\left(x\right)}{X\left(x\right)}
            \]
            One thing to notice at this point is that both sides, though dependent on two independent
            variables, are equivalent. This means that both are equal to some constant that we will
            call $-\lambda$ (the negative is for simplicity later on).

            Now we have the following:
            \[
                \frac{T'\left(t\right)}{\alpha T\left(t\right)} = \frac{X''\left(x\right)}{X\left(x\right)} = -\lambda
            \]
            \[
                \begin{cases}
                    \frac{T'\left(t\right)}{\alpha T\left(t\right)} = -\lambda \\
                    \frac{X''\left(x\right)}{X\left(x\right)} = -\lambda
                \end{cases}
            \]
            Let us take the first equation, $\frac{T'\left(t\right)}{\alpha T\left(t\right)} = -\lambda$:
            \[
                \frac{1}{T\left(t\right)}\cdot\frac{dT}{dt} = -\lambda \alpha
            \]
            \[
                \int{\frac{1}{T\left(t\right)}}dT = -\int{\lambda \alpha}dt
            \]
            \[
                \ln{T\left(t\right)} + K_1= -\lambda \alpha t + K_2
            \]
            \[
                T\left(t\right) = e^{-\lambda \alpha t + K}
            \]
            \begin{equation}
                T\left(t\right) = A e^{-\lambda \alpha t}
            \end{equation}
            Now let us take the second equation, $\frac{X''\left(x\right)}{X\left(x\right)} = -\lambda$
            \[
                \frac{d^{2}X}{dx^{2}} = -\lambda X\left(x\right)
            \]
            This may seem daunting at first, but let us quickly observe a few things about this
            differential equation:
            \begin{itemize}
                \item Derived twice, it has a coefficient $-\lambda$
                \item Derived twice, it is still the same basic function
            \end{itemize}
            The latter point is very interesting. Only one other function remains the same after
            deriving: $e^{x}$. We have found the characteristic function for $X\left(x\right)$. All that remains
            is calculating the coefficients ($a$, in this case):
            \[
                \frac{d^{2} e^{ax}}{dx^{2}} = -\lambda e^{ax}
            \]
            \[
                a^{2} e^{ax} + \lambda e^{ax} = 0
            \]
            \[
                e^{ax}\left(a^{2} + \lambda\right) = 0
            \]
            Given that $e^{ax} \neq 0$:
            \[
                a^{2} + \lambda = 0
            \]
            \[
                a = \pm\sqrt{-\lambda}
            \]
            Therefore,
            \[
                X\left(x\right) = \sum_{b \in a}{B e^{bx}}
            \]
            \begin{equation}
                X\left(x\right) = B e^{\sqrt{-\lambda}x} + C e^{-\sqrt{-\lambda}x}
            \end{equation}

        \subsection{Limiting $\lambda$ to nontrivial solutions}
            Let us begin with $\lambda < 0$, and consider equation $\left(6\right)$:
            \[
                X\left(x\right) = B e^{\sqrt{\lambda}x} + C e^{-\sqrt{\lambda}x}
            \]
            Now, let us consider the boundary condition defined by equation $\left(3\right)$:
            \[
                X\left(0\right) = 0 = B e^{\sqrt{\lambda} \cdot 0} + C e^{-\sqrt{\lambda} \cdot 0}
            \]
            \[
                X\left(0\right) = 0 = B e^{0} + C e^{0}
            \]
            Which would mean that $B = 0 = C$, a trivial solution.
            
            Now let us consider $\lambda = 0$, again with equation $\left(6\right)$:
            \[
                X\left(0\right) = 0 = B e^{\sqrt{0}x} + C e^{-\sqrt{0}x}
            \]
            \[
                X\left(0\right) = 0 = B e^{0} + C e^{0}
            \]
            Which would again mean that $B = 0 = C$.

            Thus, $\lambda > 0$. With this knowledge, we can make a couple of changes to equation $\left(6\right)$
            using Euler's formula:
            \[
                X\left(x\right) = B e^{i\sqrt{\lambda} x} + C e^{-i\sqrt{\lambda} x}
            \]
            This is where $-\lambda$ becomes preferable to $\lambda$, as it only becomes possible to use Euler's
            formula on the exponent when it is $\sqrt{-\lambda} x$, as we can factor out an $i$ and make it a
            complex number. The expansion of the above equation yields
            \[
                X\left(x\right) = B\left(\cos{\left(\sqrt\lambda x\right)} + i\sin{\left(\sqrt\lambda x\right)}\right) + C\left(\cos{\left(\sqrt\lambda x\right)} - i\sin{\left(\sqrt\lambda x\right)}\right)
            \]
            Grouping like terms results in the equation
            \[
                X\left(x\right) = B\sin{\left(\sqrt\lambda x\right)} + C\cos{\left(\sqrt\lambda x\right)}
            \]
            From there, solving for the boundary conditions defined in equation $\left(3\right)$ yields:
            \[
                X\left(0\right) = 0 = B\sin{\left(\sqrt\lambda \cdot 0\right)} + C\cos{\left(\sqrt\lambda \cdot 0\right)}
            \]
            \[
                X\left(0\right) = 0 = B\sin{0} + C\cos{0}
            \]
            Which means that $C = 0$. However, another piece of information can be gained from
            this. If we look closely at the following equation, given $C = 0$:
            \[
                X\left(L\right) = 0 = B\sin{\left(\sqrt\lambda \cdot L\right)}
            \]
            We can see that we can calculate possible values for $\sqrt\lambda$
            \[
                \sin{\left(\sqrt\lambda \cdot L\right)} = 0
            \]
            \[
                \sqrt\lambda = \frac{n \pi}{L}
            \]

        \subsection{Finishing the proof}
            Now that $\lambda$ is defined, we can apply it to our functions $X\left(x\right)$ and $T\left(t\right)$
            \[
                \begin{cases}
                    X\left(x\right) = B \sin\left(\frac{n \pi x}{L}\right) \\
                    T\left(t\right) = A e^{-\frac{n^{2} \pi^{2} \alpha t}{L^{2}}}
                \end{cases}
            \]
            We can now apply this to equation $\left(4\right)$, yielding the particular solution
            \[
                u_{par}\left(x,t\right) = B_{n} \sin\left(\frac{n \pi x}{L}\right) \cdot e^{-\frac{n^{2} \pi^{2} \alpha t}{L^{2}}}
            \]
            Computing the summation of $u_{par}$ for all $n$ yields the general solution
            \[
                u\left(x,t\right) = \sum_{n=1}^{\infty}{u_{par}}
            \]
            \[
                u\left(x,t\right) = \sum_{n=1}^{\infty}{B_{n} \sin\left(\frac{n \pi x}{L}\right) \cdot e^{-\frac{n^{2} \pi^{2} \alpha t}{L^{2}}}}
            \]
            Let us now solve for the initial condition stated in equation $\left(2\right)$
            \[
                u\left(x,0\right) = \sum_{n=1}^{\infty}{B_{n} \sin\left(\frac{n \pi x}{L}\right) \cdot e^{-\frac{n^{2} \pi^{2} \alpha 0}{L^{2}}}}
            \]
            \[
                u\left(x,0\right) = \sum_{n=1}^{\infty}{B_{n} \sin\left(\frac{n \pi x}{L}\right) \cdot e^{0}}
            \]
            \[
                u\left(x,0\right) = \sum_{n=1}^{\infty}{B_{n} \sin\left(\frac{n \pi x}{L}\right)}
            \]
            By the definition of the Fourier Transform, we can now define $B_{n}$
            \[
                B_{n} = \frac{2}{L} \int_{0}^{L}{f\left(x\right) \sin\left(\frac{n \pi x}{L}\right)} dx
            \]
            and solve for $u\left(x,t\right)$, finishing the problem
            \[
                u\left(x,t\right) = \sum_{n=1}^{\infty}{\frac{2}{L} \left(\int_{0}^{L}{f\left(x\right) \sin\left(\frac{n \pi x}{L}\right)} dx\right) \sin\left(\frac{n \pi x}{L}\right) \cdot e^{-\frac{n^{2} \pi^{2} \alpha t}{L^{2}}}}
            \]

    %--------------------------------------------------------------------------------------------------%

    \section{Reflection}

        \vfill

    %--------------------------------------------------------------------------------------------------%

    \nocite{fourierj}
    \printbibliography
\end{document}