%!TEX program = xelatex

\documentclass[a4paper,twocolumn]{article}

\usepackage{polyglossia}
\setdefaultlanguage{english}

\usepackage[backend=biber]{biblatex}
\usepackage[autostyle]{csquotes}
\usepackage{fontenc}

\newcommand{\HRule}{\rule{\linewidth}{0.5mm}}

\title{Investigating Fourier's Heat Equation\\$\frac{\partial{u}}{\partial{t}}-\alpha\nabla^{2}u=0$}
\date{June 2015}
\author{Tarik Onalan}

\begin{document}
    \maketitle

    %--------------------------------------------------------------------------------------------------%

    \tableofcontents

    %--------------------------------------------------------------------------------------------------%

    \section{Introduction}
    \addcontentsline{toc}{section}{Introduction}

        The heat equation is used to describe the distribution of heat in an object over time. While
        not introduced by Fourier, it is a method for solving the equation proposed by Fourier in
        his \textit{Théorie analytique de la chaleur}---translated as ``an analytical theory of
        heat''---that we will be investigating. I chose to investigate Fourier's solution to the heat
        equation because it is through his solution that the Fourier Transform, a popular tool in signal
        processing, was defined. I frequently use the Fourier Transform while doing data analysis, and I
        wanted to learn how and why it was created.

    %--------------------------------------------------------------------------------------------------%

    \section{Proof}
    \addcontentsline{toc}{section}{Proof}

        \subsection{Visual Proof}


    \printbibliography
\end{document}