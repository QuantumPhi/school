\documentclass[a4paper,12pt]{article}

\usepackage[english]{babel}
\usepackage[backend=biber,style=mla]{biblatex}
\usepackage[autostyle]{csquotes}
\usepackage{ifpdf}
\usepackage{mla}

\addbibresource{./Ethics_Assessment.bib}

\begin{document}
    \begin{mla}{Tarik}{Onalan}{Kilby}{Theory of Knowledge}{11 June 2015}{Ethics Assessment}
        \section{Argument}
            Are embryos humans? Technically, yes. They have nearly the same genes as every other
            human on the planet. If an embryo is a human, however, how can you justify ``killing''
            it by extracting stem cells from it? That, I feel, is another question. While I would
            have to classify an embryo as ``human'' from a purely genetic standpoint, I would
            not call it ``alive'', thus making it impossible to kill in the first place. An embryo
            is, in my mind, similar to a brain-dead individual. Yes, it is human: it has all
            of the features---the genes---that define a human. However, it is not a functional
            organism; the brain---the control centre---is no longer functional.

            I believe it is apparent that the crux of my argument is that the embryo does not have
            a functional brain, which makes it inanimate. However, this introduces another question:
            is being brain-dead (or simply brain-absent) the same as being dead? This question
            can be discussed as an entirely new argument, but, for the sake of simplicity, I will
            define death (or, in the case of the embryo, never being alive) as either:
            \begin{enumerate}
                \item ``Irreversible cessation of circulatory and respiratory functions''
                \item ``Irreversible cessation of all functions of the entire brain, including the
                    brain stem''
            \end{enumerate}
            \cite{death}
            
            Let us lay out the information we have now:
            \begin{enumerate}
                \item A human is any entity with roughly the same genes as the rest of the human population
                \item Death is defined by one of the two occurring:
                \begin{enumerate}
                    \item breathing/circulation stopping function
                    \item brain stopping function
                \end{enumerate}
                \item An embryo is a human, as defined by point 1
                \item An embryo is ``dead'' (as in, not alive), as defined by point 2
            \end{enumerate}
            With this, we can finally approach the actual question, which I have intentionally not
            mentioned until now: \textit{Should embryonic stem cell research be supported? Is it
            ethical?}

            Let us now set up the possible outcomes. The ``null hypothesis'', that embryonic stem cell
            research is not ethical and should not be supported; and the ``alternative hypothesis'',
            that embryonic stem cell research is ethical and should be supported. The basis for our
            ``null hypothesis'' stems primarily from the following two areas:
            \begin{enumerate}
                \item ``The destruction of an embryo is akin to killing human life''
                \item ``Embryonic stem cells have not been successfully used to help cure disease''
            \end{enumerate}
            \cite{beliefs}

            The first point we disproved in the preceding paragraph. How about the second point? This
            is one of those statements that is technically correct, but practically incorrect. Yes,
            it is true that embryonic stem cells have not been used to help cure disease. However, it
            would never be possible to have embryonic stem cells cure diseases if they were not allowed
            to be researched in the first place, hence the practical incorrectness. However, simply
            saying the statement is \textit{practically} incorrect is not enough; I need to prove that
            it is ``absolutely'' incorrect.

            The statement claims that embryonic stem cells have not been successfully used to help
            cure disease. Let us assume that by disease, it is meant human disease, as to give the
            statement the benefit of doubt. Now, the embryonic stem cell does not have to cure the
            disease at this point; all it has to do is help. In that case, let me pose a question:
            if a drug that can cure a disease is tested on animals in a lab and proven effective, is
            that helping cure the disease? Given that the lab animals are designed to function similarly
            to humans (in the physiological sense) \cite{NIH}, it would be prudent to assume that
            most medical tests carried out on lab animals can be translated to human test subjects.

            So what? So what that a test carried out on an animal produces similar results on a human?
            That fact becomes interesting when you learn that embryonic stem cells were used to help
            restore motor function in rats with spinal cord damage \cite{stem}. Spinal
            cord damage is a disease, by definition: ``a disorder of a structure or function that
            affects all or part of an organism'' \cite{oxford}. Given that embryonic stem cells
            could help rats, we could state that it serves as a good proof-of-concept for human trials.
            Lab tests are not perfect; while animals are good at modeling human physiology, there are
            functions in the human body that differ. Thus, it would only be safe to assume that the
            lab test would confirm the possibility of a cure, which is help.

            Now, the second statement is disproved. In short, our ``null hypothesis'' has been refuted,
            and our ``alternate hypothesis'', that embryonic stem cell research is ethical and should
            be researched, is confirmed. 
        \section{Metrics}
            \subsection{Obstacles}
                \begin{itemize}
                    \item Merely expressing your preference---A preference is individualized; one person
                        may like chocolate ice cream, another may not. Merely expressing your preference
                        in a moral argument is analogous to telling someone who likes chocolate ice cream
                        that they (the person who likes chocolate ice cream) do not like chocolate ice
                        cream.
                    \item Merely expressing your feelings---The same issue as above: feelings about a
                        topic change from person to person, so basing an argument on feelings is insubstantial.
                    \item Merely expressing what you think---Thinking also varies from individual to
                        individual. Stating a thought is no different than stating a preference.
                    \item Citing majority opinion---Just because a majority thinks a certain way does
                        not guarantee the accuracy of that opinion. A majority can have power, but
                        it does not necessarily have truth.
                    \item Appeal to a moral authority---An ideal argument can function without the support
                        of a moral authority. The existence and interpretation of a moral authority (like
                        god) are all variable, again subject to the first three obstacles of individual
                        perception.
                \end{itemize}
            \subsection{Objectives}
                \begin{itemize}
                    \item Conceptual clarity---Main concepts should be clearly defined. 
                    \item Accurate information---Using facts instead of anecdotal evidence.
                    \item Following the rules of logic---Recognize the connection between ideas.
                    \item Impartiality---Do not favor an outcome.
                    \item Keeping a cool head---Do not get caught up in the ``heat'' of emotions.
                    \item Appealing to justifiable moral principles---Declare how a rational, free being
                        must act.
                \end{itemize}
        \section{Analysis}
            \subsection{Obstacles}
                \begin{itemize}
                    \item \textbf{3}---By the time I began the argument, I already had an idea of what I
                        thought the correct response would be. I preferred one response (positive response).
                        It shows in my tone, and occasionally in my writing style, when the sentences become
                        more terse.
                    \item \textbf{4}---I feel that there are not many emotional statements in my argument,
                        but I did notice that I did sound slightly dry at times, which gives away my emotions.
                    \item \textbf{2}---When disproving the statement ``Embryonic stem cells have not been
                        successfully used to help cure disease'' in the latter portion of my argument, I
                        did not supply good evidence for my argument. Especially when I was talking about
                        the lab animals, I felt that I was stating what I thought was true.
                    \item \textbf{2}---I cited a majority opinion by extension, per se. One of my citations
                        was for the Universal Determination of Death Act; this act was passed by a majority
                        opinion by a group of individuals voted into power through majority opinion. While
                        I may agree with the definition given by the act, it does not change the fact that
                        I cited majority opinion.
                    \item \textbf{5}---I do not appeal to a moral authority in my argument.
                \end{itemize}
            \subsection{Objectives}
                \begin{itemize}
                    \item \textbf{3}---While I do try to define terms that are important, such as being human,
                        dead, or diseased, I do not define them well; the terms are still relatively vague,
                        especially when I use them in confusing ways (``A human is any entity with roughly the
                        same genes as the rest of the human population'')
                    \item \textbf{4}---I used peer-reviewed, government issued, or non-profit organization
                        citations for all of my out-of-document research.
                    \item \textbf{3}---I lay out my step-by-step thought process when analyzing the argument.
                        I try to make as few logical jumps as possible, but I do recognize that I do skip steps
                        in my argument; the section with the lab animals was particularly problematic, and I
                        believe that the connection between the cited paper and my conclusion---that the
                        second statement is false---is slightly rushed and incomplete.
                    \item \textbf{2}---I favored the positive outcome. I want embryonic stem cell research to
                        be supported, regardless of whether or not stem cells came from embryos or near-infants.
                        My logic jumps and expression of thinking are products of this.
                    \item \textbf{2}---My writing began to take on a tone that was slightly too dry to be called
                        emotionless. While I would classify it as a colder emotional output, it was an output
                        nonetheless. An example of the ``dryness'' is here: ``However, it would never be possible
                        to have embryonic stem cells cure diseases if they were not allowed to be researched in
                        the first place, hence the practical incorrectness.''
                    \item \textbf{1}---I do not explicitly define a justifiable moral principle.
                \end{itemize}
        \printbibliography
    \end{mla}
\end{document}