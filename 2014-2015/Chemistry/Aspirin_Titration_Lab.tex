\documentclass[a4paper]{article}

\usepackage[version=3]{mhchem}
\usepackage{siunitx}

\title{Analyzing the Effectiveness of Various Types of Aspirin}
\date{9 February 2015}
\author{Tarik Onalan}

\sisetup{ per-mode=fraction }

\begin{document}
    \maketitle
    \section{Price}
        Cost per gram:
        \begin{itemize}
            \item Baby Aspirin: $\frac{\SI{2.99}[\$]{}}{1~pack}\cdot\frac{1~pack}{36~tablets}\cdot\frac{1~tablet}{\SI{81}{\mg}}\cdot\frac{\SI{1000}{\mg}}{\SI{1}{g}}=1.02\frac{\$}{\si{\g}}$
            \item Normal Aspirin: $\frac{\SI{11.99}[\$]{}}{1~pack}\cdot\frac{1~pack}{300~tablets}\cdot\frac{1~tablet}{\SI{325}{\mg}}\cdot\frac{\SI{1000}{\mg}}{\SI{1}{g}}=0.12\frac{\$}{\si{\g}}$
        \end{itemize}
        Cost per tablet:
        \begin{itemize}
            \item Baby Aspirin: $\frac{\SI{2.99}[\$]{}}{1 pack}\cdot\frac{1~pack}{36~tablets}=\frac{\SI{0.08}[\$]{}}{1~tablet}$
            \item Normal Aspirin: $\frac{\SI{2.99}[\$]{}}{1 pack}\cdot\frac{1~pack}{36~tablets}=\frac{\SI{0.04}[\$]{}}{1~tablet}$
        \end{itemize}
    \section{Performance}
        \begin{enumerate}
            \item We used \SI{4.5}{\ml} and \SI{16.5}{\ml} of base to titrate the baby and normal aspirin, respectively.
            \begin{itemize}
                \item Baby Aspirin: $\SI{4.5}{\ml}\cdot\SI{1}{\g\per\ml}\cdot\SI{0.001}{\ml\per\l}\cdot\SI{0.02}{\mol\per\g}=\SI{0.00009}{\mol}$
                \item Normal Aspirin: $\SI{16.5}{\ml}\cdot\SI{1}{\g\per\ml}\cdot\SI{0.001}{\ml\per\l}\cdot\SI{0.02}{\mol\per\g}=\SI{0.00033}{\mol}$
            \end{itemize}
            \item Baby aspirin neutralized \SI{0.09}{\mol} of \ce{OH-}, while normal aspirin neutralized \SI{0.33}{\mol}
            of \ce{OH-}. (See above list for calculations)
            \item The number of moles of \ce{H+} is equal to the number of moles of \ce{OH-} required to titrate the
            solution. If the number of moles of \ce{H+} were equal to the number of moles of acid, then the mass of 
            tablets would be:
            \begin{itemize}
                \item Baby Aspirin: $\SI{0.00009}{\mol}\cdot\SI{180.157}{\g\per\mol}=\SI{0.0162}{\g}=\SI{16.2}{\g}$
                \item Normal Aspirin: $\SI{0.00033}{\mol}\cdot\SI{180.157}{\g\per\mol}=\SI{0.0595}{\g}=\SI{59.5}{\g}$
            \end{itemize}
            \item The actual masses of the tablets should be \SI{81}{\mg} and \SI{325}{\mg} for the baby and normal aspirins,
            respectively. The differences between the actual masses and the calculated masses arises from the fact that
            the tablet is not purely acetylsalicylic acid. There are other compounds in the tablet that contribute to its
            mass.
        \end{enumerate}
    \section{Conclusion}
        \begin{enumerate}
            \item The baby aspirin costs $2$ times as much as normal aspirin per tablet and $8.5$ times as much per gram.
            \item I would buy normal aspirin because it is cheaper both per tablet and per gram.
            \item The biggest source of error in the lab was the fact that there were pieces of tablet that did not dissolve,
            possibly skewing titration results.
            \item As the active ingredient of aspirin is an acid, it can cause gastrointestinal irritation and/or bleeding
            if it concentrates on the stomach wall, etc. However, its blood-thinning effects are useful for people who
            have a high risk of clotting.
        \end{enumerate}
\end{document}