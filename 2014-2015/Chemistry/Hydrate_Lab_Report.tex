\documentclass[12pt]{article}

\usepackage{amsmath}

\title{Determination of the Formula of a Hydrated Salt}
\date{12 September 2014}
\author{Tarik Onalan}

\begin{document}
    \maketitle
    \section{Questions}
        1. \textit{What is the name of the salt (not the hydrate part)?} \\*
        The name of the salt is \(CaSO_{4}\) \\\\
        2. \textit{Calculate the percent water in the hydrated sample. Be sure to
        report the answer to the proper number of significant figures.} \\*
        \[\frac{0.20g}{1.00g}=0.20\to20\%\] \\\\
        3. \textit{Calculate the moles of water and the moles of anhydrous compound
        in your sample. Calculate the number of waters of hydration in the formula
        from this information.} \\*
        \[\frac{0.20g}{18.02\frac{g}{mol} H_{2}O}=0.011mol\]
        \[\frac{0.80g}{136.14\frac{g}{mol} CaSO_{4}}=0.0059mol\]
        \[\frac{0.011mol H_{2}O}{0.0059mol CaSO_{4}}\approx2\] \\
        Therefore waters of hydration: \(2\) \\\\
        4. \textit{Write the complete \& correct formula for the hydrated compound you
        started with. For examples, see the front of this sheet.} \\*
        The formula for the hydrated compound is \(CaSO_{4}\bullet{2H_{2}O}\). \\\\
        5. \textit{What effect would heating the hydrated sample for too short a time
        have on the calculated percent water? Would the calculated percent water be
        lower or higher than the actual one? Explain your answer.} \\*
        If the hydrated sample were heated for too short a time, the calculated percent
        water would be lower, because less water would appear to evaporate from the
        hydrated sample. \\\\
        6. \textit{Suppose the crucible and cover were not heated to dryness after
        being rinsed with distilled water. Would the resulting determination of the
        percent water in the hydrate be lower or higher than the actual one? Explain
        your answer.} \\*
        If the crucible were not heated to dryness after rinsing, the resulting percent
        water in the hydrate would be higher than the actual one, because the hydrate
        would have absorbed water (being a hydroscopic compound). That excess water
        would then be weighed along with the sample, and when the excess water is
        evaporated, the difference between the initial and final masses would be
        larger, inflating the percentage of water in the hydrate. \\\\
    \section{Data Table}
        Relative Mass of Hydrate with and without Water \\\\
        \begin{tabular}{|l||c|r|}
            \hline
            Initial Mass & Hydrate Mass & Dry Hydrate Mass \\
            \hline
            \(11.01g\) & \(12.01g\) & \(11.81g\) \\
            \hline
            \(0.00g\) & \(1.00g\) & \(0.80g\) \\
	 \hline
        \end{tabular}
\end{document}
