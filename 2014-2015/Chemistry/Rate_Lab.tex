\documentclass[a4paper]{article}

\usepackage{gensymb}
\usepackage[version=3]{mhchem}
\usepackage{pgfplots}
\usepackage{siunitx}

\def\mean#1{\left< #1 \right>}
\sisetup{per-mode=fraction}
\DeclareSIUnit\Molar{\textsc{m}}

\title{Analyzing Rate of Decomposition of \ce{NaBO3} in Changing Molarities of \ce{HCl}}
\date{16 January 2015}
\author{Tarik Onalan}

\begin{filecontents}{avg.dat}
    x    y    xun  yun
    1.0  377  0.025 10
    0.75 375  0.025 10
    0.50 372  0.025 10
\end{filecontents}

\begin{filecontents}{t_1.dat}
    x    y   xun  yun
    1.0  381 0.025 10
    0.75 339 0.025 10
    0.50 323 0.025 10
\end{filecontents}

\begin{filecontents}{t_2.dat}
    x    y   xun  yun
    1.0  373 0.025 10
    0.75 369 0.025 10
    0.50 429 0.025 10
\end{filecontents}

\begin{filecontents}{t_3.dat}
    x    y   xun  yun
    1.0  377 0.025 10
    0.75 418 0.025 10
    0.50 365 0.025 10
\end{filecontents}

\begin{document}
    \maketitle
    \section{Introduction}
        The objective of this lab is to understand how rate of reaction changes in
        varying environments. This lab will focus on the decomposition of sodium
        perborate (\ce{NaBO3}), the active ingredient in Efferdent denture cleaning
        tablets, in \ce{HCl} solutions of varying molarity.
        \subsection{Question}
            How does the rate of decomposition of \ce{NaBO3} change in \ce{HCl} solutions
            of varying molarity?
        \subsection{Hypothesis}
            I predict that as the molarity of the \ce{HCl} solution decreases, the
            rate of decomposition of \ce{NaBO3} will decrease, as there will be less
            solution to react with the salt.
            \centerline{\(\ce{HCl}(aq)+\ce{NaBO3}(s)\to\ce{HBO3}(aq)+\ce{NaCl}(aq)\)}
        \subsection{Variables}
            \begin{itemize}
                \item \textbf{Manipulated Variable:} Molarity of \ce{HCl} solution
                \begin{itemize}
                    \item \SI{0.50}{\Molar}
                    \item \SI{0.75}{\Molar}
                    \item \SI{1.0}{\Molar}
                \end{itemize}
                \item \textbf{Dependent Variable:} Time of decomposition of \ce{NaBO3} tablet
                \item \textbf{Controlled Variables:}
                \begin{itemize}
                    \item Type of tablet used (Efferdent)
                    \item Mass of tablet (\(\sim\SI{5.0}{\g}\))
                    \item Temperature of \ce{HCl} (\(\sim\SI{21.0}{\celsius}\))
                    \item Temperature of \ce{H2O} (\(\sim\SI{21.0}{\celsius}\))
                    \item Water used to dilute \ce{HCl} solution (distilled water)
                \end{itemize}
                \item \textbf{Trials:} Three trials per iteration of molarity
            \end{itemize}
        \subsection{Materials}
            \begin{itemize}
                \item 9 Efferdent tablets
                \item 250 mL \ce{HCl} Solution
                \item 250 mL distilled \ce{H2O}
                \item 3 beakers
                \item 2 graduated cylinders
                \item 3 timers
            \end{itemize}
        \subsection{Procedure}
            \begin{enumerate}
                \item Gather materials
                \item As necessary, heat \ce{H2O} and \ce{HCl} solutions until \SI{21.0}{\celsius}
                \item Fill beaker with \SI{25.0}{\mL} of \SI{1.0}{\Molar} \ce{HCl} solution
                \begin{itemize}
                    \item Later trials will require lower molarities; dilute solution
                        as necessary to reach desired molarities (\SI{0.75}{\Molar},
                        \SI{0.50}{\Molar})
                \end{itemize}
                \item Drop tablet into beaker, start timer
                \begin{itemize}
                    \item Make sure tablet is whole (all tablets \(\sim\SI{5.0}{\g}\))
                \end{itemize}
                \item When tablet is fully dissolved, stop timer
                \begin{itemize}
                    \item Record time taken
                    \item Reset timer
                \end{itemize}
                \item Repeat items 2-6 for trials 2-3
                \item Repeat items 2-7 for \SI{0.75}{\Molar} and \SI{0.50}{\Molar}
            \end{enumerate}
            \textbf{Note that trials on differing or changing molarities can be run simultaneously
                with the three beakers in the materials list}
    \section{Data}
        \subsection{Table}
            \begin{tabular}{|c||c|c|c|c|}
                \hline
                \bfseries{Molarity} & \bfseries{Trial 1} & \bfseries{Trial 2} & \bfseries{Trial 3} & \bfseries{Average}
                \\\hline
                \SI{0.50}{\Molar} \(\pm 0.025\) & \SI{323}{\s} \(\pm \SI{10.}{\s}\) & \SI{429}{\s} \(\pm \SI{10.}{\s}\) & \SI{365}{\s} \(\pm \SI{10.}{\s}\) & \SI{372}{\s} \(\pm \SI{10.}{\s}\)
                \\\hline
                \SI{0.75}{\Molar} \(\pm 0.025\) & \SI{339}{\s} \(\pm \SI{10.}{\s}\) & \SI{369}{\s} \(\pm \SI{10.}{\s}\) & \SI{418}{\s} \(\pm \SI{10.}{\s}\) & \SI{375}{\s} \(\pm \SI{10.}{\s}\)
                \\\hline
                \SI{1.0}{\Molar} \(\pm 0.025\) & \SI{381}{\s} \(\pm \SI{10.}{\s}\) & \SI{373}{\s} \(\pm \SI{10.}{\s}\) & \SI{377}{\s} \(\pm \SI{10.}{\s}\) & \SI{377}{\s} \(\pm \SI{10.}{\s}\)
                \\\hline
            \end{tabular}
        \subsection{Graph}
            \begin{tikzpicture}
                \begin{axis} [
                    scale=1.25,
                    title={Reaction Speed Relative to Molarity of \ce{HCl} Solution},
                    xlabel={Molarity of \ce{HCl} Solution [\si{\Molar}]},
                    ylabel={Speed of Reaction [\si{\s}]},
                    xmin=0.5, xmax=1.0,
                    ymin=0.0, ymax=500.0,
                    legend pos=south east,
                    ymajorgrids=true,
                    grid style=dashed
                ]
                    \addplot [
                        color=black,
                        mark=*
                    ] plot [
                        error bars/.cd,
                            x dir=both,
                            y dir=both,
                            x explicit,
                            y explicit
                    ] table [
                        x=x,
                        y=y,
                        x error=xun,
                        y error=yun
                    ]{avg.dat};

                    \addplot [
                        color=red,
                        mark=*
                    ] plot [
                        error bars/.cd,
                            x dir=both,
                            y dir=both,
                            x explicit,
                            y explicit
                    ] table [
                        x=x,
                        y=y,
                        x error =xun,
                        y error=yun
                    ]{t_1.dat};

                    \addplot [
                        color=green,
                        mark=*
                    ] plot [
                        error bars/.cd,
                            x dir=both,
                            y dir=both,
                            x explicit,
                            y explicit
                    ] table [
                        x=x,
                        y=y,
                        x error=xun,
                        y error=yun
                    ]{t_2.dat};

                    \addplot [
                        color=blue,
                        mark=*
                    ] plot [
                        error bars/.cd,
                            x dir=both,
                            y dir=both,
                            x explicit,
                            y explicit
                    ] table [
                        x=x,
                        y=y,
                        x error=xun,
                        y error=yun
                    ]{t_3.dat};

                    \addlegendentry{Average}
                    \addlegendentry{Trial 1}
                    \addlegendentry{Trial 2}
                    \addlegendentry{Trial 3}
                \end{axis}
            \end{tikzpicture}
        \subsection{Calculations}
            \begin{itemize}
                \item Dilution:
                    \[M_{1} \cdot V_{1} = M_{2} \cdot V_{2}\]
                    \centerline{Example\(\to\)\SI{0.75}{\Molar}}
                    \[\SI{1.0}{\Molar} \cdot \SI{25.0}{\mL} = \SI{0.75}{\Molar} \cdot x\]
                    \[x=\frac{\SI{1.0}{\Molar} \cdot \SI{25.0}{\mL}}{\SI{0.75}{\Molar}}\]
                    \[x=\SI{33.3}{\mL}\]
                \item Average:
                    \[t_{avg}=\mean{t_{i}}\]
                    \[t_{avg}=\frac{t_{0}+t_{1}+t_{2}}{3}\]
                    \centerline{t\(\to\)Trial 1}
                    \[t_{avg}=\frac{\SI{381}{\s}+\SI{373}{\s}+\SI{377}{\s}}{3}\]
                    \[t_{avg}=\SI{377}{\s}\]
            \end{itemize}
    \section{Analysis}
        My hypothesis, that the rate of reaction would decrease as molarity of the \ce{HCl}
        solution decreased, was invalid according to my data. There seemed to be very little
        correlation between the molarity of the acid and the rate of the reaction between
        \ce{HCl} and the tablets. The averages of the data could even suggest that reaction
        rate increases as the molarity of \ce{HCl} decreases. The rate of reaction went
        from an average of \SI{377}{\s} in \SI{1.0}{\Molar} \ce{HCl} to \SI{375}{\s} in
        \SI{0.75}{\Molar}, and decreasing even further to \SI{372}{\s} in \SI{0.50}{\Molar}.
        However, it is worth noting that the range of \SI{5}{\s} in these values only
        makes up a \(\sim{}1.3\%\) difference, well within the margins of error--\SI{10}{\s}--
        thus making it statistically insignificant. However, looking at the raw data shows
        an interesting \textit{lack} of a pattern. Sometimes, the solution with the lowest molarity
        had the fastest reaction. In fact, the solution with the highest molarity never had
        the fastest reaction. In trial 1 of the three molarities, for example, the lowest
        molarity solution, \SI{0.50}{\Molar} \ce{HCl}, had the fastest reaction at \SI{323}{\s}.
        This was followed by the \SI{0.75}{\Molar} solution, with a reaction length of \SI{339}{\s}.
        The slowest reaction was the \SI{1.0}{\Molar} solution, with a reaction length of \SI{381}{\s}.
        \\
        I believe that the impurities in the individual tablets was the cause of the discrepancies
        in the data. In a lab with pure salts, our class observed that the salts would react
        faster in a high concentration acid as opposed to a low concentration acid. However, the tablets
        are not pure. They contain, among other chemicals, citric acid, food coloring and odor-releasing
        chemicals. One of the important assumptions that I had to make during this lab is that tablets
        of the same brand and "model", per se, would behave the same. However, that was not necessarily
        a guarantee. Production is not perfect, and different masses of chemicals can go into different
        tablets, possibly creating a tablet-to-tablet behavior difference. I knew, during the lab, that
        the \ce{HCl} solution was uniform, as it had all been taken from a single source, and, of course,
        any aqueous impurities would most likely have spread themselves throughout the solution. Beyond
        the tablets, judging the end of the reaction was also a struggle. As the tablet begins to near
        complete dissolution, it begins to float to the surface, making it especially difficult to see,
        as the bubbles "hide" the tablet. This is why there is a considerable uncertainty in the time
        measurement, as the timer could have been stopped in a 20-second window where the tablet was
        extremely close to dissolving completely.
        \\
        In the future, students should use more pure substances instead of tablets. In this experiment,
        the tablets proved to be an uncontrolled variable that skewed results. Not every tablet was
        the same as the other, producing contradictory results. Pure substances would remove the
        variables of the production processes of tablets, leaving the reaction to take place without
        the "interference" of extraneous chemicals. The other issue I mentioned in the previous paragraph--
        judging the end of a reaction--comes down to the persistence of the student, I believe. It is
        not impossible to see the tablet as it gets smaller; the tablet stays in the water. Looking
        under the beaker instead of from the side would probably allow for better spotting of the tablet
        as it dissolves, as that would provide a view with fewer bubbles.
\end{document}
