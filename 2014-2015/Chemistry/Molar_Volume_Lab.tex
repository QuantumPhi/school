\documentclass[a4paper]{article}

\usepackage[version=3]{mhchem}
\usepackage{siunitx}

\DeclareSIUnit\atm{atm}
\DeclareSIUnit\molm{\g\per\mol}
\DeclareSIUnit\mvol{\L\per\mol}
\DeclareSIUnit\dens{\g\per\L}

\sisetup{per-mode=fraction}

\def\abs#1{\left| #1 \right|}

\title{Determining the Molar Volume of a Gas}
\date{8 December 2014}
\author{Tarik Onalan}

\begin{document}
    \maketitle
    \begin{enumerate}
        \item Calculate the theoretical number of moles of hydrogen gas produced:\\
            \[P_{total}=P_{H_{2}}+P_{H_{2}O}\]
            \[\SI{766.6}{\mmHg}=P_{H_{2}}+\SI{19.8}{\mmHg}\]
            \[P_{H_{2}}=\SI{746.8}{\mmHg}\]
            \[PV=nRT\]
            \[\SI{746.8}{\mmHg} \cdot \SI{49.4}{\mL}=n \cdot \SI{62363.8}
                {\ml\mmHg\per\mol\per\K}\cdot\SI{295}{\K}\]
            \[n=2.01 \cdot 10^{-3}\si{\mol}\]
        \item Find the vapor pressure of water at the temperature of the water bath
            in this experiment. Calculate the partial pressure of hydrogen gas produced:\\
            \[P_{total}=P_{H_{2}}+P_{H_{2}O}\]
            \[\SI{766.6}{\mmHg}=P_{H_{2}}+\SI{19.8}{\mmHg}\]
            \[P_{H_{2}}=\SI{746.8}{\mmHg}\]
        \item Use the combined gas law to convert the measured volume of hydrogen to
            the ideal volume the hydrogen gas would occupy at STP:
            \[\frac{P_{1} \cdot V_{1}}{T_{1}} = \frac{P_{2} \cdot V_{2}}{T_{2}}\]
            \[\frac{\SI{1.009}{\atm} \cdot \SI{49.4}{\ml}}{\SI{295}{\K}}=
                \frac{\SI{1}{\atm} \cdot V_{2}}{\SI{295}{\K}}\]
            \[V_{2}=\SI{49.8}{\ml}\]
        \item Divide the volume of hydrogen gas at STP by the theoretical number
            of moles of hydrogen to calculate the molar volume of hydrogen:
            \[V_{mol}=\frac{V_{2}}{n} \to \SI{24.6}{\mvol}\]
        \item Calculate the percent error in your experimental determination of
            the molar volume of hydrogen:
            \[\delta_{V_{mol}}=\frac{\abs{V_{mol}-\SI{22.4}{\mvol}}}
                {\SI{22.4}{\mvol}} \cdot 100\% \to 9.82\%\]
        \item Calculate the density of hydrogen gas and determine the percent error
            of your measurements:\\
            \[mol_{V}=V^{-1}_{mol} \to 0.0407\]
            \[\rho=mol_{V} \cdot \SI{2.02}{\molm} \to \SI{0.0822}{\dens}\]
            \[\delta_{\rho}=\frac{\abs{\rho-\SI{0.0899}{\dens}}}{\SI{0.0899}{\dens}}
                \cdot 100\% \to 8.57\%\]
        \item If a bubble leaked into the eudiometer during the experiment, what effect
            would the bubble have on the measured volume of hydrogen gas, and would
            the calculated molar volume of hydrogen be too high or too low as a result:\\
            The measured volume of hydrogen gas would be higher, meaning that the molar
            volume of hydrogen would also seem higher. This would occur because the
            molar volume of hydrogen has a direct relationship with the measured volume
            of hydrogen, meaning that if the measured volume of hydrogen increases,
            the molar volume of hydrogen also increases.
        \item If the magnesium ribbon was oxidized before the experiment, what effect
            would that have on the measured volume of hydrogen gas, and would the
            calculated molar volume of hydrogen be too high or too low as a result:\\
            The measured volume of hydrogen gas would be lower, meaning that the molar
            volume of hydrogen would also seem lower. This would occur because magnesium
            which would have otherwise reacted with the \ce{HCl} would react with
            oxygen in the air, creating a dull coating of \ce{MgO} on the magnesium
            strip. The reaction between \ce{MgO} and \ce{HCl} would then produce
            \ce{MgCl2} and \ce{H2O}, not \ce{MgCl2} and \ce{H2} as the reaction between
            \ce{Mg} and \ce{HCl} would. The molar volume, which has a direct relationship
            with the measured volume, would then decrease, because the measured volume
            of hydrogen would be lower.
    \end{enumerate}
    \centerline {
        \begin{tabular}{|c|c|}
            \hline
            \bfseries{Question} & \bfseries{Response}
            \\\hline
            1 & \(2.01 \cdot 10^{-3}\si{\mol}\)
            \\\hline
            2 & \(P_{H_{2}}=\SI{746.8}{\mmHg}\)
            \\\hline
            3 & \(V_{2}=\SI{49.8}{\ml}\)
            \\\hline
            4 & \(V_{mol}=\SI{24.6}{\mvol}\)
            \\\hline
            5 & \(\delta_{V_{mol}}=9.82\%\)
            \\\hline
            6 & \(\delta_{\rho}=8.57\%\)
            \\\hline
        \end{tabular}
    }
\end{document}
