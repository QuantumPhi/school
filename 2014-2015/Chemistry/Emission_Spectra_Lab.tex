\documentclass{article}

\usepackage{siunitx}

\title{Predicting the Elements Present in a Common Fluorescent Light Fixture}
\date{31 October 2014}
\author{Tarik Onalan}

\begin{document}
    \maketitle

    I believe that the element in the common fluorescent light is mercury. The emission
    spectra of mercury and fluorescent lights are very similar. Both have thick bands at violet
    (\(\SI{700}{\nm}+\)), blue (\(\SI{700}{\nm}-\SI{600}{\nm}\)), green (\(\SI{600}{\nm}-\SI{500}{\nm}\)),
    yellow (\(\SI{500}{\nm}-\SI{400}{\nm}\)), and red (\(\SI{400}{\nm}-\SI{300}{\nm}\)). Additionally,
    the locations of the colors on the spectrum were similar, implying that the base element of
    the fluorescent light is mercury. While these are qualitative measurements, there were no other
    tools available to measure precise data. Other difficulties, in my mind, included the fact
    that the fluorescent light was observed with the covers still on, meaning that some of the light
    emissions would be absorbed by the light cover. While the former point is difficult to solve,
    taking the covers off of a light is quite simple, and would introduce less error to the result
    by removing an uncontrolled variable.
\end{document}
