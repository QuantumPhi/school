\documentclass[a4paper]{article}

\usepackage{booktabs}
\usepackage[version=3]{mhchem}
\usepackage{pgfplots}
\usepackage{pgfplotstable}
\usepackage{siunitx}

\newcommand{\pH}{\text{pH}}
\DeclareSIUnit\Molar{\textsc{m}}
\sisetup{
    round-mode=places,
    round-precision=2
}

\title{Analyzing the Relationship Between the Acidity and the Conductivity of a Solution}
\date{23 February 2015}
\author{Tarik Onalan}

\begin{document}
    \maketitle
    \section{Purpose}
        Understand how standing \ce{H+} and/or \ce{OH-} ion concentration in a solution affects conductivity
        measured in \si{\micro\siemens}.
        \subsection{Hypothesis}
            Conductivity will decrease as \pH{} approaches 7, as the ions that would
            otherwise facilitate the conduction of electricity (\ce{H+}/\ce{OH-})
            would be neutralized.
    \section{Materials}
        \begin{itemize}
            \item \SI{150}{\mL} \SI{0.1}{\Molar} \ce{HCl}
            \item \SI{150}{\mL} \SI{0.1}{\Molar} \ce{NaOH}
            \item 1 \SI{100}{\mL} buret$+$stand
            \item 1 \SI{250}{\mL} beaker
            \item 1 \SI{50}{\mL} graduated cylinder
            \item 1 stir plate$+$bar
            \item 1 logging device
            \item 1 \pH probe
            \item 1 conductivity probe
            \item 1 USB flash disk
            \item phenolphthalein
        \end{itemize}
    \section{Procedure}
        \begin{enumerate}
            \item Turn on logger
            \item Plug in probes and USB disk
            \item Set up buret with stand
            \item Put stir plate below buret
            \item Fill graduated cylinder with \SI{50}{\mL} \ce{HCl}
            \item Fill buret with \SI{50}{\mL} \ce{NaOH}
            \item Transfer \ce{HCl} to beaker
            \item Put 2-3 drops of phenolphthalein in the beaker
            \item Calibrate probes
            \item Put beaker on top of stir plate
            \item Put the stir bar in the beaker
            \item Set the stir bar spinning at $\frac{1}{4}$ speed
            \item Open the buret so there is $\sim1$ drop \si{\per\second}
            \item Start logger
            \item Let experiment run for 50 seconds
            \item Stop logger
            \item Export data to USB disk
            \item Neutralize any remaining \ce{HCl}
            \item Turn off stir plate
            \item Remove stir bar
            \item Clean beaker
            \item Repeat $5-21$ as necessary for data collection
        \end{enumerate}
    \section{Data}
    \begin{table}[p]
        \begin{center}
            \caption{Trial 1}
            \label{trial1}
            \pgfplotstabletypeset[
                multicolumn names,
                col sep=comma,
                display columns/0/.style={
                    column name=$Time$,
                    column type={S},string type},
                display columns/1/.style={
                    column name=$\pH$,
                    column type={S},string type},
                display columns/2/.style={
                    column name=$Conductivity$,
                    column type={S}, string type},
                every head row/.style={
                    before row={\toprule},
                    after row={
                        \si\s & & \si{\micro\siemens}\\
                        \midrule}
                },
                columns/pH/.append style={
                    postproc cell content/.append style={
                        /pgfplots/table/@cell content/.add={}{$\pm0.20$}
                    }
                },
                columns/conductivity/.append style={
                    postproc cell content/.append style={
                        /pgfplots/table/@cell content/.add={}{$\pm800.$}
                    }
                },
                every last row/.style={after row=\bottomrule}
            ]{foo1.csv}
        \end{center}
    \end{table}
        \begin{table}[p]
            \begin{center}
                \caption{Trial 3}
                \label{trial2}
                \pgfplotstabletypeset[
                    multicolumn names,
                    col sep=comma,
                    display columns/0/.style={
                        column name=$Time$,
                        column type={S},string type},
                    display columns/1/.style={
                        column name=$\pH$,
                        column type={S},string type},
                    display columns/2/.style={
                        column name=$Conductivity$,
                        column type={S}, string type},
                    every head row/.style={
                        before row={\toprule},
                        after row={
                            \si\s & & \si{\micro\siemens}\\
                            \midrule}
                    },
                    columns/pH/.append style={
                        postproc cell content/.append style={
                            /pgfplots/table/@cell content/.add={}{$\pm0.20$}
                        }
                    },
                    columns/conductivity/.append style={
                        postproc cell content/.append style={
                            /pgfplots/table/@cell content/.add={}{$\pm800.$}
                        }
                    },
                    every last row/.style={after row=\bottomrule}
                ]{foo2.csv}
            \end{center}
        \end{table}
        \begin{table}[p]
            \begin{center}
                \caption{Trial 3}
                \label{trial3}
                \pgfplotstabletypeset[
                    multicolumn names,
                    col sep=comma,
                    display columns/0/.style={
                        column name=$Time$,
                        column type={S},string type},
                    display columns/1/.style={
                        column name=$\pH$,
                        column type={S},string type},
                    display columns/2/.style={
                        column name=$Conductivity$,
                        column type={S}, string type},
                    every head row/.style={
                        before row={\toprule},
                        after row={
                            \si\s & & \si{\micro\siemens}\\
                            \midrule}
                    },
                    columns/pH/.append style={
                        postproc cell content/.append style={
                            /pgfplots/table/@cell content/.add={}{$\pm0.20$}
                        }
                    },
                    columns/conductivity/.append style={
                        postproc cell content/.append style={
                            /pgfplots/table/@cell content/.add={}{$\pm800.$}
                        }
                    },
                    every last row/.style={after row=\bottomrule}
                ]{foo3.csv}
            \end{center}
        \end{table}
        \begin{table}[p]
            \begin{center}
                \caption{Average}
                \label{avg1}
                \pgfplotstabletypeset[
                    multicolumn names,
                    col sep=comma,
                    display columns/0/.style={
                        column name=$Time$,
                        column type={S},string type},
                    display columns/1/.style={
                        column name=$\pH$,
                        column type={S},string type},
                    display columns/2/.style={
                        column name=$Conductivity$,
                        column type={S}, string type},
                    every head row/.style={
                        before row={\toprule},
                        after row={
                            \si\s & & \si{\micro\siemens}\\
                            \midrule}
                    },
                    columns/pH/.append style={
                        postproc cell content/.append style={
                            /pgfplots/table/@cell content/.add={}{$\pm0.20$}
                        }
                    },
                    columns/conductivity/.append style={
                        postproc cell content/.append style={
                            /pgfplots/table/@cell content/.add={}{$\pm800.$}
                        }
                    },
                    every last row/.style={after row=\bottomrule}
                ]{avg1.csv}
            \end{center}
        \end{table}
        \begin{tikzpicture}
            \begin{axis}[
                scale=1.75,
                title={Conductivity Relative to $\pH$ of Solution},
                xlabel={Acidity [$\pH$]},
                ylabel={Conductivity [\si{\micro\siemens}]},
                legend pos=south west
            ]
                \addplot plot [
                    error bars/.cd,
                        x dir=both,
                        y dir=both,
                        x explicit,
                        y explicit
                ] table [
                    x=pH,
                    y=conductivity,
                    col sep=comma
                ]{foo1.csv};
                \addplot plot [
                    error bars/.cd,
                        x dir=both,
                        y dir=both,
                        x explicit,
                        y explicit
                ] table [
                    x=pH,
                    y=conductivity,
                    col sep=comma
                ]{foo2.csv};
                \addplot plot [
                    error bars/.cd,
                        x dir=both,
                        y dir=both,
                        x explicit,
                        y explicit
                ] table [
                    x=pH,
                    y=conductivity,
                    col sep=comma
                ]{foo3.csv};
                \addplot plot [
                    error bars/.cd,
                        x dir=both,
                        y dir=both,
                        x explicit,
                        y explicit,
                        x fixed=0.2,
                        y fixed=800
                ] table [
                    x=pH,
                    y=conductivity,
                    col sep=comma
                ]{avg1.csv};
                \addplot[
                    color=red,
                    mark=none,
                    domain=0.74:1.34
                ]{-2672.8*x^2 + 4270.6*x + 17618};
                \addlegendentry{Trial 1}
                \addlegendentry{Trial 2}
                \addlegendentry{Trial 3}
                \addlegendentry{Average}
                \addlegendentry{$-2672.8x^2+4270.6x+17618$}
            \end{axis}
        \end{tikzpicture}
    \section{Conclusion}
        My hypothesis, that the conductivity would decrease as \pH approached $7$, was
        partially correct. I was only able to test with an acidic starting point, meaning
        that I only observed the \pH approaching $7$ from $0\le{\pH}\l{7}$. However, I can
        conclude that as the \pH approaches $7$ from $0\le{\pH}\l{7}$, the conductivity decreases.
        Simply from a mathematical standpoint, the Pearson correlation coefficient---the measure of the
        "correlatedness" of any two datasets---of the \pH and conductivity returns
        $-0.95$, which is an almost exact (negative) linear relationship. This is evident
        by looking at the graph, which shows that as the \pH increased, the conductivity
        decreased. When the \pH was $\sim0.74$ (starting \pH), the conductivity was
        \SI{19348.0}{\micro\siemens}. When the \pH increased to $\sim1.34$, however,
        the conductivity decreased to \SI{18453.7}{\micro\siemens}. Of course, there are
        many more values in between, but the general trend is similar to the one described.
        \\
        There were many difficulties in the course of this lab. Among the more mundane ones,
        I accidentally tried to titrate \ce{HCl} with \ce{HCl}, which, as expected, did not
        have much effect. However, my biggest difficulty was keeping the probes calibrated,
        and it shows in my data. In my lab, I was using \SI{0.1}{\Molar} \ce{HCl} and \ce{NaOH},
        which have \pH values of $1$ and $13$, respectively. However, the \pH probe in particular
        would not hold its calibration, and in my second and third trials, the starting \pH
        of the \ce{HCl} is incorrectly reported as $0.82$ and $0.40$. Even so, this does not
        invalidate my conclusion, because there is still an obvious downward trend in the
        conductivity-\pH graphs.
        \\
        The first problem I stated is easy to fix; I need to be more attentive while I am
        at my lab station. However, the latter problem is more difficult. Lab probes---especially
        ones that are accurate---are expensive, and are most likely too expensive for my
        school to purchase. As such, the only practical way to compensate for calibration
        error right now is to carry out many trials, so that the average data may cancel
        out any random error introduced by deviation.
\end{document}
