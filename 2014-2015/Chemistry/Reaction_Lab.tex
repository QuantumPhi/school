\documentclass[a4paper]{article}

\usepackage{amsmath, amssymb}
\usepackage{booktabs}
\usepackage{graphicx}
\usepackage[version=3]{mhchem}
\usepackage{pgfplots, pgfplotstable}
\usepackage{siunitx}

\def\abs#1{\left| #1 \right|}
\def\mean#1{\left< #1 \right>}
\DeclareSIUnit\Molar{\textsc{m}}
\sisetup{
    round-mode=places,
    round-precision=3
}

\title{Analyzing the Relationship Between the Concentration of \ce{KIO3} and the
Speed of the Iodine Clock Reaction}
\date{31 March 2015}
\author{Tarik Onalan}

\begin{document}
    \maketitle
    \section{Introduction}
        \subsection{Background}
            \begin{center}
                \textbf{Iodine Clock Reaction}
                \\
                The iodine clock reaction occurs as follows
                \begin{equation}
                    \left\{
                        \begin{array}{l}
                            \ce{NaHSO3}+\ce{H2SO4}\to{}\ce{H2SO3}+\ce{NaHSO4} \\
                            \ce{KIO3}+3\ce{H2SO3}\to{}\ce{KI}+3\ce{H2SO4} \\
                            \ce{KIO3}+3\ce{H2SO4}+5\ce{KI}\to{}3\ce{K2SO4}+3\ce{H2O}+3\ce{I2} \\
                            \ce{I2}+\ce{H2SO3}+\ce{H2O}\to{}\ce{H2SO4}+2\ce{H+}+2\ce{I-} \\
                            \ce{I2}+\ce{(C6H10O5)_n}\to{}\ce{(C6H10O5)_n}\cdot\ce{I2}\; \text{complex}
                        \end{array}
                    \right.
                \end{equation}
            \end{center}
        \subsection{Purpose}
            Understand how concentration of products in a solution affects the speed
            of a reaction.
        \subsection{Hypothesis}
            The reaction will take longer to complete  as \ce{KIO3} concentration decreases,
            as there will be fewer \ce{KIO3} molecules available to react at any one time.
        \subsection{Variables}
            \textbf{Independent Variable}
            \begin{itemize}
                \item \ce{KIO3} concentration [\si\Molar]
            \end{itemize}
            \textbf{Dependent Variable}
            \begin{itemize}
                \item Reaction speed [\si\s]
            \end{itemize}
            \textbf{Controlled Variables}
            \begin{itemize}
                \item Molarity of \ce{NaHSO3} solution [\SI{0.02}{\Molar}]
                \item Volume of \ce{KIO3} solution [\SI{10.0}{\mL}]
                \item Volume of \ce{NaHSO3} solution [\SI{10.0}{\mL}]
                \item Temperature of solutions [\SI{25.2}{\celsius}]
                \item Distilled Water
            \end{itemize}
    \section{Materials}
        \begin{itemize}
            \item \SI{75}{\mL} \SI{0.02}{\Molar} \ce{KIO3}
            \item \SI{100}{\mL} \SI{0.02}{\Molar} \ce{NaHSO3}
            \item 9 $\cdot$ test tubes
            \item 2 $\cdot$ \SI{100}{\mL} beaker
            \item 2 $\cdot$ \SI{10}{\mL} graduated cylinder
            \item 1 $\cdot$ \SI{2}{\mL} pipette
            \item 1 $\cdot$ timer
            \item 1 $\cdot$ thermometer
        \end{itemize}
    \section{Procedure}
        \begin{enumerate}
            \item Pour \ce{KIO3} and \ce{NaHSO3} solutions into beakers
            \item Measure \SI{10}{\mL} \ce{KIO3} solution in graduated cylinder
            \item Measure \SI{10}{\mL} \ce{NaHSO3} solution in graduated cylinder
            \item If \ce{KIO3} solution is not \SI{10}{\mL}, fill up to \SI{10}{\mL} with distilled water
            \item Pour both solutions into test tube
            \item Start timer
            \item Wait until solution turns blue/black
            \item Stop timer
            \item Record time taken to react
            \item Repeat steps $2-9$ $2$ times with \SI{8}{\mL} and \SI{6}{\mL} \ce{KIO3} solution, respectively
            \item Repeat steps $1-10$ as necessary for data collection
        \end{enumerate}
    \newpage
    \section{Data}
        \begin{center}
            Table 1: Data
        \end{center}
        \resizebox{\linewidth}{!}{
            \pgfplotstabletypeset[
                multicolumn names,
                col sep=comma,
                display columns/0/.style={
                    column name=Molarity,
                    column type={S},string type},
                display columns/1/.style={
                    column name=Trial 1,
                    column type={S},string type},
                display columns/2/.style={
                    column name=Trial 2,
                    column type={S},string type},
                display columns/3/.style={
                    column name=Trial 3,
                    column type={S}, string type},
                columns/m/.append style={
                    postproc cell content/.append style={
                        /pgfplots/table/@cell content/.add={}{$\pm0.50$}
                    }
                },
                columns/1/.append style={
                    postproc cell content/.append style={
                        /pgfplots/table/@cell content/.add={}{$\pm0.50$}
                    }
                },
                columns/2/.append style={
                    postproc cell content/.append style={
                        /pgfplots/table/@cell content/.add={}{$\pm0.50$}
                    }
                },
                columns/3/.append style={
                    postproc cell content/.append style={
                        /pgfplots/table/@cell content/.add={}{$\pm0.50$}
                    }
                },
                columns/avg/.append style={
                    postproc cell content/.append style={
                        /pgfplots/table/@cell content/.add={}{$\pm0.50$}
                    }
                },
                every head row/.style={
                    before row={\toprule},
                    after row={
                         & \si\s & \si\s & \si\s\\
                        \midrule}
                },
                every last row/.style={after row=\bottomrule}
            ]{iodate_data.csv}
        }
        \begin{tikzpicture}
            \begin{axis}[
                scale=1.75,
                title={Reaction Speed Relative to the Molarity of \ce{KIO3} Solution},
                xlabel={Molarity [\si\Molar]},
                ylabel={Reaction Speed [\si\s]},
                legend pos=south west
            ]
                \addplot[
                    only marks
                ] plot [
                    error bars/.cd,
                        y dir=both,
                        y explicit,
                        y fixed=0.5
                ] table [
                    x=m,
                    y=avg,
                    col sep=comma
                ]{iodate_data.csv};
                \addplot[
                    color=red,
                    mark=none,
                    very thick,
                    domain=0.0105:0.0225
                ]{0.10396041/(1.23436068*x-0.00717099)+2.97172032};
                \addlegendentry{Average}
                \addlegendentry{$\frac{0.104}{1.234x-0.00717}+2.972$}
            \end{axis}
        \end{tikzpicture}
    \section{Calculations}
        \subsection{Average}
            \begin{equation}
                \mean{\si\s(\si\Molar)}
            \end{equation}
            \begin{equation}
                \frac{\displaystyle\sum_i{\si\s(\si\Molar)}}{i}
            \end{equation}
            \begin{equation}
                \frac{\si\s(\si\Molar_1)+\si\s(\si\Molar_2)+...+\si\s(\si\Molar_{i-1})+\si\s(\si\Molar_i)}{i}
            \end{equation}
        \subsection{Error and Minimization}
            \begin{center}
                Error was calculated using sum squared error, which is defined as follows:
                \begin{equation}
                    E=\displaystyle\sum_{i=1}^n{(y_i-f(x_i))^2}
                \end{equation}
                A program carried out the minimization of the above function, which entails
                tracing the derivative of the function with respect to a given parameter
                ---defined here as $k$---to the local minima of the function
                \begin{equation}
                    E'=\frac{\partial}{\partial k}(\displaystyle\sum_{i=1}^n{(y_i-f(x_i))^2})
                \end{equation}
            \end{center}
        \subsection{Line of Best Fit}
            \begin{center}
                When choosing my line of best fit, I first tried to piece together all of
                the information that I could gather without data. First I knew that my function
                $?(x)$ had to satisfy the following conditions for the following scenarios:
                \begin{equation}
                    ?(x) = \left\{
                        \begin{array}{lr}
                            \text{Zero Order:} & [A]=-kt+[A]_0 \\
                            \text{First Order:} & \ln{[A]}=-kt+\ln{[A]_0} \\
                            \text{Second Order:} & \frac{1}{[A]}=kt+\frac{1}{[A]_0} \\
                            \text{nth Order:} & \frac{1}{[A]^{n-1}}=(n-1)kt+\frac{1}{[A]_0^{n-1}}
                        \end{array}
                    \right.
                \end{equation}
                which I simplified to
                \begin{equation}
                    ?(x) = \left\{
                        \begin{array}{l}
                            f(x)=a(bx+c)+d \\
                            g(x)=a\ln(bx+c)+d \\
                            h(x)=\frac{a}{bx+c}+d
                        \end{array}
                    \right.
                \end{equation}
                I ended up having to assume that the rate of reaction was less than or
                equal to $2$ primarily due to the fact that ternary (third-order or more)
                reactions are extremely rare, and testing for one would subsequently be
                impractical.
                \\
                Judging by the appearance of the graphed data---steep slope for smaller
                $x$, small slope for larger $x$---I had an intuition that the best model
                would be the second order reaction:
                \begin{equation}
                    h(x)=\frac{a}{bx+c}+d
                \end{equation}
                Testing my prediction proved me correct, as the error was much lower
                for the function $h(x)$---representing a second order reaction---than
                for $g(x)$ or $f(x)$---representing first and zeroeth order reactions,
                respectively. These results are shown in the table below.
                \\
                \bf{Error}
                \\
                \begin{tabular}{|c|c|c|}
                    \hline
                    $f(x)$ & $g(x)$ & $h(x)$
                    \\\hline
                    $\num[round-precision=5]{1.51001666371}$ & $\num[round-precision=5]{13.9546825882}$ & $\num[round-precision=5]{6.5138436219}\cdot10^{-8}$
                    \\\hline
                \end{tabular}
            \end{center}
    \section{Conclusion}
        My hypothesis, that the iodine clock reaction would take longer the lower the
        concentration of \ce{KIO3}, was correct. The data I collected shows---throughout
        all 3 trials---that as \ce{KIO3} concentration decreases, the reaction takes
        longer to ``complete'', per se. The average, as shown by the graph, has a definite
        downward trend, starting at $(0.012, 16.58)$ and ending at $(0.02, 8.91)$. Another
        proof could be the fact that the data almost perfectly fits the line of best fit,
        with a sum squared error of $\num[round-precision=5]{6.5138436219}\cdot10^{-8}$;
        the best-fit model is a power function with an exponent of $-1$, which means that
        it is monotonically decreasing on the interval $(0,+\infty)$, so it would be prudent
        to assume that any further data points---further dilutions pushing the concentration
        to the ``left'' of the graph---would follow the model, increasing the time taken to
        react.
        \\
        The only major difficulty in this lab was making sure that the \ce{KIO3} and
        \ce{NaHSO3} solutions did not contaminate each other. I did experience minor
        issues in trial 1, molarity $0.02$ and trial 1, molarity $0.012$. In the former,
        I dropped a trace amount of \ce{KIO3} into my \ce{NaHSO3}, causing the reaction
        to proceed faster than normal. In the latter reaction, I did the opposite, dropping
        a trace amount of \ce{NaHSO3} into my \ce{KIO3}, reacting and further lowering the
        already low concentration of \ce{KIO3}. Even so, my conclusion is still valid, as the
        other trials ($2-3$) display the same inverse correlation between molarity and time
        taken for the solutions to react.
        \\
        In the future, I should make sure that all of my solutions are spaced sufficiently
        apart, so there is little likelihood that I drop solution into a graduated cylinder
        or test tube before I mean to mix the two solutions. Given that this is still
        a ``likelihood'', however, I could also run more trials and increase the number
        of test cases---both of which were 3 for this experiment---both to compensate
        for any possible execution errors, and to have a more robust dataset by which
        to make my conclusions, as I found that 3 points over 3 trials was much too
        sparse to make a certain conclusion with any level of comfort.
\end{document}
