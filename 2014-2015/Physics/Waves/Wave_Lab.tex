\documentclass[a4paper]{article}

\usepackage{amsmath, amssymb}
\usepackage{booktabs}
\usepackage{siunitx}
\usepackage{pgfplots, pgfplotstable}

\def\mean#1{\left< #1 \right>}
\sisetup{
    round-mode=places,
    round-precision=1
}

\title{Investigating the Effect of String Tension on Fundamental Frequency}
\date{18 March 2015}
\author{Tarik Onalan}

\begin{document}
    \maketitle
    \section{Introduction}
        \subsection{Purpose}
            Understand how string tension affects the fundamental frequency of the
            string.
        \subsection{Hypothesis}
            As tension increases, the fundamental frequency of the string will also
            increase.
        \subsection{Variables}
            \textbf{Independent Variable}
            \begin{itemize}
                \item Tension [\si\N]
            \end{itemize}
            \textbf{Dependent Variable}
            \begin{itemize}
                \item Fundamental frequency [\si\hertz]
            \end{itemize}
            \textbf{Controlled Variables}
            \begin{itemize}
                \item String length [\SI{0.50}{\m}]
                \item String mass [\SI{0.085}{\kg}]
                \item Type of string
            \end{itemize}
    \section{Materials}
        \begin{itemize}
            \item 1 $\cdot$ \SI{0.50}{\m} string
            \item 1 $\cdot$ pitch sensor
            \item 1 $\cdot$ spring scale
        \end{itemize}
    \section{Procedure}
        \begin{enumerate}
            \item Turn on pitch sensor
            \item Tighten string to \SI{0}{\N}
            \item ``Pluck'' string
            \item Record frequency
            \item Repeat steps 2-4 with \SI{5}{\N}, \SI{10}{\N}, and \SI{15}{\N}
            \item Repeat steps 2-5 as necessary for data collection
        \end{enumerate}
    \section{Data}
        \begin{center}
            Table 1: Raw Data:
            \pgfplotstabletypeset[
                multicolumn names,
                col sep=comma,
                display columns/0/.style={
                    column name=Tension,
                    column type={S},string type},
                display columns/1/.style={
                    column name=Trial 1,
                    column type={S},string type},
                display columns/2/.style={
                    column name=Trial 2,
                    column type={S},string type},
                display columns/3/.style={
                    column name=Trial 3,
                    column type={S}, string type},
                display columns/4/.style={
                    column name=Average,
                    column type={S}, string type},
                columns/t/.append style={
                    postproc cell content/.append style={
                        /pgfplots/table/@cell content/.add={}{$\pm0.1$}
                    }
                },
                columns/1/.append style={
                    postproc cell content/.append style={
                        /pgfplots/table/@cell content/.add={}{$\pm0.1$}
                    }
                },
                columns/2/.append style={
                    postproc cell content/.append style={
                        /pgfplots/table/@cell content/.add={}{$\pm0.1$}
                    }
                },
                columns/3/.append style={
                    postproc cell content/.append style={
                        /pgfplots/table/@cell content/.add={}{$\pm0.1$}
                    }
                },
                columns/avg/.append style={
                    postproc cell content/.append style={
                        /pgfplots/table/@cell content/.add={}{$\pm0.1$}
                    }
                },
                every head row/.style={
                    before row={\toprule},
                    after row={
                         & \si\s & \si\s & \si\s & \si\s\\
                        \midrule}
                },
                every last row/.style={after row=\bottomrule}
            ]{foobar.csv}
        \end{center}
        \begin{tikzpicture}
            \begin{axis}[
                scale=1.75,
                title={Fundamental Frequency Relative to String Tension},
                xlabel={Tension [\si\N]},
                ylabel={Fundamental Frequency [\si\hertz]},
                legend pos=south east
            ]
                \addplot[
                    color=blue,
                    only marks
                ] plot [
                    error bars/.cd,
                        x dir=both,
                        y dir=both,
                        x explicit,
                        y explicit,
                        x fixed=0.1,
                        y fixed=0.1
                ] table [
                    x=t,
                    y=avg,
                    col sep=comma
                ]{foobar.csv};
                \addplot[
                    color=red,
                    mark=none,
                    very thick,
                    domain=0:15
                ]{1.97318355*sqrt(1.50045554*x)};
                \addlegendentry{Average}
                \addlegendentry{$1.97\sqrt{1.50x}$}
            \end{axis}
        \end{tikzpicture}
    \section{Calculations}
        \subsection{Average}
            \begin{equation}
                \mean{\si\hertz(\si\N)}
            \end{equation}
            \begin{equation}
                \frac{\displaystyle\sum_i{\si\hertz(\si\N)}}{i}
            \end{equation}
            \begin{equation}
                \frac{\si\hertz(\si\N_1)+\si\hertz(\si\N_2)+...+\si\hertz(\si\N_{i-1})+\si\hertz(\si\N_i)}{i}
            \end{equation}
        \subsection{Error and Minimization}
            \begin{center}
                Error was calculated using sum squared error, which is defined as follows:
                \begin{equation}
                    E=\displaystyle\sum_{i=1}^n{(y_i-f(x_i))^2}
                \end{equation}
                A program carried out the minimization of the above function, which entails
                tracing the derivative of the function with respect to a given parameter
                $k \in \{a, b, c, d, ...\}$ to the local minima of the function
                \begin{equation}
                    E'=\frac{\partial}{\partial k}(\displaystyle\sum_{i=1}^n{(y_i-f(x_i))^2})
                \end{equation}
            \end{center}
        \subsection{Line of Best Fit}
            \begin{center}
                When choosing a model for my data, I had to keep in mind the basic
                features.  I knew that the tension and fundamental frequency were related
                by the function
                \begin{equation}
                    ?(x)=\frac{\sqrt{\frac{T}{m/L}}}{2L}
                \end{equation}
                where $T$ is tension, $m$ is mass, and $L$ is length. Because of this,
                I assumed my data would also follow a square-root function.
                \\
                I constructed my function like so:
                \begin{equation}
                    f(x)=a\sqrt{bx+c}+d
                \end{equation}
                As a control, I also constructed a linear function:
                \begin{equation}
                    g(x)=a(bx+c)+d
                \end{equation}
                As I had predicted, the square-root function yielded much less error
                when modeling the data, which is why I chose it as my model. The errors
                for the square-root and linear functions are shown below:
                \\
                \bf{Error}
                \\
                \begin{tabular}{|c|c|}
                    \hline
                    $f(x)$ & $g(x)$
                    \\\hline
                    $\num[round-precision=5]{0.00471860747464}$ & $\num[round-precision=5]{3.94833443333}$
                \end{tabular}
            \end{center}
    \section{Conclusion}
        My hypothesis, that the fundamental frequency of the string would increase as the
        tension of the string increased, was correct. The graph of the data shows
        a marked upward trend. When there was \SI{0.0}{\N} of force on the string, the
        fundamental frequency was \SI{0.0}{\hertz}. However, when the tension was increased
        to \SI{15.0}{\N}, the fundamental frequency increased to \SI{9.3}{\hertz}. Also,
        given that the data very closely matches the square-root used to model it, and
        the square root function is a monotonically increasing function, it would be
        safe to assume that the relationship shown by the data, that the fundamental
        frequency increases as string tension increases, is true.
        \\
        A difficulty I experienced with this lab was keeping the spring scale still
        while conducting a trial. While I did have it attached to an immobile object,
        the scale was prone to slippage, and would occasionally slacken the string.
        This is slightly shown in the data, where there are dips in the fundamental
        frequency (see \SI{5}{\N} trial 2, \SI{10}{\N} trial 1, \SI{15}{\N} trial 2).
        However, these errors are not large enough to nullify my conclusion.
        \\
        In the future, I would like to be able to conduct more trials, so any errors
        like the one described above could be ``averaged away'', per se. While the
        errors in my data were not particularly large, it would have been better to
        conduct more trials to see exactly where the fundamental frequency was.
\end{document}
