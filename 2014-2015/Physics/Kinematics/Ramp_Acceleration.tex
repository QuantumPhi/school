\documentclass{article}

\usepackage{array}
\usepackage{gensymb}
\usepackage{graphicx}
\usepackage{pgfplots}
\usepackage{siunitx}

\title{Determining the Effect of Ramp Incline on Acceleration}
\date{3 October 2014}
\author{Tarik Onalan}

\begin{filecontents}{data.dat}
    xVal   yVal     xDel  yDel
    0.6016 0.034414 0.075 0.003536
    1.249  0.1666   0.075 0.0019
    2.057  0.30314  0.075 0.00964
    2.981  0.4459   0.075 0.0048
    3.750  0.58386  0.075 0.00484
\end{filecontents}

\begin{document}
    \maketitle
    \section{Introduction}
        The goal of this lab was to determine the effect of ramp incline on the
        acceleration of a cart. I predict that as ramp incline increases, acceleration
        of the cart will increase linearly.
    \section{Materials}
        \begin{enumerate}
            \item 1 Cart
            \item 1 Ramp
            \item 1 Ruler
            \item 1 Vernier Logger
            \item 1 Position Tracker
            \item 1 Computer
            \item 5 Books
        \end{enumerate}
    \section{Procedure}
        \begin{enumerate}
            \item Set up Vernier box with position logger
            \item Place one book on a flat surface
            \item Indicate a constant distance on the ramp
            \item Lay one end of the ramp on the book
            \item Place position logger on the elevated end of the ramp
            \item Place cart at beginning of indicated distance
            \item Let go of cart, track acceleration of cart
            \item Record average acceleration for the cart
            \item Repeat steps 2-7, iterating the book count (\(1\to5\))
        \end{enumerate}
    \section{Diagram}
        \includegraphics[width=\textwidth]{diagram}
    \section{Data}
        \noindent\resizebox{\textwidth}{!}{
            \Huge
            \begin{tabular}{|c|c||c|c|c|c|c|c|} \hline
                Height & Slope & Trial 1 & Trial 2 & Trial 3 & Trial 4 & Trial 5 & Average \\\hline
                \SI{1.5}{\cm} & \(0.6016\degree\) & \SI{0.03172}{\m\per\second\squared} & \SI{0.03257}{\m\per\second\squared} & \SI{0.03346}{\m\per\second\squared} & \SI{0.03795}{\m\per\second\squared} & \SI{0.03637}{\m\per\second\squared} & \SI{0.034414}{\m\per\second\squared} \\\hline
                \SI{3.1}{\cm} & \(1.249\degree\) & \SI{0.1659}{\m\per\second\squared} & \SI{0.1670}{\m\per\second\squared} & \SI{0.1683}{\m\per\second\squared} & \SI{0.1647}{\m\per\second\squared} & \SI{0.1671}{\m\per\second\squared} & \SI{0.1666}{\m\per\second\squared} \\\hline
                \SI{5.1}{\cm} & \(2.057\degree\) & \SI{0.3036}{\m\per\second\squared} & \SI{0.3099}{\m\per\second\squared} & \SI{0.3007}{\m\per\second\squared} & \SI{0.3080}{\m\per\second\squared} & \SI{0.2935}{\m\per\second\squared} & \SI{0.30314}{\m\per\second\squared} \\\hline
                \SI{7.4}{\cm} & \(2.981\degree\) & \SI{0.4492}{\m\per\second\squared} & \SI{0.4431}{\m\per\second\squared} & \SI{0.4485}{\m\per\second\squared} & \SI{0.4476}{\m\per\second\squared} & \SI{0.4411}{\m\per\second\squared} & \SI{0.4459}{\m\per\second\squared} \\\hline
                \SI{9.3}{\cm} & \(3.750\degree\) & \SI{0.5887}{\m\per\second\squared} & \SI{0.5813}{\m\per\second\squared} & \SI{0.5839}{\m\per\second\squared} & \SI{0.5808}{\m\per\second\squared} & \SI{0.5846}{\m\per\second\squared} & \SI{0.58386}{\m\per\second\squared} \\\hline\hline
                \multicolumn{8}{|c|}{Uncertainty}\\\hline
                \SI{0.05}{\cm} & \(0.075\degree\) & \SI{0.003536}{\m\per\second\squared} & \SI{0.0019}{\m\per\second\squared} & \SI{0.00964}{\m\per\second\squared} & \SI{0.0048}{\m\per\second\squared} & \SI{0.00484}{\m\per\second\squared} &\\\hline
            \end{tabular}
        }\\

        \begin{tabular}{|c|c|c|c|}
            \hline
            Start Point & End Point & Length of Track & Uncertainty \\\hline
            \SI{50}{\cm} & \SI{192.2}{\cm} & \SI{142.2}{\cm} & \SI{0.1}{\cm} \\
            \hline
        \end{tabular}

        \begin{tikzpicture}
            \begin{axis}[
                scale=1.75,
                title={Acceleration Relative to Ramp Incline},
                xlabel={Slope [\(\degree\)]},
                ylabel={Acceleration [\si{\m\per\second\squared}]},
                xmin=0.0, xmax=4.0,
                ymin=0.0, ymax=0.75,
                legend pos=north west,
                ymajorgrids=true,
                grid style=dashed
            ]
                \addplot[
                    color=blue,
                    mark=*,
                ] plot [
                    error bars/.cd,
                        x dir=both,
                        y dir=both,
                        x explicit,
                        y explicit
                ] table [
                    x=xVal,
                    y=yVal,
                    x error=xDel,
                    y error=yDel
                ]{data.dat};

                \addplot[
                    color=red,
                    mark=none,
                    domain=0:4
                ]{0.171249*x-0.0575874};

                \draw[
                    red,
                    thin
                ] (axis cs:0.5266,0.030878) rectangle (axis cs:0.6766,0.03795);

                \draw[
                    red,
                    thin
                ] (axis cs:1.174,0.1647) rectangle (axis cs:1.324,0.1685);

                \draw[
                    red,
                    thin
                ] (axis cs:1.982,0.2935) rectangle (axis cs:2.132,0.31278);

                \draw[
                    red,
                    thin
                ] (axis cs:2.906,0.4411) rectangle (axis cs:3.056,0.4507);

                \draw[
                    red,
                    thin
                ] (axis cs:3.675,0.57902) rectangle (axis cs:3.825,0.5887);

                \addlegendentry{Average}
                \addlegendentry{0.171249x-0.0575874}
            \end{axis}
        \end{tikzpicture}
    \section{Analysis}
        The collected data was remarkably consistent with the linear approximation
        of the acceleration, with an \(R^{2}\) value of \(0.998\). However, there
        is one error with the data: the acceleration is predicted to be zero when
        the angle of the ramp is \(0.3363\degree\), when the expected value would
        be \(0\degree\). This could be caused by human error at cart launch (pushing
        the cart forward at launch), of which there were many during data collection,
        requiring that we repeated some trials. Obviously not all errors were
        corrected, but this is expected with human uncertainty in data collection.
        The simple fact is, however, that acceleration increases as ramp incline
        increases, which supports my prediction.
\end{document}
