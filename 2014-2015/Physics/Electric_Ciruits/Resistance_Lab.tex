\documentclass{article}

\usepackage{amsmath, amssymb}
\usepackage{circuitikz}
\usepackage{csvsimple}
\usepackage{pgfplots}
\usepackage{siunitx}

\title{Analyzing Circuit Resistance with Various Lengths of Nichrome Wire}
\date{1 December 2014}
\author{Tarik Onalan}

\def\mean#1{\left< #1 \right>}
\def\abs#1{\left| #1 \right|}
\numberwithin{equation}{subsection}

\begin{filecontents}{data.dat}
    Len  Len_Inv T_1   T_2   T_3   T_4   T_5   Avg   Err   Len_Err
    12.2 0.0820  0.620 0.621 0.619 0.622 0.622 0.621 0.002 0.05
    18.6 0.0538  0.406 0.430 0.423 0.422 0.423 0.421 0.017 0.05
    31.6 0.0316  0.232 0.231 0.229 0.234 0.233 0.232 0.003 0.05
    42.8 0.0234  0.210 0.208 0.206 0.206 0.208 0.208 0.002 0.05
    48.6 0.0206  0.182 0.181 0.180 0.179 0.181 0.181 0.002 0.05
\end{filecontents}

\begin{document}
    \maketitle
    \section{Introduction}
        \textbf{How does the length of nichrome wire in a circuit affect the current
        of the circuit?}\\
        The goal of this lab was to analyze how varying lengths of a resistor, nichrome
        wire, affected current flow in a circuit. My hypothesis is that as resistor
        length increases, current flow within the circuit will decrease, as the more
        resistance there is in a circuit, the lower the current will be. For the experiment,
        increasing lengths of nichrome wire will be inserted into a circuit; if the
        current in the circuit decreases as the length of the nichrome wire increases,
        my hypothesis will be validated. The manipulated and responding variables are
        the length of the nichrome wire and the current through the circuit, respectively.
        The controlled variables are the battery used (for power level), the voltage of
        the battery, the length of the low-resistance wire, the ammeter used, and the
        sensitivity setting used on the ammeter.

    \section{Materials}
        \begin{itemize}
            \item 1 Ammeter
            \item 1 Battery
            \item 1 Section of Low-Resistance Wire
            \item 5 Sections of Nichrome Wire (Varying lengths)
        \end{itemize}

    \section{Procedure}
        \begin{enumerate}
            \item Set up ammeter with battery and low-resistance wire:
            \begin{itemize}
                \item Positive terminal of battery connected to positive terminal
                    of ammeter
                \item Negative terminal of battery connected to negative terminal
                    of ammeter
                \item Low-resistance wire attached to one of two terminals, other
                    terminal reserved for nichrome wire
                \begin{itemize}
                    \item Note: Use \SI{1}{\A} sensitivity instead of \SI{5}{\A}
                        sensitivity
                \end{itemize}
            \end{itemize}
            \item Attach ends of nichrome wire to open terminals of battery and ammeter
            \begin{itemize}
                \item For simplicity, start with the shortest length of nichrome wire
            \end{itemize}
            \item Record amperage shown on ammeter
            \item Repeat steps 2-3 as necessary for data averaging
            \item Repeat steps 2-4 with increasing nichrome wire lengths
        \end{enumerate}

    \section{Diagram}
        \begin{circuitikz}
            \draw
                (0,0) to [battery]  (0,4)
                      to [ammeter]  (4,4) -- (4,0)
                      to [vR] (0,0)
                ;
        \end{circuitikz}

    \section{Data}
        \begin{tabular}{|c|||c|c|c|c|c||c|}
            \hline
            \bfseries Length & \bfseries Trial 1 & \bfseries Trial 2 & \bfseries Trial 3 & \bfseries Trial 4 & \bfseries Trial 5 & \bfseries Average
            \\\hline
            \SI{12.2}{\cm} & \SI{0.620}{\A} & \SI{0.621}{\A} & \SI{0.619}{\A} & \SI{0.622}{\A} & \SI{0.622}{\A} & \SI{0.621}{\A}
            \\\hline
            \SI{18.6}{\cm} & \SI{0.406}{\A} & \SI{0.430}{\A} & \SI{0.423}{\A} & \SI{0.422}{\A} & \SI{0.423}{\A} & \SI{0.421}{\A}
            \\\hline
            \SI{31.6}{\cm} & \SI{0.232}{\A} & \SI{0.231}{\A} & \SI{0.229}{\A} & \SI{0.234}{\A} & \SI{0.233}{\A} & \SI{0.232}{\A}
            \\\hline
            \SI{42.8}{\cm} & \SI{0.210}{\A} & \SI{0.208}{\A} & \SI{0.206}{\A} & \SI{0.206}{\A} & \SI{0.208}{\A} & \SI{0.208}{\A}
            \\\hline
            \SI{48.6}{\cm} & \SI{0.182}{\A} & \SI{0.181}{\A} & \SI{0.180}{\A} & \SI{0.179}{\A} & \SI{0.181}{\A} & \SI{0.181}{\A}
            \\\hline
        \end{tabular}
        \\
        \begin{tabular}{|c|c|}
            \hline
            \bfseries Length Error & \bfseries Amperage Error
            \\\hline
            0.05 & 0.002
            \\\hline
            --- & 0.017
            \\\hline
            --- & 0.003
            \\\hline
            --- & 0.002
            \\\hline
            --- & 0.002
            \\\hline
        \end{tabular}
        \\
        \begin{tikzpicture}[trim axis left, trim axis right]
            \begin{axis}[
                scale=1.75,
                title={Current Relative to Resistor Length},
                xlabel={Length [\si{\cm}]},
                ylabel={Current [\si{\A}]},
                xmin=0.0, xmax=50.0,
                ymin=0.0, ymax=0.7,
                legend pos=north east,
                ymajorgrids=true,
                grid style=dashed
            ]
                \addplot[
                    color=blue,
                    mark=*,
                    only marks
                ] plot [
                    error bars/.cd,
                        x dir=both,
                        y dir=both,
                        x explicit,
                        y explicit
                ] table [
                    x=Len,
                    y=Avg,
                    x error=Len_Err,
                    y error=Err
                ]{data.dat};

                \addplot[
                    color=red,
                    mark=none,
                    domain=0:50
                ]{6.28197/x^0.926264};

                \addlegendentry{Average}
                \addlegendentry{\(6.28197x^{-0.926264}\)}
            \end{axis}
        \end{tikzpicture}
        \\
        \begin{tikzpicture}
            \begin{axis}[
                scale=1.75,
                title={Current Relative to Inverse of Resistor Length},
                xlabel={\(\frac{1}{Length [\si{\cm}]}\)},
                ylabel={Current [\si{\A}]},
                xmin=0.0, xmax=0.1,
                ymin=0.0, ymax=0.7,
                x dir=reverse,
                legend pos=north east,
                ymajorgrids=true,
                grid style=dashed
            ]
                \addplot[
                    color=blue,
                    mark=*,
                    only marks
                ] plot [
                    error bars/.cd,
                    x dir=both,
                    y dir=both,
                    x explicit,
                    y explicit
                ] table [
                    x=Len_Inv,
                    y=Avg,
                    y error=Err
                ]{data.dat};

                \addplot[
                    color=red,
                    mark=none,
                    domain=0:0.1
                ]{7.2379*x+0.0265815};

                \addlegendentry{Average}
                \addlegendentry{\(7.2379x+0.0265815\)}
            \end{axis}
        \end{tikzpicture}

        \section{Calculations}
            \centerline{For below calculations, \(t_{i}=t_{1}\)}
            \subsection{Average}
                \begin{equation} \label{eq:avg}
                    t_{avg}=\mean{t_{i}}
                \end{equation}
                \centerline{where \(t_{i}\) is the set of results for wire \(i\).
                Therefore,}
                \begin{equation}
                    t_{avg}=\mean{[0.620,0.621,0.619,0.622,0.622]}=0.621
                \end{equation}

            \subsection{Error}
                \begin{equation} \label{eq:err}
                    t_{err}=\pm\abs{t_{avg}-t_{E}}
                \end{equation}
                \centerline{where \(t_{E}\) is the element of set \(t_{i}\) with
                the highest deviation from \(t_{avg}\). Therefore,}
                \begin{equation}
                    t_{err}=0.621-0.619=\pm0.002
                \end{equation}

        \section{Conclusion}
            It can be observed through the data that the current through the example
            circuit decreased as the length of the nichrome wire increased. When
            the nichrome wire was \SI{12.2}{\cm} long, the average current was
            \SI{0.621}{\A}. However, when the length increased to \SI{18.6}{\cm},
            the current decreased to \SI{0.421}{\A} This was to be expected, as we
            observed through tests with bulbs that as bulbs were added to the circuit,
            resistance increased, lowering the current of the circuit. In this case,
            however, bulbs are replaced by nichrome wire, and the number of bulbs
            is analogous to the length of the wire. The graphs show the inverse
            relationship between resistor length and current; the best-fit graph
            is nearly equivalent to an inverse function: \(y=ax^{-1}\) and
            \(y=6.28197x^{-0.926264}\). This validates my prediction that current
            would decrease as resistance increased.\\
            One of the major difficulties with the lab was getting an accurate length
            for the nichrome wire. As our group used longer lengths of wire, the wire
            curved more, making accurate measurement of its length more difficult. This
            is slightly reflected in the graph, with more deviation from the best-fit
            line as the length of the wire increased.\\
            The goal of this lab is to understand how resistance affects current flow
            in a circuit. Noting that, the length of nichrome wire in a circuit seems
            less important. Instead, the effective \textit{resistance} of the wire
            seems more important. So instead of using resistors of inconsistent resistance,
            it would be simpler to use resistors of constant resistance, and simply
            add more resistors in series to increase the resistance. This would remove
            the variable of the length of the wire, removing a source of possible error.
\end{document}
