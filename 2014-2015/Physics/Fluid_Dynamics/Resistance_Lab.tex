\documentclass[a4paper]{article}

\usepackage{amsmath, amssymb}
\usepackage{booktabs}
\usepackage[version=3]{mhchem}
\usepackage{siunitx}
\usepackage{pgfplots, pgfplotstable}

\def\mean#1{\left< #1 \right>}
\sisetup{
    round-mode=figures,
    round-precision=3
}

\title{Investigating Effect of Resistance on the Net Force on a Falling Object}
\date{3 March 2015}
\author{Tarik Onalan}

\begin{document}
    \maketitle
    \section{Introduction}
        \subsection{Purpose}
            Understand how density affects a falling object by comparing how a
            steel ball bearing falls in air and in water.
        \subsection{Hypothesis}
            As resistance increases, the net force on the ball bearing will decrease,
            as more fluid will work against the force of gravity.
        \subsection{Variables}
            \textbf{Independent Variable}
            \begin{itemize}
                \item Resistance
            \end{itemize}
            \textbf{Dependent Variable}
            \begin{itemize}
                \item Net force ($F_{net}$)
                \begin{itemize}
                    \item Acceleration
                \end{itemize}
            \end{itemize}
            \textbf{Controlled Variables}
            \begin{itemize}
                \item Drop height in air (\SI{100.0}{\cm})
                \item Drop height in resistance medium (\SI{25.0}{\cm})
                \item Resistance medium
                \item Object mass$+$size
            \end{itemize}
    \section{Materials}
        \begin{itemize}
            \item \SI{1.00}{\L} \ce{H2O}
            \item 1 $\cdot$ \SI{1.0}{\L} graduated cylinder
            \item 1 $\cdot$ metre-stick
            \item 1 $\cdot$ scale
            \item 1 $\cdot$ timer
            \item 1 $\cdot$ ball bearing
        \end{itemize}
    \section{Procedure}
        \begin{enumerate}
            \item Measure, record mass and diameter of ball bearing
            \item Suspend top of ball bearing from \SI{100.0}{\cm}
            \item Drop ball, start timer
            \item Stop timer when ball hits ``ground''
            \item Record time taken to fall
            \item Repeat steps 3-6 as necessary for data collection
            \item Fill graduated cylinder with \SI{1.0}{\L} water
            \item Suspend top of ball bearing at surface of water
            \item Drop ball, start timer
            \item Stop timer when ball hits ``ground''
            \item Record time taken to fall
            \item Repeat steps 8-12 as necessary for data collection
        \end{enumerate}
    \section{Data}
        \begin{center}
            Table 1: Raw Data
            \resizebox{\linewidth}{!}{
                \pgfplotstabletypeset[
                    multicolumn names,
                    col sep=comma,
                    display columns/0/.style={
                        column name=Medium,
                        column type={S},string type},
                    display columns/1/.style={
                        column name=Trial 1,
                        column type={S},string type},
                    display columns/2/.style={
                        column name=Trial 2,
                        column type={S},string type},
                    display columns/3/.style={
                        column name=Trial 3,
                        column type={S}, string type},
                    display columns/4/.style={
                        column name=Trial 4,
                        column type={S}, string type},
                    display columns/5/.style={
                        column name=Trial 5,
                        column type={S}, string type},
                    display columns/6/.style={
                        column name=Average,
                        column type={S}, string type},
                    columns/1/.append style={
                        postproc cell content/.append style={
                            /pgfplots/table/@cell content/.add={}{$\pm0.05$}
                        }
                    },
                    columns/2/.append style={
                        postproc cell content/.append style={
                            /pgfplots/table/@cell content/.add={}{$\pm0.05$}
                        }
                    },
                    columns/3/.append style={
                        postproc cell content/.append style={
                            /pgfplots/table/@cell content/.add={}{$\pm0.05$}
                        }
                    },
                    columns/4/.append style={
                        postproc cell content/.append style={
                            /pgfplots/table/@cell content/.add={}{$\pm0.05$}
                        }
                    },
                    columns/5/.append style={
                        postproc cell content/.append style={
                            /pgfplots/table/@cell content/.add={}{$\pm0.05$}
                        }
                    },
                    columns/avg/.append style={
                        postproc cell content/.append style={
                            /pgfplots/table/@cell content/.add={}{$\pm0.05$}
                        }
                    },
                    every head row/.style={
                        before row={\toprule},
                        after row={
                             & \si\s & \si\s & \si\s & \si\s & \si\s & \si\s\\
                            \midrule}
                    },
                    every last row/.style={after row=\bottomrule}
                ]{foo.csv}
            }
            \\\hrulefill\\
            Table 2: Acceleration and Net Force
            \\
            \pgfplotstabletypeset[
                multicolumn names,
                col sep=comma,
                display columns/0/.style={
                    column name=Medium,
                    column type={S},string type},
                display columns/1/.style={
                    column name=$a$,
                    column type={S},string type},
                display columns/2/.style={
                    column name=$F_{net}$,
                    column type={S},string type},
                every head row/.style={
                    before row={\toprule},
                    after row={
                        & \si{\m\per\s\squared} & \si\N \\
                        \midrule}
                },
                every last row/.style={after row=\bottomrule}
            ]{bar.csv}
        \end{center}
    \section{Calculations}
        \subsection{Average}
            \begin{center}
                Given that $\si\s(r)$ is the function of time taken to drop given
                resistance:
                \begin{equation}
                    \mean{\si\s(r)}
                \end{equation}
                \begin{equation}
                    \frac{\displaystyle\sum_i{\si\s(r)}}{i}
                \end{equation}
                \begin{equation}
                    \frac{\si\s(r_1)+\si\s(r_2)+...+\si\s(r_{i-1})+\si\s(r_i)}{i}
                \end{equation}
            \end{center}
        \subsection{Acceleration and Force}
            \begin{center}
                Acceleration can be calculated from the distance traveled and the
                taken to travel as follows:
                \begin{equation}
                    a=\frac{2d}{t^2}
                \end{equation}
                Let $a_g$ equal the acceleration in air and $a_l$ equal the acceleration
                in the water. Solving for $a_{g,l}$ with their respective parameters
                yields:
                \begin{equation}
                    a_g=\frac{2*100.0}{0.45^2}=\SI{987}{\cm\per\s\squared}=\SI{9.87}{\m\per\s\squared}
                \end{equation}
                \begin{equation}
                    a_l=\frac{2*25.0}{0.25^2}=\SI{800}{\cm\per\s\squared}=\SI{8.00}{\m\per\s\squared}
                \end{equation}
                From there, calculating $F_{net}$ only requires multiplying the acceleration
                by the mass of the ball bearing:
                \begin{equation}
                    F_g=a_g*0.0079=\SI{0.078}{\N}
                \end{equation}
                \begin{equation}
                    F_l=a_l*0.0079=\SI{0.063}{\N}
                \end{equation}
            \end{center}
    \section{Conclusion}
        My hypothesis, that net force on an object will decrease as the density of
        a fluid increases, is correct. In air, the net force of gravity on the ball
        bearing was \SI{0.078}{\N}, while in water, the net force was \SI{0.063}{\N}.
        This is proportional to the acceleration, which was \SI{9.87}{\m\per\s\squared}
        in air, but dropped to \SI{8.00}{\m\per\s\squared} in water.
        \\
        A difficulty with this lab was the fact that, being human, I could not perfectly
        time how long it took for the ball bearing to drop. There was obvious error;
        acceleration in air was reported as being around \SI{9.87}{\m\per\s\squared},
        while it should have been $\leq$ \SI{9.81}{\m\per\s\squared}. However, given
        that the collected data was rather consistent---standard deviations of $0.037$
        and $0.031$ for air and water, respectively---it would be safe to assume that
        my hypothesis holds true, as the other data points adhere to the same result.
        \\
        In the future, I could conduct experiments with more mediums to get more information
        before making a conclusion. While I did try to compensate for my lack of variation
        with an increased trial count, I did not feel comfortable making a conclusion
        with only two manipulations.
\end{document}
