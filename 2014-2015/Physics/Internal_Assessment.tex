\documentclass[aip,jmp,amsmath,amssymb,reprint,author-numerical]{revtex4-1}

\usepackage{bm}
\usepackage{booktabs}
\usepackage{dcolumn}
\usepackage{graphicx}
\usepackage{pgfplots}
\usepackage{pgfplotstable}
\usepackage{siunitx}

\sisetup{
    round-mode=places,
    round-precision=2
}

\begin{document}
    \title[Addition of a Magnet to Newton's Cradle]{Addition of a Magnet to Newton's Cradle}

    \author{Tarik Onalan}
    \affiliation{Physics Department, SCHOOL, LOCATION, PLANET EARTH}

    \date{22 June 2015}

    \begin{abstract}
        Newton's Cradle is commonly used to demonstrate the conservation of momentum and energy;
        provided there is no outside influence on the system, an ideal Newton's Cradle can continue
        moving indefinitely. However, in reality, this is not the case. Energy is always lost to an
        outside force---friction---in an imperfect system, which is why there are no perpetual motion
        machines of the third kind in existence. In this paper, we investigate the opposite phenomenon;
        the effect of adding energy to a Newton's Cradle.
    \end{abstract}

    %--------------------------------------------------------------------------------------------------%

    \maketitle

    %--------------------------------------------------------------------------------------------------%

    \section{\label{sec:intro}Introduction}

        Putting a magnet into a Newton's Cradle provides a way to add energy to the system. In this
        paper, we will compare the results of various different experiments: changing the number of
        balls in a standard Newton's Cradle, changing the beginning height in a standard Newton's
        Cradle, changing the number of balls in a Newton's Cradle with a magnet, and changing the
        beginning height in a Newton's Cradle with a magnet.

    %--------------------------------------------------------------------------------------------------%

    \section{\label{sec:theory}Theory}

        Let us define the relevant equations in a standard Newton's Cradle:
        \begin{equation}
            F_{net}=m(a-\mu g)
        \end{equation}
        \begin{equation}
            p=mv
        \end{equation}
        \begin{equation}
            KE=\frac{1}{2}mv^2
        \end{equation}
        \begin{equation}
            PE=mgh
        \end{equation}
        \begin{equation}
            m_{1}v_{1}=m_{2}v_{2}
        \end{equation}
        \begin{equation}
            KE_{1}+PE_{1}=KE_{2}+PE_{2}
        \end{equation}

        These are all standard equations, taught in physics classes in schools. However, we are not
        just investigating a standard Newton's Cradle; we are also observing how adding energy through
        a magnet affects the system. In that case, let us define the additional equations we will need
        for the system:
        \begin{equation}
            F=\frac{\mu q_{m1}q_{m2}}{4\pi r^2}
        \end{equation}

        Our approximate equation for the force in a Newton's Cradle with a magnet is then the following:
        \begin{equation}
            F_{net}=m(a-\mu g) + \frac{\mu q_{m1}q_{m2}}{4\pi r^2}
        \end{equation}

    %--------------------------------------------------------------------------------------------------%

    \section{\label{sec:expset}Experimental Setup}

        We will investigate the following experiments:
        \begin{enumerate}
            \item How does the starting height of the ball bearing in the standard Newton's cradle
                affect the energy of the system?
            \item How does changing the amount of standing ball bearings in the standard Newton's cradle
                affect the energy of the system?
            \item How does changing the starting height of the magnet in the ``magnetic'' Newton's
                cradle affect the energy of the system?
            \item How does changing the amount of standing ball bearings in the ``magnetic'' Newton's
                cradle affect the energy of the system?
        \end{enumerate}

        For experiments $(1)$ and $(3)$, the setup is rather similar. In experiment $(1)$, the ball
        bearing will be dropped from varying heights, and the energy of the rebounding ball bearing
        will be measured; experiment $(3)$ will only see the dropped ball bearing replaced by a
        spherical neodymium magnet.

        Experiments $(2)$ and $(4)$ also have similar setups. In experiment $(2)$, the number of ball
        bearings between the dropped ball bearing and the rebounding ball bearing (at the end of the
        stack), will be changed. The ball bearings will be kept in place before collision by a pair of
        toothpicks. In experiment $(4)$, the dropped ball bearing is again replaced by a spherical
        neodymium magnet.

    %--------------------------------------------------------------------------------------------------%

    \section{\label{sec:result}Results and Data Analysis}

        \begin{center}
            Table 1: Standard Cradle, Changing Height
            \resizebox{\linewidth}{!}{
                \pgfplotstabletypeset[
                    multicolumn names,
                    col sep=comma,
                    display columns/0/.style={
                        column name=Height,
                        column type={S},string type},
                    display columns/1/.style={
                        column name=Trial 1,
                        column type={S},string type},
                    display columns/2/.style={
                        column name=Trial 2,
                        column type={S},string type},
                    display columns/3/.style={
                        column name=Trial 3,
                        column type={S}, string type},
                    display columns/4/.style={
                        column name=Trial 4,
                        column type={S}, string type},
                    display columns/5/.style={
                        column name=Trial 5,
                        column type={S}, string type},
                    display columns/6/.style={
                        column name=Average Energy,
                        column type={S}, string type},
                    columns/1/.append style={
                        postproc cell content/.append style={
                            /pgfplots/table/@cell content/.add={}{$\pm0.05$}
                        }
                    },
                    columns/2/.append style={
                        postproc cell content/.append style={
                            /pgfplots/table/@cell content/.add={}{$\pm0.01$}
                        }
                    },
                    columns/3/.append style={
                        postproc cell content/.append style={
                            /pgfplots/table/@cell content/.add={}{$\pm0.01$}
                        }
                    },
                    columns/4/.append style={
                        postproc cell content/.append style={
                            /pgfplots/table/@cell content/.add={}{$\pm0.01$}
                        }
                    },
                    columns/5/.append style={
                        postproc cell content/.append style={
                            /pgfplots/table/@cell content/.add={}{$\pm0.01$}
                        }
                    },
                    columns/e_avg/.append style={
                        postproc cell content/.append style={
                            /pgfplots/table/@cell content/.add={}{$\pm0.01$}
                        }
                    },
                    every head row/.style={
                        before row={\toprule},
                        after row={
                            \si\cm & \si\joule & \si\joule & \si\joule & \si\joule & \si\joule & \si\joule\\
                            \midrule}
                    },
                    every last row/.style={after row=\bottomrule}
                ]{IA_T1.dat}
            }
            \\[0.5cm]
            Table 2: Standard Cradle, Changing Standing Bearings
            \resizebox{\linewidth}{!}{
                \pgfplotstabletypeset[
                    multicolumn names,
                    col sep=comma,
                    display columns/0/.style={
                        column name=Count,
                        column type={S},string type},
                    display columns/1/.style={
                        column name=Trial 1,
                        column type={S},string type},
                    display columns/2/.style={
                        column name=Trial 2,
                        column type={S},string type},
                    display columns/3/.style={
                        column name=Trial 3,
                        column type={S}, string type},
                    display columns/4/.style={
                        column name=Trial 4,
                        column type={S}, string type},
                    display columns/5/.style={
                        column name=Trial 5,
                        column type={S}, string type},
                    display columns/6/.style={
                        column name=Average Energy,
                        column type={S}, string type},
                    columns/1/.append style={
                        postproc cell content/.append style={
                            /pgfplots/table/@cell content/.add={}{$\pm0.05$}
                        }
                    },
                    columns/2/.append style={
                        postproc cell content/.append style={
                            /pgfplots/table/@cell content/.add={}{$\pm0.01$}
                        }
                    },
                    columns/3/.append style={
                        postproc cell content/.append style={
                            /pgfplots/table/@cell content/.add={}{$\pm0.01$}
                        }
                    },
                    columns/4/.append style={
                        postproc cell content/.append style={
                            /pgfplots/table/@cell content/.add={}{$\pm0.01$}
                        }
                    },
                    columns/5/.append style={
                        postproc cell content/.append style={
                            /pgfplots/table/@cell content/.add={}{$\pm0.01$}
                        }
                    },
                    columns/e_avg/.append style={
                        postproc cell content/.append style={
                            /pgfplots/table/@cell content/.add={}{$\pm0.01$}
                        }
                    },
                    every head row/.style={
                        before row={\toprule},
                        after row={
                             & \si\joule & \si\joule & \si\joule & \si\joule & \si\joule & \si\joule\\
                            \midrule}
                    },
                    every last row/.style={after row=\bottomrule}
                ]{IA_T2.dat}
            }
            \\[0.5cm]
            Table 3: Magnetic Cradle, Changing Height
            \resizebox{\linewidth}{!}{
                \pgfplotstabletypeset[
                    multicolumn names,
                    col sep=comma,
                    display columns/0/.style={
                        column name=Height,
                        column type={S},string type},
                    display columns/1/.style={
                        column name=Trial 1,
                        column type={S},string type},
                    display columns/2/.style={
                        column name=Trial 2,
                        column type={S},string type},
                    display columns/3/.style={
                        column name=Trial 3,
                        column type={S}, string type},
                    display columns/4/.style={
                        column name=Trial 4,
                        column type={S}, string type},
                    display columns/5/.style={
                        column name=Trial 5,
                        column type={S}, string type},
                    display columns/6/.style={
                        column name=Average,
                        column type={S}, string type},
                    columns/1/.append style={
                        postproc cell content/.append style={
                            /pgfplots/table/@cell content/.add={}{$\pm0.05$}
                        }
                    },
                    columns/2/.append style={
                        postproc cell content/.append style={
                            /pgfplots/table/@cell content/.add={}{$\pm0.01$}
                        }
                    },
                    columns/3/.append style={
                        postproc cell content/.append style={
                            /pgfplots/table/@cell content/.add={}{$\pm0.01$}
                        }
                    },
                    columns/4/.append style={
                        postproc cell content/.append style={
                            /pgfplots/table/@cell content/.add={}{$\pm0.01$}
                        }
                    },
                    columns/5/.append style={
                        postproc cell content/.append style={
                            /pgfplots/table/@cell content/.add={}{$\pm0.01$}
                        }
                    },
                    columns/e_avg/.append style={
                        postproc cell content/.append style={
                            /pgfplots/table/@cell content/.add={}{$\pm0.01$}
                        }
                    },
                    every head row/.style={
                        before row={\toprule},
                        after row={
                            \si\cm & \si\joule & \si\joule & \si\joule & \si\joule & \si\joule & \si\joule\\
                            \midrule}
                    },
                    every last row/.style={after row=\bottomrule}
                ]{IA_T3.dat}
            }
            \\[0.5cm]
            Table 4: Magnetic Cradle, Changing Standing Bearings
            \resizebox{\linewidth}{!}{
                \pgfplotstabletypeset[
                    multicolumn names,
                    col sep=comma,
                    display columns/0/.style={
                        column name=Count,
                        column type={S},string type},
                    display columns/1/.style={
                        column name=Trial 1,
                        column type={S},string type},
                    display columns/2/.style={
                        column name=Trial 2,
                        column type={S},string type},
                    display columns/3/.style={
                        column name=Trial 3,
                        column type={S}, string type},
                    display columns/4/.style={
                        column name=Trial 4,
                        column type={S}, string type},
                    display columns/5/.style={
                        column name=Trial 5,
                        column type={S}, string type},
                    display columns/6/.style={
                        column name=Average Energy,
                        column type={S}, string type},
                    columns/1/.append style={
                        postproc cell content/.append style={
                            /pgfplots/table/@cell content/.add={}{$\pm0.05$}
                        }
                    },
                    columns/2/.append style={
                        postproc cell content/.append style={
                            /pgfplots/table/@cell content/.add={}{$\pm0.01$}
                        }
                    },
                    columns/3/.append style={
                        postproc cell content/.append style={
                            /pgfplots/table/@cell content/.add={}{$\pm0.01$}
                        }
                    },
                    columns/4/.append style={
                        postproc cell content/.append style={
                            /pgfplots/table/@cell content/.add={}{$\pm0.01$}
                        }
                    },
                    columns/5/.append style={
                        postproc cell content/.append style={
                            /pgfplots/table/@cell content/.add={}{$\pm0.01$}
                        }
                    },
                    columns/e_avg/.append style={
                        postproc cell content/.append style={
                            /pgfplots/table/@cell content/.add={}{$\pm0.01$}
                        }
                    },
                    every head row/.style={
                        before row={\toprule},
                        after row={
                             & \si\joule & \si\joule & \si\joule & \si\joule & \si\joule & \si\joule\\
                            \midrule}
                    },
                    every last row/.style={after row=\bottomrule}
                ]{IA_T4.dat}
            }
        \end{center}

        \begin{tikzpicture}
            \begin{axis}[
                title={Energy Relative to Starting Height (Standard)},
                xlabel={Height [\si\cm]},
                ylabel={Energy [\si\joule]},
                legend pos=north west,
                ymajorgrids=true,
                grid style=dashed
            ]
                \addplot [
                    color=blue,
                    only marks,
                    mark=*
                ] plot [
                    error bars/.cd,
                        x dir=both,
                        y dir=both,
                        x fixed=0.05,
                        y fixed=0.01
                ] table [
                    x=h,
                    y=e_avg,
                    col sep=comma
                ]{IA_T1.dat};

                \addplot [thick, red] table [y={create col/linear regression={y=e_avg}},col sep=comma]{IA_T1.dat};

                \addlegendentry{Average}
                \addlegendentry{$\pgfmathprintnumber{\pgfplotstableregressiona} \cdot x \pgfmathprintnumber[print sign]{\pgfplotstableregressionb}$}
            \end{axis}
        \end{tikzpicture}

        \begin{tikzpicture}
            \begin{axis}[
                title={Energy Relative to Standing Bearings (Standard)},
                xlabel={Bearings},
                ylabel={Energy [\si\joule]},
                xtick={2,3,...,5},
                legend pos=north west,
                ymajorgrids=true,
                grid style=dashed
            ]
                \addplot [
                    color=blue,
                    only marks,
                    mark=*
                ] plot [
                    error bars/.cd,
                        y dir=both,
                        y fixed=0.01
                ] table [
                    x=n,
                    y=e_avg,
                    col sep=comma
                ]{IA_T2.dat};

                \addplot [thick, red] table [y={create col/linear regression={y=e_avg}},col sep=comma]{IA_T2.dat};

                \addlegendentry{Average}
                \addlegendentry{$\pgfmathprintnumber{\pgfplotstableregressiona} \cdot x \pgfmathprintnumber[print sign]{\pgfplotstableregressionb}$}
            \end{axis}
        \end{tikzpicture}

        \begin{tikzpicture}
            \begin{axis}[
                title={Energy Relative to Starting Height (Magnetic)},
                xlabel={Height [\si\cm]},
                ylabel={Energy [\si\joule]},
                legend pos=north west,
                ymajorgrids=true,
                grid style=dashed
            ]
                \addplot [
                    color=blue,
                    only marks,
                    mark=*
                ] plot [
                    error bars/.cd,
                        x dir=both,
                        y dir=both,
                        x fixed=0.05,
                        y fixed=0.01
                ] table [
                    x=h,
                    y=e_avg,
                    col sep=comma
                ]{IA_T3.dat};

                \addplot [thick, red] table [y={create col/linear regression={y=e_avg}},col sep=comma]{IA_T3.dat};

                \addlegendentry{Average}
                \addlegendentry{$\pgfmathprintnumber{\pgfplotstableregressiona} \cdot x \pgfmathprintnumber[print sign]{\pgfplotstableregressionb}$}
            \end{axis}
        \end{tikzpicture}

        \begin{tikzpicture}
            \begin{axis}[
                title={Energy Relative to Standing Bearings (Magnetic)},
                xlabel={Bearings},
                ylabel={Energy [\si\joule]},
                xtick={2,3,...,5},
                legend pos=north west,
                ymajorgrids=true,
                grid style=dashed
            ]
                \addplot [
                    color=blue,
                    only marks,
                    mark=*
                ] plot [
                    error bars/.cd,
                        y dir=both,
                        y fixed=0.01
                ] table [
                    x=n,
                    y=e_avg,
                    col sep=comma
                ]{IA_T4.dat};

                \addplot [thick, red] table [y={create col/linear regression={y=e_avg}},col sep=comma]{IA_T4.dat};

                \addlegendentry{Average}
                \addlegendentry{$\pgfmathprintnumber{\pgfplotstableregressiona} \cdot x \pgfmathprintnumber[print sign]{\pgfplotstableregressionb}$}
            \end{axis}
        \end{tikzpicture}

        Above are all of the data that I collected. It immediately becomes clear that the Newton's
        Cradle with a magnet has more energy than the standard Newton's Cradle.  A couple of
        interesting trends in the data can be seen. As the number of standing ball-bearings in the
        Newton's Cradle (with the magnet) increases, the ending energy of the rebounding ball also
        increases.

        This can be explained using equation $(7)$; as the number of standing ball bearings increases,
        the distance between the magnet and the rebounding ball bearing also increases. Magnetic force
        roughly follows the inverse-square law, meaning that as the distance between the objects
        increases, the force will decrease. Also important to note is the fact that the normal ball
        bearing is paramagnetic, meaning that its magnetic field will be proportional to the magnetic
        field strength at a given point. We can then observe that there are two terms in equation $(7)$
        with inverse correlation to the distance between the magnet and the ball bearing: $r$ and
        $q_{m2}$ (the magnetic field strength of the ball bearing). Knowing all this, we can see that
        the ``pulling'' force exerted on the rebounding ball bearing will be less as the number of
        ball bearings increases, as the distance between the ball bearing and the magnet will increase.

        Another interesting trend is that while the energy of the rebounding ball bearing is higher with
        the magnet, the height only slightly correlates with the energy of the rebounding bearing.
        The correlation is much more pronounced in the standard Newton's Cradle, where the starting
        height is the only source of energy for the system, but with a magnet, there is another force
        acting on the system that compensates for the lack of starting height.

    %--------------------------------------------------------------------------------------------------%

    \section{\label{sec:concl}Conclusion}

        Though theoretically, the following is mathematically true:

        \begin{itemize}
            \item putting a magnet in a Newton's Cradle adds energy to the system
            \item the energy of the rebounding ball in the magnet-based Newton's Cradle is slightly
                dependent on the starting height
            \item the energy of the rebounding ball in the magnet-based Newton's Cradle is dependent
                on the number of standing ball-bearings between the magnet and the rebounding ball at
                collision
        \end{itemize}

        These conclusions are not supported by my experimental evidence. The error in my data is too
        large to allow for an accurate judgment of the above claims. In some cases, the error itself
        is over twice as large as the measured data ($0.5*10^{-2}\pm 1*10^{-2}$), which is very
        inaccurate, and, more importantly, physically impossible, as it would imply that the ball bearings
        could have negative energy, which is not possible.

\end{document}